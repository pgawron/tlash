\chapter{Setup for GNU/Linux and UNIX}
\label{chapter:setup-linux}



This chapter discusses how to obtain, configure, compile, and install
\libflamens in GNU/Linux and UNIX-like environments.



\section{Before obtaining \libflame}

We encourage new users to read this section before proceeding to download
the \libflame source code.

\subsection{System software requirements}

\index{\libflame!for GNU/Linux and UNIX!software requirements}

Before you attempt to build \libflamens, be sure you have the following
software tools:
\begin{itemize}
\item
{\bf GNU/Linux or UNIX.}
\libflame should compile under a wide variety of GNU/Linux
distributions\footnote{\libflame has been known to compile successfully under
cygwin.
However, cygwin is not an environment in which we routinely test our software.
If this is your preferred environment, we welcome you to give it a try,
even if we will not be able to provide support.}
and also on any of the mainstream flavors of UNIX, provided that a somewhat
sane development environment is present.
\item
{\bf GNU tools.}
We strongly recommend the availability of a GNU development
environment.
If a full GNU environment is not present, then at the very least we
absolutely require that reasonably recent versions of GNU \make
(3.79 or later) and GNU \bash (2.0 or later) are installed and
specified in the user's \PATH shell environment variable.\footnote{On
some UNIX systems, such as AIX and Solaris, GNU \make may be named
\gmake while the older UNIX/BSD implementation retains the name \makens.
In these environments, the user must be sure to invoke \gmakens, as the
\libflame build system utilizes functionality that is present only in GNU
\makens.}
\item
{\bf A C compiler.}
Most of \libflame is written in C, and therefore building \libflame on for
GNU/Linux or UNIX requires a C (or C++) compiler.\footnote{
The \libflame configuration script will probe for and query a Fortran compiler
in order to detect the name mangling conventions necessary for
C and Fortran functions to call each other.
If your build fails because a Fortran compiler was not present at
configure-time, please contact the \libflame developers.}
The GNU {\tt gcc}, Intel {\tt icc}, IBM {\tt xlc}, and Pathscale {\tt pathcc}
compilers are explicitly supported.
A generic compiler named {\tt cc} should work too.
Later subsections in this chapter describe how the user may specify a preferred
C compiler.
\item
{\bf A working BLAS library.}
%BLAS routines, such as {\tt dgemm()} and {\tt dsyrk()}, are invoked to perform
%subproblems within virtually all blocked and unblocked algorithms, and also
%algorithms-by-blocks.
Users must link against an implementation of the BLAS in order
to use \libflamens.
%Why does \libflame need an external BLAS library? \libflame developers have
%adopted a philosophy of separation, whereby certain low-level operations
%are written and optimized by external developers.
Currently, \libflame functions make extensive use of BLAS routines such as
{\tt dgemm()} and {\tt dsyrk()} to perform subproblems that inherently
occur within almost all linear algebra algorithms.
\libflame also provides access to BLAS routines by way of wrappers that map
object-based APIs to the traditional Fortran-77 routine interface.
Any library that adheres to the BLAS interface should work fine.
However, we strongly encourage the use of Kazushige Goto's GotoBLAS
\cite{Goto,Goto:2008:AHP,taccsoftware2008}.
GotoBLAS provides excellent performance on a wide variety of mainstream
architectures.
Other BLAS libraries, such as ESSL (IBM), MKL (Intel), ACML (AMD),
and netlib's BLAS, have also been successfully tested with \libflamens.
Of course, performance will vary depending on which library is used. 
\end{itemize}
The following items are not required in order to build \libflamens, but may
still be useful to certain users, depending on how the library is
configured.
%The following items are optional, depending on how you would like to configure
%and use \libflamens:
\begin{itemize}
\item
{\bf A working LAPACK library.}
Most of the computationally-intensive operations implemented in \libflame
are expressed as blocked algorithms or algorithms-by-blocks, both of which
cast some of their computation in terms of smaller subproblems.
\libflame provides optimized, low-overhead unblocked functions to perform
these small matrix computations.
However, for performance reasons, some users might want these computations
to be performed instead by an external implementation of LAPACK.
See Section \ref{sec:configure-options} for more information on making use of
this optional feature.
\item
{\bf An OpenMP-aware C compiler.}
\libflame supports parallelism for several operations via the SuperMatrix
runtime scheduling system.
SuperMatrix requires either a C compiler that supports OpenMP (1.0 or later),
or a build environment that supports POSIX threads.
As of this writing, the GNU C compiler does not support OpenMP.
Therefore, the user must either ensure that \libflame is configured to use a
commercial OpenMP-aware compiler, or configure \libflame so that SuperMatrix
uses POSIX threads.\footnote{Whether there is an advantage in using OpenMP
over POSIX threads will depend on the specific OpenMP and POSIX
implementations.
However, preliminary evidence suggests that configuring SuperMatrix to derive
its parallelism from OpenMP results in slightly higher and slightly more
consistent performance.}
\end{itemize}

\subsection{System hardware support}

\index{\libflame!for GNU/Linux and UNIX!hardware support}

Over time, \libflame has been tested on a wide swath of modern architectures,
including but not limited to:
\begin{itemize}
\item x86 (Pentium, Athlon, Celeron, Duron, older Xeon series)
\item x86\_64 (Opteron, Athlon64, recent Xeon, Core2 series)
\item ia64 (Itanium series)
\item PowerPC/POWER series
\end{itemize}
Support by an architecture is primarily determined by the presence of an
appropriate compiler.
At configure-time, the \configure script will attempt to find an appropriate
compiler for a given architecure according to a predetermined search order
for that architecture.
For example, The first C compiler searched for on an Itanium2 system is Intel's
\iccns.
If \icc is not found, then the search continues for GNU \gcc.
If neither \icc nor \gcc is present, then the script checks for a generic
compiler named \ccns.
Table \ref{fig:search-order} summarizes the search order of C
compilers for some of the more common architectures supported by \libflamens.
Here, the architecture is identified by the canonical build system type, which
is a string of three dash-separated substrings, identifying the CPU type,
vendor, and operating system of the system which is performing the build.
The build system type is determined by the helper shell script
{\tt config.guess} and output by \configure at configure-time.

It is also possible for the user to specify the C compiler
explicitly at configure-time.
For more information on this and related topics, refer to Section
\ref{sec:configure-options}.

\begin{table}
\begin{center}
\begin{tabular}{|l|l|}
\hline
{\bf Build system type} & {\bf C compiler search order} \\ \hline
% --------------------------------------
%\multirow{3}{*}{
%\begin{tabular}[t]{@{}l}
%{\tt i386-*-*} \\
%{\tt i586-*-*} \\
%{\tt i686-*-*}
%\end{tabular}
%}
{\tt i386-*-*  i586-*-*  i686-*-*}
&
\gcc  \icc  \cc \\ \hline
%& & \\ \hline
% --------------------------------------
{\tt x86\_64-*-*}
&
\gcc  \icc  \pathcc  \cc \\ \hline
% --------------------------------------
{\tt ia64-*-*}
&
\icc  \gcc  \cc \\ \hline
% --------------------------------------
{\tt powerpc*-ibm-aix*}
&
\xlc \\ \hline
% --------------------------------------
{\tt powerpc64-*-linux-gnu}
&
\gcc  \xlc \\ \hline
% --------------------------------------
All others
&
\gcc  \cc \\ \hline
% --------------------------------------
\end{tabular}
\end{center}
\caption{
The list of compilers that are searched for as a function of build system
type, which consists three strings, separated by dashes, identifying
the build system CPU type, vendor, and operating system, where '{\tt *}' will
match any substring.
The actual build system string is determined by the helper shell script
{\tt config.guess} and output by \configure at configure-time.
Note that the search for the appropriate system type is performed from
top to bottom.
Once a matching string is found, the search for each compiler/tool is
performed from left to right.
}
\label{fig:search-order}
\end{table}




\subsection{License}

\index{\libflame!license}

\libflame is intellectual property of The University of Texas.
Unless you or your organization has made other arrangements, \libflame is
provided as free software under version 2.1 of the GNU Lesser General
Public License (LGPL).
Please refer to Appendix \ref{appendix:license} for the full text of this
license.





\subsection{Source code}

\index{\libflame!for GNU/Linux and UNIX!source code}


%The \libflame source code is available in two forms: the previous milestone
%release and the latest nightly snapshots.
%{\bf Milestones and nightly snapshots.}
%We encourage users to download and use the latest nightly snapshot.
%These packages contain full copies of the \libflame source tree, including
%all the relevant build system scripts and makefiles.
%Though it may seem like the milestone releases would be more stable and the
%nightly snapshots more prone to bugs, this is not really the case.
%Milestone releases are fine immediately after they are released, but they
%quickly grow out-of-date.
%It is more likely that the bugs present in the previous milestone release
%have been identified and fixed in the latest nightly snapshot.
%Furthermore, we do not offer any ``updates'' to milestone releases in
%between major versions.
%It is for this reason that we strongly encourage users to use the latest
%nightly snapshot, especially if you encounter problems compiling the
%source code.
The \libflame source code is available in two forms:
\begin{itemize}
\item
{\bf Nightly snapshots.}
We encourage users to download and use the latest nightly snapshot.
These packages contain full copies of the \libflame source tree, including
all the relevant build system scripts and makefiles as well as items of
interest to developers of \libflamens.
As their name suggests, these releases represent the state of the \libflame
repository each night (early morning, actually).
Expect these nightly snapshots to incorporate the latest interfaces,
code improvements, and bug fixes.
\item
{\bf Milestone releases.}
The \libflame team also makes previous milestone releases available to
users.
These releases are similar to the nightly snapshots, except that they are
also associated with an incremented version number and an update of the
\changelogns.
\end{itemize}
Though it may seem like the milestone releases would be more stable and the
nightly snapshots more prone to bugs, we have actually found the opposite
to be true.
Milestone releases are fine immediately after they are released, but they
quickly grow out-of-date.
%Users should be aware that by downloading an older milestone release, you
%may be downloading code with bugs that have already been fixed.
%Bugs are fixed as soon as possible after they are identified;
%however, it should be noted that these fixes are not applied as ``updates''
%to older releases
%Furthermore, it should be noted that we do not offer ``updates'' to
%milestone releases in between major versions.
%We try to fix bugs as soon as they are identified, however, these updates
%are not applied to older releases.
We check in updates to the library often, sometimes several times in one day.
However, these updates are not applied to older releases.
If an error exists in an older milestone release and the fix has already
been applied by \libflame developers, then the user must obtain
a more recent nightly snapshot to obtain the corrected code.\footnote{The user may also
find bug fixes by downloading a more recent milestone release.
However, since milestone releases are quite infrequent, roughly one per year,
obtaining a more recent milestone release is oftentimes not an option.}
It is for this reason that we {\em strongly} encourage users to use nightly
snapshots over milestones.

%If an error exists in an older release (milestone or nightly) and you suspect
%that the issue has not yet been addressed, please don't hesitate to contact
%the developers!
%You may email us anytime at \flameemailns.

%It is likely that a number of the bugs present in the previous milestone
%release have been identified and fixed in subsequent nightly snapshots.
%%It is for this reason that we strongly encourage users to use nightly
%%snapshots over milestones.%, especially if you encounter problems compiling the
%source code.
%Common problems such as errors while compiling code are likely to have
%already 





\subsection{Tracking source code revisions}


\index{\libflame!revision numbering}

Each copy of \libflame is named according to its subversion\footnote{We use
the subversion version control system to manage changes and synchronize
updates among developers.
}
%For more information about subversion and other topics of interest to
%developers, please see Chapter \ref{chapter:maint}.}
{\em revision} number.
These revision numbers are positive integers which uniquely identify various
states of the \libflame source tree and are usually preceded by a lowercase
``r''.
By contrast, milestone {\em version} numbers, such as ``2.0'', are somewhat
arbitrary labels that refer to long contiguous revision intervals.
As an example, version 1.0 was associated with revisions r1307 through r1753.
Revision numbers are incremented automatically every time a developer commits
a change or set of changes to the \libflame source tree.
Version numbers, however, identify milestone releases and increase rather
infrequently.
%Usually, the version number is increased only when \libflame developers decide
%that enough features and bug fixes have been added to be considered newsworthy
%to its target audience.
Usually, milestones are released (and the version number bumped) only
when \libflame developers decide that enough features and bug fixes have been
added to be considered newsworthy to its target audience.





\subsection{If you have problems}


If you encounter trouble while trying to build and install \libflamens, if
you think you've found a bug, or if you have a question not answered in
this document, we invite you to email your question to \flameemailns.
A \libflame developer will try to get back in touch with you as soon as
possible.





\section{Obtaining \libflame}

\index{\libflame!obtaining}

The source code for \libflame may be obtained through the FLAME project
website:

\begin{Verbatim}[frame=none,framesep=2.5mm,xleftmargin=5mm,commandchars=\\\{\},fontsize=\normalsize]
http://www.cs.utexas.edu/users/flame/libflame/
\end{Verbatim}

\noindent
This webpage also contains important information related to configuring,
compiling, installing, and linking against \libflamens.
Most of the information provided there is repeated and expanded upon in
this chapter.



\section{Preparation}

Download the {\tt .tar.gz} package from the website, and then un-tar and
un-gzip the source code.
Here, we assume that we've downloaded revision r4784 from the nightly
snapshots directory.

\begin{Verbatim}[frame=single,framesep=2.5mm,xleftmargin=5mm,commandchars=\\\{\},fontsize=\footnotesize]
> tar xzf libflame-r4784.tar.gz
> ls
libflame-r4784  libflame-r4784.tar.gz
\end{Verbatim}

\noindent
Change into the {\tt libflame-r4784} directory:

\begin{Verbatim}[frame=single,framesep=2.5mm,xleftmargin=5mm,commandchars=\\\{\},fontsize=\footnotesize]
> cd libflame-r4784
\end{Verbatim}

\noindent
The top-level directory of the source tree should look something like this:

\begin{Verbatim}[frame=single,framesep=2.5mm,xleftmargin=5mm,commandchars=\\\{\},fontsize=\footnotesize]
> ls
AUTHORS       Doxyfile  Makefile   build         docs      run-conf  tmp
CHANGELOG     INSTALL   README     configure     examples  src       windows
CONTRIBUTORS  LICENSE   bootstrap  configure.ac  play      test
\end{Verbatim}

\noindent
Table \ref{fig:top-level-files} describes each file present here.
In addition, the figure lists files that are created and overwritten
only upon running \configurens.

\begin{table}
\begin{center}
\begin{tabular}{llp{4.2in}}
{\bf File} & {\bf Type} & {\bf Description} \\
% --------------------------------------
{\tt AUTHORS}
&
peristent
&
Credits for authorship of various sub-components of \libflame. \\
% --------------------------------------
{\tt CHANGELOG}
&
peristent
&
A list of major changes in each major milestone release version. \\
% --------------------------------------
{\tt CONTRIBUTORS}
&
peristent
&
Credits for co-authors of working notes, conference papers, and journal
articles that have influenced the development of \libflamens. \\
% --------------------------------------
{\tt Doxyfile}
&
peristent
&
The configuration file for running {\tt doxygen}, which we use to
automatically generate a map of the source tree. \\
% --------------------------------------
{\tt INSTALL}
&
peristent
&
Generic instructions for configuring, compiling, and installing the software
package, courtesy of the Free Software Foundation. \\
% --------------------------------------
{\tt LICENSE}
&
peristent
&
The file specifying the license under which the software is made available.
As of this writing, \libflame is available as free software under version
2.1 of the GNU Lesser General Public License (LGPL). \\
% --------------------------------------
{\tt Makefile}
&
peristent
&
The top-level makefile for compiling \libflamens.
This makefile uses the GNU {\tt include} directive to recursively include
the makefile fragments that are generated at configure-time.
Therefore, it is inoperable until \configure has been run. \\
% --------------------------------------
{\tt README}
&
peristent
&
A short set of release notes directing the user to the \libflame web page
and the \libflame reference manual for more detailed information concerning
installation and usage. \\
% --------------------------------------
{\tt bootstrap}
&
peristent
&
A shell script used by developers to regenerate the \configure script. \\
% --------------------------------------
{\tt build}
&
peristent
&
This directory contains auxiliary build system files and shell scripts.
These files are probably only of interest to developers of \libflamens, and
so most users may safely ignore this directory. \\
% --------------------------------------
{\tt config}
&
build
&
A directory containing intermediate architecture-specific build files. \\
% --------------------------------------
{\tt config.log}
&
build
&
Logs information as it is gathered and processed by \configurens. \\
% --------------------------------------
{\tt config.status}
&
build
&
A helper script invoked by \configurens. \\
% --------------------------------------
{\tt config.sys\_type}
&
build
&
This file is used to communicate the canonical build system type between
\configure and {\tt config.status} helper script. \\
%The reasons this file is needed relate to variable scoping quirks and
%are beyond the scope of this description. \\
% --------------------------------------
{\tt configure}
&
peristent
&
The script used to configure \libflame for compiling and installation.
\configure accepts many options, which may be queried by running
\configurehelpns. \\
% --------------------------------------
{\tt configure.ac}
&
peristent
&
An input file to {\tt autoconf} that specifies how to build the \configure
script based on a sequence of {\tt m4} macros.
This file is only of interest to \libflame developers. \\
% --------------------------------------
{\tt docs}
&
peristent
&
A directory containing documentation related to \libflamens.
The LaTeX source to the \libflame reference manual resides here. \\
% --------------------------------------
{\tt examples}
&
peristent
&
A directory containing a few examples of \libflame algorithm
implementations and \libflame usage. \\
% --------------------------------------
{\tt lib}
&
build
&
A directory containing the libraries created after compilation. \\
% --------------------------------------
{\tt obj}
&
build
&
A directory containing the object files created during compilation. \\
% --------------------------------------
{\tt revision}
&
build/persistent
&
A file containing the subversion revision number of the source code. \\
% --------------------------------------
{\tt run-conf}
&
peristent
&
A directory containing a wrapper script to \configure that the user
may use to help them specify multiple options.
The script is strictly a convenience; some users will opt to instead
invoke \configure directly. \\
% --------------------------------------
{\tt src}
&
peristent
&
The root directory of all source code that goes into building \libflamens. \\
% --------------------------------------
{\tt test}
&
peristent
&
A monolithic test driver to test your \libflame installation. \\
%For now, this directory is empty. \\
%In the future, \libflame will feature its own native test drivers that
%will also allow easy performance testing. \\
% --------------------------------------
%{\tt tmp}
%&
%peristent
%&
%A directory containing miscellaneous developer odds-and-ends. \\
% --------------------------------------
{\tt windows}
&
peristent
&
The directory containing the Windows build system.
See Chapter \ref{chapter:setup-windows} for detailed instructions on how to
configure, compile, install, and link against a Windows build of
\libflamens. \\
% --------------------------------------
\end{tabular}
\end{center}
\caption{A list of the files and directories the user can expect to find in the
top-level \libflame directory along with descriptions.
Files marked ``peristent'' should always exist while files marked
``build'' are build products created by the build system.
This latter group of files may be safely removed by invoking the \make
target \distcleanns.
}
\label{fig:top-level-files}
\end{table}



\section{Configuration}

\index{\libflame!for GNU/Linux and UNIX!configuration}

The first step in building \libflame is to configure the build system by
running the \configure script.
\libflame may be configured many different ways depending on which options
are passed into \configurens.
These options and their syntax are always available by running \configure
with the {\tt --help} option:

\begin{Verbatim}[frame=single,framesep=2.5mm,xleftmargin=5mm,commandchars=\\\{\},fontsize=\footnotesize]
> ./configure --help
\end{Verbatim}

\noindent
Be aware that \configurehelp lists several options that are ignored by
\libflamens.\footnote{This is due to boilerplate content that \autoconf 
inserts into the \configure script regardless of whether it is desired.}
The options that are supported are listed explicitly and described in
the next subsection.

\subsection{\configure options}
\label{sec:configure-options}

\index{\libflame!for GNU/Linux and UNIX!\configure options}

The command line options supported by the \configure script
may be broken down into standard options, which most \configure scripts
respond to, and \libflamens-specific options, which refer to functionality
unique to \libflamens.

The standard command line options are: \\

\noindent
{\tt --prefix={\em prefixdir}} \\
\configpar{
The {\em prefixdir} directory specifies the install prefix directory (ie:
the root directory at which all \libflame build products will be installed).
If the directory does not exist, it is created.
%At install-time, libraries will be placed into {\em prefixdir/lib} and
%header files will be placed into {\em prefixdir/include}.
This value defaults to {\tt \$HOME/flame}.
}
%\newpage
\noindent
{\tt --help, -h} \\
\configpar{
Display a summary of all valid options to \configurens. 
(Note that this will display more options than \libflame actually uses.
Only those options described in this section are used internally by the build
system.)
}
\noindent
{\tt --help=short} \\
\configpar{
Display a summary of only those options that are specific to \libflamens. 
}
\noindent
{\tt --version, -V} \\
\configpar{
Display \libflame and autoconf version information. 
}
\noindent
{\tt --silent, --quiet, -q} \\
\configpar{
Silent mode. Do not print ``checking...'' messages during configuration.
}

All command line options specific to \libflame fall into two categories:
those which describe a particular {\em feature} to enable or disable,
and those which instruct \configure to set up the build for use with a
particular {\em tool}.

Command line options which denote features take the form
{\tt --disable-{\em FEATURE}} or
{\tt --enable-{\em FEATURE}}, where {\em FEATURE} is a short string that
describes the feature being enabled or disabled.
Enabling some options requires that an argument be specified.
In these cases, the syntax takes the form of
{\tt --enable-{\em FEATURE}={\em ARG}}, where {\em ARG} is an argument
specific to the feature being enabled.

Command line options which request the usage of certain tools are
similar to feature options, except that tool options always take an
argument.
These options take the form
{\tt --with-{\em TOOL}={\em TOOLNAME}}, where {\em TOOL} and {\em NAME}
are short strings that identify the class of tool and the actual tool name,
respectively.

The supported command line feature options are: \\

\noindent
{\tt --enable-verbose-make-output} \\
\configpar{
Enable verbose output as \make compiles source files and archives them into
the library file.
By default, \configure instructs \make to supress the actual commands sent
to the compilers (and to \arns) and instead print out more concise progress
messages.
This option is useful to developers and advanced users who suspect that \make
may not be invoking the compilers correctly.
{\em Disabled by default.}
}
\noindent
{\tt --enable-static-build} \\
\configpar{
Create \libflame as a static library archive.
%Note that the user may also build a dynamically-linked shared library
%along-side a static archive by also using {\tt --enable-dynamic-build}.
{\em Enabled by default.}
}
\noindent
{\tt --enable-dynamic-build} \\
\configpar{
Create \libflame as a dynamically-linked shared library.
Linking an executable to a shared library has the advantage that
only one copy of the library code will ever be loaded into memory.
%Note that the user may also build a static library archive along-side
%a shared library by also using {\tt --enable-static-build}.
{\em Disabled by default.}
}
\noindent
{\tt --enable-max-arg-list-hack} \\
\configpar{
Enable a workaround for environments where the amount of memory allocated to
storing command line argument lists is too small for \ar to archive all of
the library's object files with one command.
This usually is not an issue, but on some systems the user may get an
``Argument list too long'' error message.
In those situations, the user should enable this option.
Note: \make may not be run in parallel to build \libflame when this option
is enabled!
Doing so will result in undefined behavior from \arns.
{\em Disabled by default.}
}
\noindent
{\tt --enable-autodetect-f77-ldflags} \\
\configpar{
Enable automatic detection of any linker flags that may be needed to link
against Fortran code.
These flags are useful to know about when, for example, linking libflame
to a BLAS library that was compiled with the system's Fortran compiler.
You may need to disable this option, along with autodetection of Fortran
name-mangling, if the environment's Fortran compiler is
missing or broken.
{\em Enabled by default.}
}
\noindent
{\tt --enable-autodetect-f77-name-mangling} \\
\configpar{
Enable automatic detection of the name-mangling necessary to
invoke Fortran routines from C, and C-compiled routines from Fortran.
Disabling this option causes a pre-defined default to be used, which may not
work in some environments.
You may need to disable this option, along with autodetection of Fortran
linker flags, if the environment's Fortran compiler is missing or broken.
{\em Enabled by default.}
}
\noindent
{\tt --enable-non-critical-code} \\
\configpar{
Enable code that provides non-critical functionality.
This code has been identified as unnecessary when total library size is of
concern.
{\em Enabled by default.}
}
\noindent
{\tt --enable-builtin-blas} \\
\configpar{
Enable code that provides a built-in implementation of the BLAS.
Note that some routines may not be optimized yet.
{\em Disabled by default.}
}
\noindent
{\tt --enable-lapack2flame} \\
\configpar{
Compile and build into \libflame a compatibility layer that maps LAPACK
invocations to their corresponding FLAME/C implementations.
Note that erroneous input parameters are reported according to \libflame
conventions, NOT LAPACK conventions.
That is, if \libflame error checking is disabled, no error checking is
performed, and if erroneous input parameters are detected, the library
aborts.
Also, if this option is enabled, then {\tt external-lapack-for-subproblems}
MUST be disabled.
{\em Disabled by default.}
}
\noindent
{\tt --enable-external-lapack-for-subproblems} \\
\configpar{
Enable code that causes most of the computationally-intensive functions within
\libflame to compute their smallest subproblems by invoking a corresponding
(usually unblocked) LAPACK routine.
Note that if this option is enabled, \lapacktflame MUST be disabled.
Also, if this option is enabled, then {\tt external-lapack-interfaces}
MUST also be enabled.
Enabling this option is useful when a \libflame user wishes to leverage an
optimized external implementation of LAPACK to speed up the
subproblems that arise within \libflamens's blocked algorithms and
algorithms-by-blocks.
{\em Disabled by default.}
}
\noindent
{\tt --enable-external-lapack-interfaces} \\
\configpar{
Enable code that allows the user to interface with an external LAPACK
implementation via object-based FLAME-like functions.
Note that if this option is enabled, an LAPACK library will be required at
link-time.
{\em Disabled by default.}
}
\noindent
{\tt --enable-blas3-front-end-cntl-trees} \\
\configpar{
Enable code that uses control trees\footnotemark ~to select a reasonable
variant and blocksize when level-3 BLAS front-ends are invoked.
When disabled, the front-ends invoke their corresponding external
implementations.
Note that control trees are always used for LAPACK-level operations.
{\em Enabled by default.}
}
\footnotetext{
Control trees are internal constructs designed to reduce code redundancy
within \libflamens.
They allow developers to specify parameters such as blocksize, algorithmic
variant, and parallel execution without changing the code that defines the
algorithm in question.
%As such, the control trees are mostly of interest only to \libflame
%developers.
They are described in detail in Chapter \ref{chapter:dev-apis}.
}
\noindent
{\tt --enable-multithreading={\em model}} \\
\configpar{
Enable multithreading support.
Valid values for {\em model} are {\tt pthreads} and {\tt openmp}.
Multithreading must be enabled to access the shared memory parallelized
implementations provided by SuperMatrix.
{\em Disabled by default.}
}
\noindent
{\tt --enable-supermatrix} \\
\configpar{
Enable SuperMatrix, a dependency-aware task scheduling and parallel
execution system.
Note that multithreading support must also be enabled, via
{\tt --enable-multithreading}, in order to activate parallelized
implementations.
If SuperMatrix is enabled but multithreading is not, then SuperMatrix-aware
routines will operate sequentially in a verbose ``simulation'' mode.
{\em Disabled by default.}
}
\noindent
{\tt --enable-gpu} \\
\configpar{
Enable code that takes advantage of graphical processing units (GPUs) when
performing certain computations.
If enabled, SuperMatrix must also be enabled via {\tt --enable-supermatrix}.
Note that this option is experimental.
{\em Disabled by default.}
}
\noindent
{\tt --enable-vector-intrinsics={\em type}} \\
\configpar{
Enable highly-optimized code that relies upon vector intrinsics
to specify certain operations at a very low level.
Valid values for {\em type} are {\tt sse} and {\tt none}.
Specifying {\tt none} is the same as disabling the option.
{\em Disabled by default.}
}
\noindent
{\tt --enable-memory-alignment={\em N}} \\
\configpar{
Enable code that aligns dynamically allocated memory regions at {\em N}-byte
boundaries.
Specifically, this option configures \libflame to use {\tt posix\_memalign()}
instead of {\tt malloc()} for all internal memory allocation.
Note: {\em N} must be a power of two and multiple of {\tt sizeof(void*)},
which is usually 4 on 32-bit architectures and 8 on 64-bit architectures.
{\em Disabled by default.}
}
\noindent
{\tt --enable-ldim-alignment} \\
\configpar{
If memory alignment is requested, enable code that will increase, if
necessary, the leading dimension of \libflame objects so that each matrix
row or column begins at an aligned address.
{\em Disabled by default.}
}
\noindent
{\tt --enable-optimizations} \\
\configpar{
Employ traditional compiler optimizations when compiling C source code.
{\em Enabled by default.}
}
\noindent
{\tt --enable-warnings} \\
\configpar{
Use the appropriate flag(s) to request warnings when compiling C source code.
{\em Enabled by default.}
}
\noindent
{\tt --enable-debug} \\
\configpar{
Use the appropriate debug flag (usually {\tt -g}) when compiling C source code.
{\em Disabled by default.}
}
\noindent
{\tt --enable-profiling} \\
\configpar{
Use the appropriate profiling flag (usually {\tt -pg}) when compiling C
source code.
{\em Disabled by default.}
}
\noindent
{\tt --enable-internal-error-checking={\em level}} \\
\configpar{
Enable various internal runtime checks of function parameters and object
properties to prevent functions from executing with unexpected values.
Note that this option determines the default level, which may be changed at
runtime (via \flacheckerrorlevelsetns).
Valid values for {\em level} are {\tt full}, {\tt minimal}, and {\tt none}.
{\em Enabled by default to {\tt full}.}
}
\noindent
{\tt --enable-memory-leak-counter} \\
\configpar{
Enable code that keeps track of the balance between calls to
\flamalloc and \flafreens.
If enabled, the counter value is output to standard error upon calling
\flafinalizens.
Note that this option determines the default status, which may be changed at
runtime (via \flamemoryleakcountersetns).
{\em Disabled by default.}
}
\noindent
{\tt --enable-blis-use-of-fla-malloc} \\
\configpar{
Enable code that defines {\tt bli\_malloc()} in terms of {\tt FLA\_malloc()}.
One benefit of this is that BLIS memory allocations can be tracked, along
with other libflame memory allocations, if the memory leak counter is
enabled.
A second benefit is that BLIS memory allocations can be aligned to
boundaries if libflame memory alignment is enabled.
Note this option may only be set at configure-time.
{\em Enabled by default.}
}
\noindent
{\tt --enable-goto-interfaces} \\
\configpar{
Enable code that interfaces with internal/low-level functionality within
GotoBLAS, such as those symbols that may be queried for architecture-dependent
blocksize values.
When this option is disabled, reasonable static values are used instead.
Note that in order to use \libflame with a BLAS library other than GotoBLAS,
the user must disable this option.
{\em Disabled by default.}
}
\noindent
{\tt --enable-cblas-interfaces} \\
\configpar{
Enable code that interfaces \libflamens's external wrapper routines to the BLAS
via the CBLAS rather than the traditional Fortran-77 API.
{\em Disabled by default.}
}
\noindent
{\tt --enable-default-m-blocksize={\em mb}} \\
{\tt --enable-default-k-blocksize={\em kb}} \\
{\tt --enable-default-n-blocksize={\em nb}} \\
\configpar{
Enable user-defined blocksizes in the $ m $, $ k $, and $ n $ dimensions.
These options may be used to define the blocksizes that will be returned
from blocksize query functions when GotoBLAS interfaces are disabled.
Note that these options have no effect when GotoBLAS interfaces are enabled.
{\em Disabled by default.}
}
\noindent
{\tt --enable-portable-timer} \\
\configpar{
Define the {\tt FLA\_Clock()} timer function using {\tt clock\_gettime()}.
If that function is not available, then {\tt getttimeofday()} is used.
If neither function is available, {\tt FLA\_Clock()} is will return a
static value.
{\em By default, a portable timer is used (if it exists).}
}

A few command line feature options are supported by \configure but refer to
features that are experimental and/or not yet completely implemented.
Unless you are know what you are doing, you should avoid using these
options: \\

\noindent
{\tt --enable-windows-build} \\
\configpar{
Enable code that is needed for a Windows-friendly build of \libflamens.
This entails disabling all code specific to Linux/UNIX.
(Note: this option is actually never used in practice because the Windows
build of libflame does not use \configure to begin with.)
{\em Disabled by default.}
}
\noindent
{\tt --enable-scc} \\
\configpar{
Enable code that takes advantage of the SCC multicore
architecture.
When using this option, enabling SuperMatrix is recommended, though not
strictly required.
Note that this option is experimental.
{\em Disabled by default.}
}
\noindent
{\tt --enable-tidsp} \\
\configpar{
Enable code required for libflame to run under Texas Instruments' DSP.
Note that this option is experimental.
{\em Disabled by default.}
}


The supported command line tool options are: \\

\noindent
{\tt --with-cc={\em cc}} \\
\configpar{
Search for and use a C compiler named {\em cc}.
If {\em cc} is not found, then use the first compiler found from the
default search list for the detected build architecture.
}
\noindent
{\tt --with-extra-cflags={\em flags}} \\
\configpar{
When compiling C code, use the flags in {\em flags} in addition to the flags
that configure would normally use.
This is useful when the user wants some extra flags passed to the compiler
but does not want to manually set the {\tt CFLAGS} environment variable and
thus override all of the default compiler flags.
Note: Be sure to use quotations if the {\em flags} string contains spaces.
}
\noindent
{\tt --with-ar={\em ar}} \\
\configpar{
Search for and use a library archiver named {\em ar}.
If {\em ar} is not found, then use the first library archiver found
from the default search list for the detected build architecture.
Note: the library archiver search list usually consists only of {\tt ar}.
}
\noindent
{\tt --with-ranlib={\em ranlib}} \\
\configpar{
Search for and use a library archive indexer named {\em ranlib}.
If {\em ranlib} is not found, then use the first library archiver found
from the default search list for the detected build architecture.
Note: the library archiver search list usually consists only of {\tt ranlib}.
}

In addition to specifying tools via command line options, the user may
alternately make the same requests via environment variables.
Environment variables, if they are set, {\em always} override their
corresponding command line options.
\configure also supports a few related environment variables which do
not have an anologous command line option.

\begin{table}
\begin{center}
\begin{tabular}{llp{3.3in}}
{\bf Variable} & {\bf Command line option} & {\bf Description} \\
% --------------------------------------
{\tt CC}
&
{\tt --with-cc={\em cc}}
&
Use {\em CC} as the C compiler. \\
% --------------------------------------
{\tt CFLAGS}
&
{\em none}
&
The command line flags to use with the C compiler.
Note: Do not set this variable unless you know what you are doing!
It overrides C compiler flags that are set by \configure to correspond
to libflame feature options. \\
% --------------------------------------
{\tt EXTRA\_CFLAGS}
&
{\tt --with-extra-cflags={\em flags}}
&
Use {\em flags} in addition to the C compiler flags that are set
internally by \configurens.
Most users that need extra flags passed in to the C compiler will
want to set {\tt EXTRA\_CFLAGS} instead of {\tt CFLAGS}. \\
% --------------------------------------
{\tt AR}
&
{\tt --with-ar={\em ar}}
&
Use {\em AR} to create and fill static library archives. \\
% --------------------------------------
{\tt RANLIB}
&
{\tt --with-ranlib={\em ranlib}}
&
Use {\em RANLIB} to generate the index to the static library archive.
Note: In modern environments, the functionality of \ranlib has been
superceded by \arns; GNU \ranlib is equivalent to running {\tt ar -s}.  \\
% --------------------------------------
{\tt FIND}
&
{\em none}
&
The {\tt find} utility is needed by the \clean targets defined in the \makefilens. \\
% --------------------------------------
{\tt XARGS}
&
{\em none}
&
\xargsns, like \find, is needed by the \clean targets defined in the \makefilens. \\
% --------------------------------------
% --------------------------------------
\end{tabular}
\end{center}
\caption{
A list of the environment variables supported by \configurens.
Those environment variables which have corresponding command line options
are listed with entries in the middle column.
Note: Environment variables {\em always} override their corresponding
command line option, if one exists, and provided that it is passed in at
configure-time.
}
\label{fig:tool-env-vars}
\end{table}


Table \ref{fig:tool-env-vars} lists the supported
environment variables and their corresponding tool options, if one
exists.

\subsection{Running \configure}
\label{sec:running-configure}

\index{\libflame!for GNU/Linux and UNIX!running \configurens}

The simplest way to run \configure is to invoke it explicitly on the
command line, followed by any of the various options described in the
previous subsection.

\begin{Verbatim}[frame=single,framesep=2.5mm,xleftmargin=5mm,commandchars=\\\{\},fontsize=\footnotesize]
> ./configure --enable-supermatrix --enable-multithreading=pthreads --disable-internal-error-checking
\end{Verbatim}

\noindent
Alternatively, the user may invoke \configure indirectly through a convenient
wrapper script, \runconfigurens.
This script contains an invocation of \configure along with nearly all of
the default \configure options.
To specify non-default options, the user can simply edit the script and
then invoke it from the top-level directory, just as he would for
\configurens.

\begin{Verbatim}[frame=single,framesep=2.5mm,xleftmargin=5mm,commandchars=\\\{\},fontsize=\footnotesize]
> ./run-conf/run-configure.sh
\end{Verbatim}

\noindent
The benefit of using \runconfigure is twofold.
First, the user has a clear and concise way of reviewing the options passed
into \configurens.
This information is automatically output to {\tt config.log}; however, in
order to recover this information the user must sift through many lines of
logging output, which tends to be more cumbersome.
Second, the user can easily re-configure \libflame with slightly different
options by simply editing \runconfigure and then re-running the script.


The primary purpose of running \configure is to provide \make with some of
the information it needs in order to begin compiling \libflamens.
As \configure searches for and checks various parts of the build environment,
it echoes its progress to standard output.
The following is an example of a snippet of such output:

\begin{Verbatim}[frame=single,framesep=2.5mm,xleftmargin=5mm,commandchars=\\\{\},fontsize=\footnotesize]
> ./run-conf/run-configure.sh 
checking build system type... i686-pc-linux-gnu
checking host system type... i686-pc-linux-gnu
checking for GNU make... make
checking for GNU bash... bash
checking whether user requested a specific C compiler... no
configure: CC environment variable is set to gcc, which will override --with-cc option and default search
list for C compiler.
checking for C compiler default output file name... a.out
checking whether the C compiler works... yes
checking whether we are cross compiling... no
checking for suffix of executables... 
checking for suffix of object files... o
checking whether we are using the GNU C compiler... yes
checking whether gcc accepts -g... yes
\end{Verbatim}

\configure has another purpose, though: to create makefile fragments for each
directory in the source tree.
The user can see this second half of the \configure process with output
that looks something like:

\begin{Verbatim}[frame=single,framesep=2.5mm,xleftmargin=5mm,commandchars=\\\{\},fontsize=\footnotesize]
gen-make-frag.sh: creating makefile fragment in src/base/flamec
gen-make-frag.sh: creating makefile fragment in src/base/flamec/check
gen-make-frag.sh: creating makefile fragment in src/base/flamec/check/base
gen-make-frag.sh: creating makefile fragment in src/base/flamec/check/base/main
gen-make-frag.sh: creating makefile fragment in src/base/flamec/check/base/util
gen-make-frag.sh: creating makefile fragment in src/base/flamec/check/blas
gen-make-frag.sh: creating makefile fragment in src/base/flamec/check/blas/1
gen-make-frag.sh: creating makefile fragment in src/base/flamec/check/blas/2
gen-make-frag.sh: creating makefile fragment in src/base/flamec/check/blas/3
gen-make-frag.sh: creating makefile fragment in src/base/flamec/check/lapack
gen-make-frag.sh: creating makefile fragment in src/base/flamec/check/lapack/util
\end{Verbatim}

\noindent
These makefile fragments are included recursively by the top-level \makefile
and give \make access to the source files which reside throughout the source
tree.\footnote{
The idea behind generating recursively-includable makefile fragments at
configure-time is that these fragments will often change when files and
directories are added, moved, or deleted by \libflame developers, and thus
it is much more convenient for them to be generated automatically than to be
stored and maintained within the source code repository.}
The makefile fragments are all named \fragmentmkns, and thus they are hidden
from normal directory viewing.

Once \configure has completed, it invokes a secondary script, \postconfigurens,
to print out a summary of the configuration process.
Please review this summary and confirm that \libflame has been configured as
intended.

There is one section of the configuration summary that you should pay special
attention to.
If autodetection of Fortran name-mangling was enabled, at the end of the summary
there will be output that looks like:

\begin{Verbatim}[frame=single,framesep=2.5mm,xleftmargin=5mm,commandchars=\\\{\},fontsize=\footnotesize]
Autodetect Fortran linker flags................. : yes
   Fortran linker flags......................... :  -L/lusr/opt/gcc-4.2.2/lib/gcc/i686-pc-linux-gnu/4.2.2
-L/lusr/opt/gcc-4.2.2/lib/gcc/i686-pc-linux-gnu/4.2.2/../../.. -lgfortranbegin -lgfortran -lm
Autodetect Fortran name-mangling................ : yes
   Unmangled name............................... : foobar
   Mangled name................................. : foobar\_

Compile with extra C compiler flags............. : no

libflame install directory prefix............... : /u/field/flame

Compile with extra C compiler flags............. : no

libflame install directory prefix............... : /u/field/flame

Configuration complete!

NOTE: Autodetection of Fortran linker flags was enabled. The configure
script thinks that the flags listed above are necessary to successfully
link a program to Fortran object code. If your program uses any Fortran
libraries, you will probably need to link with these flags.
\end{Verbatim}

The purpose of this note is to inform the user of Fortran linker flags\footnote{
The flags shown were detected when \libflame was configured to use Intel
compilers in an {\tt i686-unknown-linux-gnu} build environment that
happens to provide both Intel and GNU compilers.
Oftentimes, \postconfigure will display link flags that appear to accomodate
linking with two different compiler packages.
In our experience, we've found that these extraneous flags do not interfere
with the compiler at link-time.
}
that may be needed in order
to successfully link \libflame and your application against Fortran code,
potentially including the BLAS or LAPACK libraries.
Sometimes, these flags are not necessary, but it is safer to always use them.
%The actual flags detected will vary depending on
%the build environment and the compiler chosen by \configurens.
Please see Section \ref{sec:linking} for further instructions on using these
flags at link-time.
In the meantime, there is no need to copy and save these flags to a separate
file.
You may view the flags detected by the previous run of \configure at
any time by opening the \postconfigure script in your favorite file editor
or viewer.
The \postconfigure script resides in the subdirectory of {\tt config} that
is identified by the build system string detected by \configurens.

\begin{Verbatim}[frame=single,framesep=2.5mm,xleftmargin=5mm,commandchars=\\\{\},fontsize=\footnotesize]
> ls -l config/i686-unknown-linux-gnu/post-configure.sh 
-rwxr--r-- 1 field dept 5861 Nov 14 13:35 config/i686-pc-linux-gnu/post-configure.sh
\end{Verbatim}
%> grep LDFLAGS config/x86_64-unknown-linux-gnu/post-configure.sh
%echo "LDFLAGS =  -L/opt/intel/fc/em64t/10.0.026/lib -L/usr/lib/gcc/x86_64-pc-linux-gnu/3.4.6/
%-L/usr/lib/gcc/x86_64-pc-linux-gnu/3.4.6/../../../../lib64 -lifport -lifcore -limf -lsvml -lm
%-lipgo -lirc -lirc_s -ldl"



%The INSTALL file contains generic instructions for configuring, compiling, and
%installing the library.
%These instructions are not specific to \libflamens.
%If you are in a hurry and wish to configure, compile, and install \libflame with
%default options, execute the following compound command:
%\begin{verbatim}
%> ./configure ; make ; make install
%\end{verbatim}


\section{Compiling}

\index{\libflame!for GNU/Linux and UNIX!compiling}

After \configure has run, the user may proceed to building the library.
The simplest way to do this is to just run \makens:

\begin{Verbatim}[frame=single,framesep=2.5mm,xleftmargin=5mm,commandchars=\\\{\},fontsize=\footnotesize]
> make
\end{Verbatim}

\noindent
This is actually shorthand for \make \allns.
That is, it tells \make to invoke the \all target, which in turn invokes
the \libs target.
Invoking the \libs target compiles and archives the library.
Table \ref{fig:make-targets} lists the most useful \make targets defined
in the \libflame \makefilens.

As \make performs individual compiles individual source files into object
files, it will output progress information.
By default, this appears as:

\begin{Verbatim}[frame=single,framesep=2.5mm,xleftmargin=5mm,commandchars=\\\{\},fontsize=\footnotesize]
Compiling src/base/flamec/main/FLA_Blocksize.c
Compiling src/base/flamec/main/FLA_Check.c
Compiling src/base/flamec/main/FLA_Error.c
Compiling src/base/flamec/main/FLA_File.c
Compiling src/base/flamec/main/FLA_Init.c
Compiling src/base/flamec/main/FLA_Lock.c
Compiling src/base/flamec/main/FLA_Memory.c
Compiling src/base/flamec/main/FLA_Misc.c
Compiling src/base/flamec/main/FLA_Obj.c
\end{Verbatim}

If \libflame was configured with {\tt --enable-verbose-make-output}, then
the output will show the actual compiler commands being executed.

\subsection{Parallel \make}

\libflame has been known to take a while to build, especially on systems
with slow processors and/or slow compilers.
If you are performing the build on an SMP or multicore system, then you
may parallelize the compilation by using the {\tt -j {\em n}} option to
\makens.
This option tells \make to perform up to {\em n} tasks in parallel.
In the following example, we request that \make avail itself to four-way
parallelism.

\begin{Verbatim}[frame=single,framesep=2.5mm,xleftmargin=5mm,commandchars=\\\{\},fontsize=\footnotesize]
> make -j4
\end{Verbatim}

\noindent
The {\em n} argument should be set to a reasonable value, such as
the number of cores or processors on the system.
Be aware that this may not necessarily speed up the build process if the
build system has an I/O bottleneck, such as a slow network-mounted
filesystem.

\begin{table}
\begin{center}
\begin{tabular}{lp{4.2in}}
{\bf Target} & {\bf Function} \\
% --------------------------------------
{\tt all}
&
Invoke the \libs target. \\
% --------------------------------------
{\tt check}
&
Verify that \configure has been run. \\
% --------------------------------------
{\tt libs}
&
Invoke the {\tt check} target and then build the libflame and
liblapack2flame library archives. \\
% --------------------------------------
{\tt install}
&
Invoke the \libs target and then copy the library archives and header
files to their respective {\tt lib} and {\tt include} subdirectories
of the install prefix directory, which is {\tt \$HOME/flame} by
default.
Also create symbolic links (with {\tt ln -s}) to the library files
and {\tt include} subdirectory so that the symbolic links do not
contain the architecture or version strings. \\
% --------------------------------------
{\tt install-without-symlinks}
&
Similar to \install except that no symbolic links are installed. \\
% --------------------------------------
{\tt clean}
&
Remove all object files and previously-built library archives from the
{\tt obj} and {\tt lib} directories for the current build system only.
(Recall that the curerent build system is written to
{\tt config.sys\_type}.)
Build products built for other systems will not be touched. \\
% --------------------------------------
{\tt cleanmk}
&
Remove all makefile fragments from the source tree. \\
% --------------------------------------
{\tt distclean}
&
Invoke \cleanns, \cleanmk, and then remove all other intermediate build files
and directories that are created either by \configure or one of the autotools
such as \autoconfns. \\
% --------------------------------------
{\tt send-thanks}
&
Use {\tt mail} to send the libflame developers a short thank-you message. \\
% --------------------------------------
\end{tabular}
\end{center}
\caption{A list of ``phony'' \make targets defined in the libflame
\makefilens.
%Note that this list is not exhaustive.
%The libflame \makefile defines other phony \make targets that are used
%internally to aid in the execution of those in this list, and still others
%that perform functions that are primarily only of interest to libflame
%developers.
Note that not all targets guarantee that action will take place.
Most targets will not fire if \make determines that the target is already
up-to-date.
For example, invoking the \clean target will not remove any object files if
they do not exist.
}
\label{fig:make-targets}
\end{table}




\section{Installation}

\index{\libflame!for GNU/Linux and UNIX!installing}

After \make has successfully completed, the \libflame library files
reside in a subdirectory of the {\tt lib} directory.
The exact subdirectory name depends on the build system type.

\begin{Verbatim}[frame=single,framesep=2.5mm,xleftmargin=5mm,commandchars=\\\{\},fontsize=\footnotesize]
> ls -l lib/i686-unknown-linux-gnu/
total 33872
-rw-r--r-- 1 field dept 22548036 Nov 14 13:45 libflame.a
-rwxr-xr-x 1 field dept 12060827 Nov 14 13:45 libflame.so
\end{Verbatim}

\noindent
In this example, \libflame was built for an i686 system with a
build system type of {\tt i686-unknown-linux-gnu}, and so the
library files reside in the directory
{\tt lib/i686-unknown-linux-gnu}.

At this point, you need to use the \install target to move the library
and header files to a more permanent location.

\begin{Verbatim}[frame=single,framesep=2.5mm,xleftmargin=5mm,commandchars=\\\{\},fontsize=\footnotesize]
> make install
Installing libflame-i686-r4784.a into /u/field/flame/lib/
Installing libflame-i686-r4784.so into /u/field/flame/lib/
Installing C header files into /u/field/flame/include-i686-r4784
Installing symlink libflame.a into /u/field/flame/lib/
Installing symlink libflame.so into /u/field/flame/lib/
Installing symlink include into /u/field/flame/
\end{Verbatim}

\noindent
Here, we can see the library and header files were moved into the default
subdirectories of the user's home directory.
Notice that the libraries and include directory are renamed to reflect
the build architecture and the revision number.

\begin{Verbatim}[frame=single,framesep=2.5mm,xleftmargin=5mm,commandchars=\\\{\},fontsize=\footnotesize]
> ls -l $HOME/flame/
total 8
lrwxrwxrwx  1 field dept     18 Nov 14 13:46 include -> include-i686-r4784
drwxr-xr-x  2 field dept  20480 Nov 14 13:46 include-i686-r4784
drwxr-xr-x  5 field dept   4096 Nov 14 13:46 lib
> ls -l $HOME/flame/lib/
total 62024
lrwxrwxrwx 1 field dept       21 Nov 14 13:46 libflame.a -> libflame-i686-r4784.a
lrwxrwxrwx 1 field dept       22 Nov 14 13:46 libflame.so -> libflame-i686-r4784.so
-rw-r--r-- 1 field dept 22548036 Nov 14 13:46 libflame-i686-r4784.a
-rw-r--r-- 1 field dept 12060827 Nov 14 13:46 libflame-i686-r4784.so
\end{Verbatim}

\noindent
Also notice that the \install target automatically creates abbreviated symbolic
links to the actual library files so that you may simply refer to
{\tt libflame.a} or {\tt libflame.so} when linking your application.
A similarly shortened symbolic link is created for the include directory
as well.
If you do not wish to create and install these symbolic links along with
\libflamens, you may instead invoke \installwithoutsymlinksns.

Now that \libflame has been installed, you are ready to use it!
%If you only plan on building \libflame for one architecture, then you may
%use the \installsymlinks target to install symbolic links to the library
%archive and the include directory.

%However, if you plan on building and installing multiple instances of
%\libflame and you wish to install them to the same directory, we offer
%the \installsymlinkswitharch target.
%Executing this target creates symbolic links similar to those of 
%\installsymlinksns, except that only the revision string is omitted.
%Keeping the architecture string provides enough differentiation to allow
%multiple installations of \libflame to coexist in the same install
%directory.


%\noindent
%sdlkfj lskd flskdj flsjd flksjdf slfj klsdf l
%
% BlueViolet
% RoyalPurple
% Violet
% Fuchsia
% 
%\begin{Verbatim}[frame=single,framesep=2.5mm,xleftmargin=5mm,commandchars=\\\{\},fontsize=\footnotesize,rulecolor=\color{Black}]
%\textcolor{Fuchsia}{> make install-symlinks-with-arch}
%\textcolor{Fuchsia}{Installing symlink libflame-x86_64.a into /home/field/flame/lib/}
%\textcolor{Fuchsia}{Installing symlink include-x86_64 into /home/field/flame/}
%\end{Verbatim}

%\noindent
%After executing the \install target,
%we are left with an install directory that looks like:
%
%\begin{Verbatim}[frame=single,framesep=2.5mm,xleftmargin=5mm,commandchars=\\\{\},fontsize=\footnotesize]
%> ls -l $HOME/flame
%total 8
%lrwxrwxrwx 1 field flame   19 Aug 29 17:08 include-x86_64 -> include-x86_64-r4784
%drwxr-xr-x 2 field flame 4096 Aug 29 17:07 include-x86_64-r4784
%drwxr-xr-x 2 field flame 4096 Aug 29 17:39 lib
%> ls -l $HOME/flame/lib
%total 17680
%-rw-r--r-- 1 field flame 17226100 Aug 29 17:07 libflame-x86_64-r4784.a
%lrwxrwxrwx 1 field flame       22 Aug 29 17:08 libflame-x86_64.a -> libflame-x86_64-r4784.a
%\end{Verbatim}
%
%\noindent
%Of course, you could execute both the \installsymlinks and
%\installsymlinkswitharch targets and get both kinds of symbolic links.




\section{Linking against \libflame}
\label{sec:linking}

\index{\libflame!for GNU/Linux and UNIX!linking against}

Since you are building \libflamens, you probably wish to use it in your
application.
This section will show you how to link \libflame with your existing application.

Let's assume that you've installed \libflame to the default location in
{\tt \$HOME/flame}.
Let's also assume that you invoked the \install target (and not the
\installwithoutsymlinks target), giving
you shorthand symbolic links to both \libflame and the directory containing
header files.

In general, you should make the following changes to your application build
process:
\begin{itemize}
\item
{\bf Add the \libflame header directory to the include path of your compiler}.
Usually, this is done by with the {\tt -I} compiler option.
For example, if you configured \libflame to use {\tt \$HOME/flame} as the
install prefix, then you would add {\tt -I\$HOME/flame/include} to the command
line when invoking the compiler.
Strictly speaking, this is only necessary when compiling source code files
that use \libflame symbols or APIs, but it is generally safe to use when compiling
all of your application's source code.
\item
{\bf Add \libflame to the link command that links your application.}
If you only wish to use the native \libflame API, then you only need to add
{\tt libflame.a} to your link command.
However, note that {\tt libflame.a} {\em must} appear in front of the
LAPACK and BLAS libraries.
This is because the linker only searches for symbols in the ``current''
archive and those that appear further down in the link command.
Placing \libflame after LAPACK or the BLAS will result in undefined symbol
errors at link-time.
\item
{\bf Use the recommended linker flags detected by \configurens.}
This topic was previously alluded to toward the end of Section
\ref{sec:running-configure}.
It is often the case that you must add various linker flags to the link
command in order to properly link your application with \libflamens.
This is usually the result of the compilers embedding certain low-level
functions into the object code.
These functions may only be resolved at link-time if the library in which
they are defined is also provided to the linker.
The list of linker flags that you will need is displayed when \configure
finished and exits.
After \configure is run, you may also find these linker flags in the
\postconfigure script, as described near the end of
Section \ref{sec:running-configure}.
\end{itemize}

Now let's give a concrete example of these changes.
Suppose you've been building your application with a \makefile that looks
something like:

\begin{Verbatim}[frame=single,framesep=2.5mm,xleftmargin=5mm,commandchars=\#\{\},fontsize=\footnotesize]
SRC_PATH   := .
OBJ_PATH   := .
INC_PATH   := .

LIB_HOME   := $(HOME)
BLAS_LIB   := $(LIB_HOME)/lib/libblas.a
LAPACK_LIB := $(LIB_HOME)/lib/liblapack.a

CC         := icc
LINKER     := $(CC)
CFLAGS     := -g -O2 -Wall -I$(INC_PATH)
LDFLAGS    := -lm 

MYAPP_OBJS := main.o file.o util.o proc.o
MYAPP_BIN  := my_app

$(OBJ_PATH)/%.o: $(SRC_PATH)/%.c
    $(CC) $(CFLAGS) -c $< -o $@

$(MYAPP): $(MYAPP_OBJS)
    $(LINKER) $(MYAPP_OBJS) $(LDFLAGS) $(LAPACK_LIB) $(BLAS_LIB) -o $(MYAPP_BIN)

clean:
    rm -f $(MYAPP_OBJS) $(MYAPP_BIN)
\end{Verbatim}

\noindent
To link against \libflame, you should change your \makefile as follows:

\begin{Verbatim}[frame=single,framesep=2.5mm,xleftmargin=5mm,commandchars=\#\{\},fontsize=\footnotesize]
SRC_PATH   := .
OBJ_PATH   := .
INC_PATH   := .

LIB_HOME   := $(HOME)
BLAS_LIB   := $(LIB_HOME)/lib/libblas.a
LAPACK_LIB := $(LIB_HOME)/lib/liblapack.a

#textcolor{red}{FLAME_HOME := $(HOME)}
#textcolor{red}{FLAME_INC  := $(FLAME_HOME)/flame/include}
#textcolor{red}{FLAME_LIB  := $(FLAME_HOME)/flame/lib/libflame.a}

CC         := icc
LINKER     := $(CC)
CFLAGS     := -g -O2 -Wall -I$(INC_PATH) #textcolor{red}{-I$(FLAME_INC)}
LDFLAGS    := #textcolor{red}{-L/opt/intel/fc/em64t/10.0.026/lib}
#textcolor{red}{LDFLAGS    += -L/usr/lib/gcc/x86_64-pc-linux-gnu/3.4.6/}
#textcolor{red}{LDFLAGS    += -L/usr/lib/gcc/x86_64-pc-linux-gnu/3.4.6/../../../../lib64}
#textcolor{red}{LDFLAGS    += -lifport -lifcore -limf -lsvml -lm -lipgo -lirc -lirc_s -ldl}

MYAPP_OBJS := main.o file.o util.o proc.o
MYAPP_BIN  := my_app

$(OBJ_PATH)/%.o: $(SRC_PATH)/%.c
    $(CC) $(CFLAGS) -c $< -o $@

$(MYAPP): $(MYAPP_OBJS)
    $(LINKER) $(MYAPP_OBJS) $(LDFLAGS) #textcolor{red}{$(FLAME_LIB)} $(LAPACK_LIB) $(BLAS_LIB) -o $(MYAPP_BIN)

clean:
    rm -f $(MYAPP_OBJS) $(MYAPP_BIN)
\end{Verbatim}

\noindent
The changes appear in red.

First, we define the locations of \libflame and the \libflame header directory.

Second, we include the location of the \libflame headers to the compilers'
command line options so that the C compiler will be able to perform type
checking against \libflame declarations and prototypes.

Third, we add the linker flags to the {\tt LDFLAGS} variable so that the
linker can find any auxiliary system libraries that might be needed in
order to link your application with the object code present in \libflamens.

Finally, we add the \libflame library to the link command, making sure to
insert it before the LAPACK and BLAS libraries.



\subsection{Linking with the {\tt lapack2flame} compatibility layer}

\index{\libflame!for GNU/Linux and UNIX!linking with \lapacktflame}

The previous section demonstrated how to modify a hypothetical makefile
to link a pre-existing application to \libflamens.
However, some users have applications which use LAPACK interfaces and
wish to use \libflame without chaning their application code.
This may be accomplished by configuring \libflame to include the
\lapacktflame compatibility layer.
When this option is provided at configure-time, \libflame is built
to include interfaces that map conventional LAPACK routine invocations to
native FLAME/C function calls.

For more information about the routines supported by \lapacktflame, refer
to Section \ref{sec:lapack2flame}.



