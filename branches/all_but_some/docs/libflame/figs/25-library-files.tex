\begin{table}[th]
\begin{center}
\begin{tabular}{|l|l|p{3.7in}|}
\hline
{\bf Link type} & {\bf Component files} & {\bf Purpose} \\ \hline
% --------------------------------------
%statically-linked
static
&
{\tt libflame-x64-r3692.lib}
&
The static library containing the \libflame object files.
Link to this file when statically linking your application to \libflamens.
\\ \hline
% --------------------------------------
%dynamically-linked
dynamic
&
{\tt libflame-x64-r3692.dll}
&
The dynamic library containing the \libflame object code.
This file is loaded into memory by the operating system at run-time
the first time a dependent program or library references \libflame
symbols.
\\ \cline{2-3}
&
{\tt libflame-x64-r3692.lib}
&
The import library.
This file contains information such as the dynamic library filename and which
symbols are available within the dynamic library.
The import library is used by the linker at link-time to resolve all function
calls referenced by the application being built.
If you plan to use a dynamic library build of \libflamens, reference this file
when linking your application.
\\ \cline{2-3}
&
{\tt libflame-x64-r3692.exp}
&
The export file.
This file is necessary only when building other dynamic libraries that
depend on a dynamic library build of \libflamens.
\\ \hline
% --------------------------------------
\end{tabular}
\end{center}
\caption{
The files generated when building revision r3692 of \libflame as either a static
or dynamic library.
The filenames reflect using ``x64'' as the architecture string when running
\configurecmdns.
}
\label{fig:library-files}
\end{table}
