\chapter{Setup for Microsoft Windows}
\label{chapter:setup-windows}



This chapter discusses how to obtain, configure, compile, and install
\libflame under Microsoft Windows.



\section{Before obtaining \libflame}

We encourage new users to read this section before proceeding to download
the \libflame source code.

\subsection{System software requirements}

\index{\libflame!for Microsoft Windows!software requirements}

Before you attempt to build \libflamens, be sure you have the following
software tools:
\begin{itemize}
\item
{\bf Microsoft Windows XP or later.}
At this time we have tested \libflame under Windows XP and Windows 7.
We have not yet been able to test the software under Windows Vista, though we
suspect it would compile, link, and run just fine.
\item
{\bf A C/C++ compiler.}
Most of \libflame is written in C, and therefore building \libflame on Windows
requires a C (or C++) compiler.
The build system may be configured to use either the Intel C/C++ compiler or
the Microsoft C/C++ compiler.
However, another compiler can be substituted by tweaking the definitions file
included into the main makefile.
\item
{\bf nmake.}
\libflame for Windows requires the Microsoft Program Maintenance Utility,
\nmakens.
\nmake is a command line tool similar to GNU \make that allows developers to
use makefiles to specify how programs and libraries should be built.
This utility is included with the Microsoft Visual Studio development
environment.
\item
{\bf Python.}
Certain helper scripts within the Windows build system are written in Python,
and therefore the user must have Python installed in the build environment
in order to run the build \libflamens.
We recommend a recent version, though version 2.6 or later should work fine.
\item
{\bf A working BLAS library.}
Users must link against an implementation of the BLAS in order
to use \libflamens.
Currently, \libflame functions make extensive use of BLAS routines such as
{\tt dgemm()} and {\tt dsyrk()} to perform subproblems that inherently
occur within almost all linear algebra algorithms.
When configured accordingly,
\libflame also provides direct access to BLAS routines by way of wrappers
that map object-based APIs to traditional Fortran-77 routine interfaces.
Any library that adheres to the BLAS interface should work fine.
On Windows, \libflame developers often use Intel's MKL, which performs well
and is included with the Intel C/C++ compiler suite.
%However, we strongly encourage the use of Kazushige Goto's GotoBLAS
%\cite{Goto,Goto:2008:AHP,taccsoftware2008}.
%GotoBLAS provides excellent performance on a wide variety of mainstream
%architectures.
%Other BLAS libraries, such as ESSL (IBM), MKL (Intel), ACML (AMD),
%and netlib's BLAS, have also been successfully tested with \libflamens.
%Of course, performance will vary depending on which library is used. 
\end{itemize}
The following items are not required in order to build \libflamens, but may
still be useful to certain users, depending on how the library is
configured.
\begin{itemize}
\item
{\bf A working LAPACK library.}
Most of the computationally-intensive operations implemented in \libflame
are expressed as blocked algorithms or algorithms-by-blocks, both of which
cast some of their computation in terms of smaller subproblems.
\libflame provides optimized, low-overhead unblocked functions to perform
these small matrix computations.
However, for performance reasons, some users might want these computations
to be performed instead by an external implementation of LAPACK.
See Section \ref{sec:configure-options} for more information on making use of
this optional feature.
\item
{\bf An OpenMP-aware C compiler.}
\libflame supports parallelism for several operations via the SuperMatrix
runtime scheduling system.
SuperMatrix requires either a C compiler that supports OpenMP (1.0 or later),
or a build environment that supports POSIX threads.
POSIX threads support is not shipped with Microsoft Windows.
However, as of this writing, both the Microsoft and Intel C/C++ compilers
support OpenMP.
Therefore, the user must either ensure that \libflame is configured to use 
an OpenMP-aware compiler.
\end{itemize}

\subsection{System hardware support}

\index{\libflame!for Microsoft Windows!hardware support}

Since \libflame for Windows is still relatively new, we have not had the time
or opportunity to test it on many hardware architectures.
We suspect it should compile and run fine on any of the modern Intel
architectures, including traditional 32-bit x86 architectures as well as
newer 64-bit em64t systems.
Other architectures, such as ia64 systems, may work, but they are untested
as of this writing.

\subsection{License}

\index{\libflame!license}

\libflame is intellectual property of The University of Texas.
Unless you or your organization has made other arrangements, \libflame is
provided as free software under version 2.1 of the GNU Lesser General
Public License (LGPL).
Please refer to Appendix \ref{appendix:license} for the full text of this
license.





\subsection{Source code}

\index{\libflame!for Microsoft Windows!source code}


%The \libflame source code is available in two forms: the previous milestone
%release and the latest nightly snapshots.
%{\bf Milestones and nightly snapshots.}
%We encourage users to download and use the latest nightly snapshot.
%These packages contain full copies of the \libflame source tree, including
%all the relevant build system scripts and makefiles.
%Though it may seem like the milestone releases would be more stable and the
%nightly snapshots more prone to bugs, this is not really the case.
%Milestone releases are fine immediately after they are released, but they
%quickly grow out-of-date.
%It is more likely that the bugs present in the previous milestone release
%have been identified and fixed in the latest nightly snapshot.
%Furthermore, we do not offer any ``updates'' to milestone releases in
%between major versions.
%It is for this reason that we strongly encourage users to use the latest
%nightly snapshot, especially if you encounter problems compiling the
%source code.
The \libflame source code is available in two forms:
\begin{itemize}
\item
{\bf Nightly snapshots.}
We encourage users to download and use the latest nightly snapshot.
These packages contain full copies of the \libflame source tree, including
all the relevant build system scripts and makefiles as well as items of
interest to developers of \libflamens.
As their name suggests, these releases represent the state of the \libflame
repository each night (early morning, actually).
Expect these nightly snapshots to incorporate the latest interfaces,
code improvements, and bug fixes.
\item
{\bf Milestone releases.}
The \libflame team also makes previous milestone releases available to
users.
These releases are similar to the nightly snapshots, except that they are
also associated with an incremented version number and an update of the
\changelogns.
\end{itemize}
Though it may seem like the milestone releases would be more stable and the
nightly snapshots more prone to bugs, we have actually found the opposite
to be true.
Milestone releases are fine immediately after they are released, but they
quickly grow out-of-date.
%Users should be aware that by downloading an older milestone release, you
%may be downloading code with bugs that have already been fixed.
%Bugs are fixed as soon as possible after they are identified;
%however, it should be noted that these fixes are not applied as ``updates''
%to older releases
%Furthermore, it should be noted that we do not offer ``updates'' to
%milestone releases in between major versions.
%We try to fix bugs as soon as they are identified, however, these updates
%are not applied to older releases.
We check in updates to the library often, sometimes several times in one day.
However, these updates are not applied to older releases.
If an error exists in an older milestone release and the fix has already
been applied by \libflame developers, then the user must obtain
a more recent nightly snapshot to obtain the corrected code.\footnote{The user may also
find bug fixes by downloading a more recent milestone release.
However, since milestone releases are quite infrequent, roughly one per year,
obtaining a more recent milestone release is oftentimes not an option.}
It is for this reason that we {\em strongly} encourage users to use nightly
snapshots over milestones.

%If an error exists in an older release (milestone or nightly) and you suspect
%that the issue has not yet been addressed, please don't hesitate to contact
%the developers!
%You may email us anytime at \flameemailns.

%It is likely that a number of the bugs present in the previous milestone
%release have been identified and fixed in subsequent nightly snapshots.
%%It is for this reason that we strongly encourage users to use nightly
%%snapshots over milestones.%, especially if you encounter problems compiling the
%source code.
%Common problems such as errors while compiling code are likely to have
%already 





\subsection{Tracking source code revisions}


\index{\libflame!revision numbering}

Each copy of \libflame is named according to its subversion\footnote{We use
the subversion version control system to manage changes and synchronize
updates among developers.
}
%For more information about subversion and other topics of interest to
%developers, please see Chapter \ref{chapter:maint}.}
{\em revision} number.
These revision numbers are positive integers which uniquely identify various
states of the \libflame source tree and are usually preceded by a lowercase
``r''.
By contrast, milestone {\em version} numbers, such as ``2.0'', are somewhat
arbitrary labels that refer to long contiguous revision intervals.
As an example, version 1.0 was associated with revisions r1307 through r1753.
Revision numbers are incremented automatically every time a developer commits
a change or set of changes to the \libflame source tree.
Version numbers, however, identify milestone releases and increase rather
infrequently.
%Usually, the version number is increased only when \libflame developers decide
%that enough features and bug fixes have been added to be considered newsworthy
%to its target audience.
Usually, milestones are released (and the version number bumped) only
when \libflame developers decide that enough features and bug fixes have been
added to be considered newsworthy to its target audience.





\subsection{If you have problems}


If you encounter trouble while trying to build and install \libflamens, if
you think you've found a bug, or if you have a question not answered in
this document, we invite you to email your question to \flameemailns.
A \libflame developer will try to get back in touch with you as soon as
possible.





\section{Obtaining \libflame}

\index{\libflame!obtaining}

The source code for \libflame may be obtained through the FLAME project
website:

\begin{Verbatim}[frame=none,framesep=2.5mm,xleftmargin=5mm,commandchars=\\\{\},fontsize=\normalsize]
http://www.cs.utexas.edu/users/flame/libflame/
\end{Verbatim}

\noindent
This webpage also contains information related to configuring,
compiling, installing, and linking against \libflame under GNU/Linux
and UNIX environments.
Most of the information provided there is repeated and expanded upon in
Chapter \ref{chapter:setup-linux}.



\section{Preparation}

Download the {\tt .zip} package from the website and then unzip the
the source code.
Here, we assume that we've downloaded revision 3692 (r3692) from the
nightly snapshots directory and unzipped the package to a directory by
the same name, minus the {\tt .zip} extension.

\begin{Verbatim}[frame=single,framesep=2.5mm,xleftmargin=5mm,fontsize=\footnotesize]
C:\field\temp>dir
 Volume in drive C has no label.
 Volume Serial Number is B4E3-D9FC

 Directory of C:\field\temp

12/01/2009  01:03 PM    <DIR>          .
12/01/2009  01:03 PM    <DIR>          ..
12/01/2009  01:04 PM    <DIR>          libflame-r3692
12/01/2009  01:02 PM         5,324,397 libflame-r3692.zip
               1 File(s)      5,324,397 bytes
               3 Dir(s)  85,294,235,648 bytes free
\end{Verbatim}

\noindent
Change into the {\tt libflame-r3692} directory:

\begin{Verbatim}[frame=single,framesep=2.5mm,xleftmargin=5mm,fontsize=\footnotesize]
C:\field\temp>cd libflame-r3692
\end{Verbatim}

\noindent
The top-level directory of the source tree should look something like this:

\begin{Verbatim}[frame=single,framesep=2.5mm,xleftmargin=5mm,fontsize=\footnotesize]
C:\field\temp\libflame-r3692>dir
 Volume in drive C has no label.
 Volume Serial Number is B4E3-D9FC

 Directory of C:\field\temp\libflame-r3692

12/01/2009  01:04 PM    <DIR>          .
12/01/2009  01:04 PM    <DIR>          ..
12/01/2009  01:03 PM               893 AUTHORS
12/01/2009  01:03 PM                91 bootstrap
12/01/2009  01:03 PM    <DIR>          build
12/01/2009  01:03 PM             5,836 CHANGELOG
12/01/2009  01:03 PM           293,036 configure
12/01/2009  01:03 PM            13,853 configure.ac
12/01/2009  01:03 PM             2,329 CONTRIBUTORS
12/01/2009  01:03 PM    <DIR>          docs
12/01/2009  01:03 PM            50,468 Doxyfile
12/01/2009  01:03 PM    <DIR>          examples
12/01/2009  01:03 PM             9,478 INSTALL
12/01/2009  01:03 PM            26,420 LICENSE
12/01/2009  01:03 PM            12,983 Makefile
12/01/2009  01:03 PM             1,216 README
12/01/2009  01:03 PM                 5 revision
12/01/2009  01:03 PM    <DIR>          run-conf
12/01/2009  01:04 PM    <DIR>          src
12/01/2009  01:04 PM    <DIR>          test
12/01/2009  01:04 PM    <DIR>          tmp
12/01/2009  01:04 PM    <DIR>          windows
              12 File(s)        416,608 bytes
              10 Dir(s)  85,294,235,648 bytes free
\end{Verbatim}

\noindent
This is the top-level directory for the default GNU/Linux and UNIX
builds.\footnote{Table \ref{fig:top-level-files} describes the files present
in the top-level GNU/Linux and UNIX build directory.}
However, since we are building \libflame for Windows, we should focus on the
{\tt windows} subdirectory.

\begin{Verbatim}[frame=single,framesep=2.5mm,xleftmargin=5mm,fontsize=\footnotesize]
C:\field\temp\libflame-r3692>cd windows

C:\field\temp\libflame-r3692\windows>dir
 Volume in drive C has no label.
 Volume Serial Number is B4E3-D9FC

 Directory of C:\field\temp\libflame-r3692\windows

12/01/2009  01:04 PM    <DIR>          .
12/01/2009  01:04 PM    <DIR>          ..
12/01/2009  01:04 PM    <DIR>          build
12/01/2009  01:04 PM             2,667 configure.cmd
12/01/2009  01:04 PM             4,057 gendll.cmd
12/01/2009  01:04 PM               434 linkargs.txt
12/01/2009  01:04 PM               452 linkargs64.txt
12/01/2009  01:04 PM            11,847 Makefile
12/01/2009  01:04 PM                 5 revision
               6 File(s)         19,462 bytes
               3 Dir(s)  85,294,235,648 bytes free
\end{Verbatim}

\noindent
Table \ref{fig:top-level-win-files} describes each file present here.
In addition, the figure lists files that are created and overwritten
only upon running \configurecmdns.



\section{Configuration}
\label{sec:configuration-win}

\index{\libflame!for Microsoft Windows!configuration}

The first step in building \libflame for Windows is to set the configuration
options.

The next three sections describe how to build \libflame as a static
library.
Please see Section \ref{sec:dll} for supplemental instructions on building a
dynamically-linked library.

The bulk of the configuration options are specified in the file
{\tt build$\bs$FLA\_config.h}.\footnote{Unlike in the GNU/Linux build system,
the user must set these options manually. We apologize for the inconvenience.}
The options correspond to C preprocessor macros.
If a macro is commented out, the feature is disabled, otherwise it is enabled.
Each macro is also preceeded with a comment containing a brief description of
its corresponding feature.
Full documentation for each feature macro in {\tt build$\bs$FLA\_config.h}
may be found in Section \ref{sec:configure-options}.

\begin{table}[htp]
\begin{center}
\begin{tabular}{llp{4.0in}}
{\bf File} & {\bf Type} & {\bf Description} \\
% --------------------------------------
{\tt Makefile}
&
persistent
&
The top-level makefile for compiling \libflame under Microsoft Windows.
This makefile is written for Microsoft's Program Maintenance Utility,
\nmakens.
It may only be run after \configurecmd is run. \\
% --------------------------------------
{\tt build}
&
persistent
&
This directory contains auxiliary build system files and scripts.
These files are probably only of interest to developers of \libflamens, and
so most users may safely ignore this directory. \\
% --------------------------------------
{\tt config}
&
build
&
A directory containing intermediate build files whose contents depend on
how \libflame was configured. \\
% --------------------------------------
{\tt configure.cmd}
&
persistent
&
The script used to prepare the Windows build environment for compiling
\libflamens.
\configurecmd has multiple required arguments, which are explained when
\configurecmd is run with no arguments (or the wrong number of arguments). \\
% --------------------------------------
{\tt dll}
&
build
&
A directory containing the dynamic library files created after compilation. \\
% --------------------------------------
{\tt gendll.cmd}
&
persistent
&
The script used to generate a dynamically-linked library and associated
files from a list of object files.
It is meant to be invoked by \nmake and so normal users should never need to
invoke it manually. \\
% --------------------------------------
{\tt include}
&
build
&
A temporary directory containing copies of the source header files gathered
from the top-level source directory tree. \\
% --------------------------------------
{\tt lib}
&
build
&
A directory containing the static library file created after compilation. \\
% --------------------------------------
{\tt nmake-cc.log}
&
build
&
A file capturing the standard output of the C compiler. \\
% --------------------------------------
{\tt nmake-fc.log}
&
build
&
A file capturing the standard output of the Fortran compiler. \\
% --------------------------------------
{\tt nmake-copy.log}
&
build
&
A file capturing the standard output of the {\tt copy} command line utility. \\
% --------------------------------------
{\tt linkargs.txt}
&
persistent
&
A list of compiler arguments used by \gendll when building a
dynamically-linked library (DLL).
This list includes link options, libraries, and library paths.
For more details on what this file should contain and in what ways
it should be customized by the user, refer to
Section \ref{sec:running-configure-win}. \\
% --------------------------------------
{\tt linkargs64.txt}
&
persistent
&
Similar to {\tt linkargs.txt}, but for use when generating 64-bit
object code.
To use this file to generate a 64-bit DLL, simply rename this
file to {\tt linkargs.txt} before invoking the \dll target.
The user may also use the file contents as a reference when
determining the compiler arguments needed to link an application
against a static 64-bit build of \libflamens. \\
% --------------------------------------
{\tt obj}
&
build
&
A directory containing the object files created during compilation. \\
% --------------------------------------
{\tt revision}
&
build/persistent
&
A file containing the subversion revision number of the source code. \\
% --------------------------------------
{\tt src}
&
build
&
A temporary directory containing copies of the source code files gathered from
the top-level source directory tree.
% --------------------------------------
\end{tabular}
\end{center}
\caption{A list of the files and directories the user can expect to find in the
\windows build directory along with descriptions.
Files marked ``persistent'' should always exist while files marked
``build'' are build products created by the build system.
This latter group of files may be safely removed by invoking the \nmake
target \distcleanns.
}
\label{fig:top-level-win-files}
\end{table}


There is a single configuration option that must be set in the
{\tt build$\bs$defs.mk}:
\begin{itemize}
\item {\bf Verboseness.}
\libflame for Windows may be compiled in verbose mode, in which actual commands
are echoed to the command line instead of the more consise output that the user
sees by default.
In order to compile in verbose mode, the variable {\tt VERBOSE} must be defined.
Thus, you may enable verbose mode by uncommenting the following line:
\begin{Verbatim}[frame=single,framesep=2.5mm,xleftmargin=5mm,fontsize=\footnotesize]
# VERBOSE = 1
\end{Verbatim}
This feature is disabled by default.
\end{itemize}

\begin{table}
\begin{center}
\begin{tabular}{|l|l|p{3.5in}|}
\hline
{\bf Argument} & {\bf Accepted Values} & {\bf Consequence} \\ \hline
% --------------------------------------
{\em architecture string}
&
{\em any string}
&
This string is inserted into the filename of the final library.
It has no effect on how \libflame is built.
\\ \hline
% --------------------------------------
%\multirow{2}{*}{
%{\em build type}
%}
{\em build type}
&
{\tt debug}
&
Enables debugging symbols and disables all compiler optimizations. \\ \cline{2-3}
&
{\tt release}
&
Disables debugging symbols and enables maximum compiler optimizations. \\ \hline
% --------------------------------------
{\em C compiler string}
&
{\tt icl}
&
Compile C source code with the Intel C/C++ compiler, {\tt icl}.
Also, if a DLL is built, use {\tt icl} to perform the linking. \\ \cline{2-3}
&
{\tt cl}
&
Compile C source code with the Microsoft C/C++ compiler, {\tt cl}.
Also, if a DLL is built, use {\tt cl} to perform the linking. \\ \hline
% --------------------------------------
\end{tabular}
\end{center}
\caption{
The arguments expected by \configurecmdns.
}
\label{fig:configure-arguments}
\end{table}


\subsection{IronPython}

Users of IronPython will need to manually change \configurecmd in order for the
script to run correctly.
If you are relying on IronPython as your Python implementation, edit the
\configurecmd file and change the following lines:

\begin{Verbatim}[frame=single,framesep=2.5mm,xleftmargin=5mm,fontsize=\footnotesize]
set GEN_CHECK_REV_FILE=.\build\gen-check-rev-file.py
set GATHER_SRC=.\build\gather-src-for-windows.py
set GEN_CONFIG_FILE=.\build\gen-config-file.py
\end{Verbatim}

\noindent
to:

\begin{Verbatim}[frame=single,framesep=2.5mm,xleftmargin=5mm,fontsize=\footnotesize]
set GEN_CHECK_REV_FILE=ipy .\build\gen-check-rev-file.py
set GATHER_SRC=ipy .\build\gather-src-for-windows.py
set GEN_CONFIG_FILE=ipy .\build\gen-config-file.py
\end{Verbatim}

\noindent
Also, be sure that the {\tt PATH} environment variable is set to contain the
path to your IronPython installation.


\subsection{Running \configurecmd}
\label{sec:running-configure-win}

\index{\libflame!for Microsoft Windows!running \configurecmdns}

Once all configuration options are set, the user may run the \configurecmd
script.
The \configurecmd script takes three mandatory arguments, which are described
in Table \ref{fig:configure-arguments}.
Usage information can also be found by running \configurecmd with no
arguments.
%The first is a string identifying the architecture.
%This string is inserted into the name of the final build products and thus
%should be set to something meaningful to the user, especially if several
%different architecture variants of \libflame are to be built.
%The second is a string identifying the type of build, and must be one of
%``debug'' or ``release''.
%Using ``debug'' will cause debugging flags to be passed into the compiler
%while choosing ``release'' will cause \libflame to be built without debugging
%symbols.

The output from running \configurecmd should look something like:

\begin{Verbatim}[frame=single,framesep=2.5mm,xleftmargin=5mm,fontsize=\footnotesize]
C:\field\temp\libflame-r3692\windows>.\configure.cmd x64 debug icl
.\configure.cmd: Checking/updating revision file.
gen-check-rev-file.py: Found export. Checking for revision file...
gen-check-rev-file.py: Revision file found containing revision string "3692". Export is valid snapshot!
.\configure.cmd: Gathering source files into local flat directories.
.\configure.cmd: Creating configure definitions file.
.\configure.cmd: Configuration and setup complete. You may now run nmake.
\end{Verbatim}

Here, we invoked the \configurecmd script with the ``x64'' architecture
string, requested that debugging be enabled (and optimizations be disabled),
and specified {\tt icl} as the C compiler to use for compilation.
The architecture string will be inserted into the library filename to help
the user distinguish between any other subsequent builds.

The \configurecmd script first checks whether the revision file needs
updating.\footnote{If the user is working with a checked-out working copy from
the  \libflame subversion repository, the script will update the file with the
latest revision based on the revision specified within the
{\tt .svn$\bs$entries} file in the top-level \windows directory.}
Then, a helper script gathers the source code from the primary source tree and
places copies within a ``flat'' directory structure inside of a new {\tt src}
subdirectory.
Header files are copied into a new {\tt include} subdirectory.
Finally, a {\tt config.mk} makefile fragment is generated with various important
definitions which will be included by the main \nmake makefile.

Before proceeding to run \nmakens, the user must execute any compiler
environment scripts that may be necessary in order to run the compiler from
the command line.
For example, the Intel C/C++ compiler typically includes a script named which
allows the user to invoke the {\tt icl} compiler command from the Windows
shell prompt.
Note that this step, wherein the user executes any applicable environment
scripts, must be performed sometime before executing \nmakens.

\begin{table}
\begin{center}
\begin{tabular}{lp{5.0in}}
{\bf Target} & {\bf Function} \\
% --------------------------------------
{\tt all}
&
Invoke the \lib target. \\
% --------------------------------------
{\tt lib}
&
Build \libflame as a static library. \\
% --------------------------------------
{\tt dll}
&
Build \libflame as a dynamically-linked library. \\
% --------------------------------------
{\tt install}
&
Invoke the \installlib and \installheaders targets. \\
% --------------------------------------
{\tt install-lib}
&
Invoke the \lib target and then copy the library file to the \lib
subdirectory of the \libflame install path specified in the \makefilens. \\
% --------------------------------------
{\tt install-dll}
&
Invoke the \dll target and then copy the library files to the \dll
subdirectory of the \libflame install path specified in the \makefilens. \\
% --------------------------------------
{\tt install-headers}
&
Copy the \libflame header files to the install path specified in
\makefilens. \\
% --------------------------------------
{\tt help}
&
Output help and usage information. \\
%% --------------------------------------
{\tt clean}
&
Invoke \cleanlog and \cleanbuild targets. \\
% --------------------------------------
{\tt clean-log}
&
Remove any log files present. \\
% --------------------------------------
{\tt clean-config}
&
Remove all products of \configurecmdns.
Namely, remove the \configns, \includens, and \src directories.
Note that invoking the {\tt clean-config} target will require the user to run
\configurecmd again before being able to run any other \nmake target. \\
% --------------------------------------
{\tt clean-build}
&
Remove all products of the compilation portion of the build process.
Namely, remove the \objns, \libns, and \dll directories. \\
% --------------------------------------
{\tt distclean}
&
Invoke \cleanlogns, \cleanconfigns, and \cleanbuild targets. \\
% --------------------------------------
\end{tabular}
\end{center}
\caption{A list of useful \nmake targets defined in the
\makefile for building \libflame for Windows.
Note that not all targets guarantee that action will take place.
Most targets will not fire if \nmake determines that the target is already
up-to-date.
For example, invoking the \cleanbuild target will not remove any object files
if they do not exist.
}
\label{fig:nmake-targets}
\end{table}



\section{Compiling}

\index{\libflame!for Microsoft Windows!compiling}

After running \configurecmd and ensuring the compilers are operational from
the command line, you may run \nmake.
Running \nmake with no target specified causes the \all target to be
invoked implicitly.
Presently, the \all target causes only the static library to be built.

\begin{Verbatim}[frame=single,framesep=2.5mm,xleftmargin=5mm,fontsize=\footnotesize]
C:\field\temp\libflame-r3692\windows>nmake
\end{Verbatim}

%Alternatively, the user may invoke the \lib and/or \dll targets individually.
%Invoking the \lib target compiles the source code into object files and then
%archives them into a static library.
%Invoking the \dll target compiles the source code and creates a dynamically-linked
%library without first creating the static library.\footnote{Dynamic library
%builds of \libflame are currently considered experimental and not guaranteed
%to link properly.}
\noindent
Table \ref{fig:nmake-targets} lists the most useful \nmake targets defined
in the \makefile that resides in the \windows directory.

As \nmake compiles individual source files into object files, it will output
progress information.
By default (ie: with verbose output disabled), this appears as:

\begin{Verbatim}[frame=single,framesep=2.5mm,xleftmargin=5mm,fontsize=\footnotesize]
C:\field\temp\libflame-r3692\windows>nmake

Microsoft (R) Program Maintenance Utility Version 9.00.30729.01
Copyright (C) Microsoft Corporation.  All rights reserved.

nmake: Creating .\obj\flamec\x64\debug directory
nmake: Compiling .\src\flamec\bli_amax.c
nmake: Compiling .\src\flamec\bli_asum.c
nmake: Compiling .\src\flamec\bli_axpy.c
nmake: Compiling .\src\flamec\bli_axpymt.c
nmake: Compiling .\src\flamec\bli_axpysmt.c
nmake: Compiling .\src\flamec\bli_axpysv.c
nmake: Compiling .\src\flamec\bli_axpyv.c
nmake: Compiling .\src\flamec\bli_check.c
nmake: Compiling .\src\flamec\bli_conjm.c
nmake: Compiling .\src\flamec\bli_conjmr.c
nmake: Compiling .\src\flamec\bli_conjv.c
\end{Verbatim}

\noindent
When compilation is complete, the library will be archived.
The output will appear as:

\begin{Verbatim}[frame=single,framesep=2.5mm,xleftmargin=5mm,fontsize=\footnotesize]
nmake: Creating .\lib\x64\debug directory
nmake: Creating static library .\lib\x64\debug\libflame-x64-r3692.lib
\end{Verbatim}

\noindent
As you can see, the ``x64'' architecture string (provided at configure-time)
and ``r3692'' revision string were inserted into the final library name.
\libflame is still under heavy development and undergoes frequent changes, and
so the revision string is helpful for obvious reasons.
Recall that the architecture string is completely arbitrary and has no effect
on how the library gets built.
However, it should be set to something reasonable to help you remember which
environment was used to compile \libflamens.


\section{Installation}

\index{\libflame!for Microsoft Windows!installing}

Upon creation, the static library file resides in a subdirectory of the \lib
directory, depending on the architecture and build type strings given to
\configurecmdns.

\begin{Verbatim}[frame=single,framesep=2.5mm,xleftmargin=5mm,fontsize=\footnotesize]
C:\field\temp\libflame-r3692\windows>dir lib\x64\debug
 Volume in drive C has no label.
 Volume Serial Number is B4E3-D9FC

 Directory of C:\field\temp\libflame-r3692\windows\lib\x64\debug

12/01/2009  01:19 PM    <DIR>          .
12/01/2009  01:19 PM    <DIR>          ..
12/01/2009  01:19 PM        45,800,190 libflame-x64-r3692.lib
               1 File(s)     45,800,190 bytes
               2 Dir(s)  85,181,444,096 bytes free
\end{Verbatim}

Once library has been built, it may be copied out manually for use by the
application developer.
Alternatively, the user may specify an installation directory in the
{\tt build$\bs$defs.mk} file by setting the following variable:\footnote{
Of course, if the user is going to invoke an \install target, he should first
verify that he has permission to access and write to the directory specified
in {\tt build$\bs$defs.mk}.
Otherwise, the file copy will fail.
}
\begin{Verbatim}[frame=single,framesep=2.5mm,xleftmargin=5mm,fontsize=\footnotesize]
INSTALL_PREFIX = c:\field\lib
\end{Verbatim}

\noindent
After this variable is set, the \nmake install target may be invoked.
This results in the static library being built, if it was not already, and
then copied to its destination directory, specified by the
{\tt INSTALL\_PREFIX} \nmake variable.
The \install target also copies the \libflame header files.

\begin{Verbatim}[frame=single,framesep=2.5mm,xleftmargin=5mm,fontsize=\footnotesize]
C:\field\temp\libflame-r3692\windows>nmake install

Microsoft (R) Program Maintenance Utility Version 9.00.30729.01
Copyright (C) Microsoft Corporation.  All rights reserved.

nmake: Installing .\lib\x64\debug\libflame-x64-r3692.lib to c:\field\lib\libflame\lib
nmake: Installing libflame header files to c:\field\lib\libflame\include-x64-r3692
\end{Verbatim}

At this point, the static library and header files are ready to use.



\section{Dynamic library generation}
\label{sec:dll}

The Windows build system is equipped to optionally generate a
dynamically-linked library (DLL).
At the time of this writing, \libflame developers consider the DLL generation
to be experimental and likely to not work.
Still, we provide instructions in this section for intrepid users,
or experts who wish to tinker and/or provide us with feedback.

After running \configurecmdns, edit the contents of the \linkargs file.
This file should be modified to include (1) any linker options the user
may need or want, (2) a list of system libraries necessary for
successful linking, (3) a list of library paths to add to the list
used when the aforementioned libraries are being searched for by the linker,
and (4) a path to the BLAS (and LAPACK if the user enabled external
LAPACK interfaces at configure-time).
The file format is simple; each line is a line passed to the compiler when
it is invoked as a linker.
Simply modify the existing lines, and/or add additional lines if you have
more options, libraries, and/or library paths.
The following is an example of the contents of \linkargsns.

\begin{Verbatim}[frame=single,framesep=2.5mm,xleftmargin=5mm,fontsize=\footnotesize]
/nologo
/LD /MT
/LIBPATH:"C:\Program Files\Microsoft SDKs\Windows\v6.0A\Lib"
/LIBPATH:"C:\Program Files (x86)\Microsoft Visual Studio 9.0\VC\lib"
/nodefaultlib:libcmt /nodefaultlib:libc /nodefaultlib:libmmt
msvcrt.lib
/LIBPATH:"C:\Program Files (x86)\Intel\Compiler\11.1\048\lib\ia32"
/LIBPATH:"C:\Program Files (x86)\Intel\Compiler\11.1\048\mkl\ia32\lib"
mkl_intel_c.lib
mkl_sequential.lib
mkl_core.lib
\end{Verbatim}

\noindent
The \libflame distribution also includes a file named {\tt linkargs64.txt}
which contains the equivalent paths and flags necessary for 64-bit linking:

\begin{Verbatim}[frame=single,framesep=2.5mm,xleftmargin=5mm,fontsize=\footnotesize]
/nologo
/LD /MT 
/LIBPATH:"C:\Program Files\Microsoft SDKs\Windows\v6.0A\Lib\x64"
/LIBPATH:"C:\Program Files (x86)\Microsoft Visual Studio 9.0\VC\lib\amd64"
/nodefaultlib:libcmt /nodefaultlib:libc /nodefaultlib:libmmt
msvcrt.lib
/LIBPATH:"C:\Program Files (x86)\Intel\Compiler\11.1\048\lib\intel64"
/LIBPATH:"C:\Program Files (x86)\Intel\Compiler\11.1\048\mkl\em64t\lib"
mkl_intel_lp64.lib
mkl_sequential.lib
mkl_core.lib
\end{Verbatim}

\begin{table}[th]
\begin{center}
\begin{tabular}{|l|l|p{3.7in}|}
\hline
{\bf Link type} & {\bf Component files} & {\bf Purpose} \\ \hline
% --------------------------------------
%statically-linked
static
&
{\tt libflame-x64-r3692.lib}
&
The static library containing the \libflame object files.
Link to this file when statically linking your application to \libflamens.
\\ \hline
% --------------------------------------
%dynamically-linked
dynamic
&
{\tt libflame-x64-r3692.dll}
&
The dynamic library containing the \libflame object code.
This file is loaded into memory by the operating system at run-time
the first time a dependent program or library references \libflame
symbols.
\\ \cline{2-3}
&
{\tt libflame-x64-r3692.lib}
&
The import library.
This file contains information such as the dynamic library filename and which
symbols are available within the dynamic library.
The import library is used by the linker at link-time to resolve all function
calls referenced by the application being built.
If you plan to use a dynamic library build of \libflamens, reference this file
when linking your application.
\\ \cline{2-3}
&
{\tt libflame-x64-r3692.exp}
&
The export file.
This file is necessary only when building other dynamic libraries that
depend on a dynamic library build of \libflamens.
\\ \hline
% --------------------------------------
\end{tabular}
\end{center}
\caption{
The files generated when building revision r3692 of \libflame as either a static
or dynamic library.
The filenames reflect using ``x64'' as the architecture string when running
\configurecmdns.
}
\label{fig:library-files}
\end{table}


\noindent
Simply replace the contents of {\tt linkargs.txt} with the contents of
{\tt linkargs64.txt} if you wish to generate a 64-bit library.
The file may need some tweaking, depending on your development environment.

%\noindent
%Note the format used when listing library paths.
%Each path must be preceeded by {\tt /link /LIBPATH:} and the path must be
%enclosed in quotes if it contains spaces.
%These paths are typically necessary so that system libraries may be located
%by the compiler at link-time.
%If, when attempting to build a dynamically-linked copy of \libflamens, you
%encounter errors that indicate the compiler could not find certain libraries,
%try locating the libraries manually and then add those directory paths to
%the \linkargs file.

Note that in the above examples we link against MKL.
The dynamic build of \libflame requires a BLAS implementation at the time
the DLL is generated.
This is necessary so the linker can resolve all BLAS symbol references within
\libflame at the time the library is built.
To specify a different BLAS library, simply replace the {\tt /LIBPATH} entries
and {\tt .lib} filenames accordingly.

After building the static library, the user may re-use the object files
to generate the DLL.
Simply invoke the \dll target:

\begin{Verbatim}[frame=single,framesep=2.5mm,xleftmargin=5mm,fontsize=\footnotesize]
C:\field\temp\libflame-r3692\windows>nmake dll

Microsoft (R) Program Maintenance Utility Version 9.00.30729.01
Copyright (C) Microsoft Corporation.  All rights reserved.

nmake: Creating dynamic library .\dll\x64\debug\libflame-x64-r3692.dll
   Creating library libflame-x64-r3692.lib and object libflame-x64-r3692.exp
\end{Verbatim}

The purpose of each file produced for static and dynamic builds of \libflame
is described in Table \ref{fig:library-files}.
The filenames in this table correspond to those that would result from building
revision r3692 with the architecture string ``x64''.

Once generated, the dynamic library files reside in a directory named \dllns:

\begin{Verbatim}[frame=single,framesep=2.5mm,xleftmargin=5mm,fontsize=\footnotesize]
C:\field\temp\libflame-r3692\windows>dir dll\x64\debug
 Volume in drive C has no label.
 Volume Serial Number is B4E3-D9FC

 Directory of C:\field\temp\libflame-r3692\windows\dll\x64\debug

12/01/2009  01:33 PM    <DIR>          .
12/01/2009  01:33 PM    <DIR>          ..
12/01/2009  01:33 PM        19,907,584 libflame-x64-r3692.dll
12/01/2009  01:33 PM               618 libflame-x64-r3692.dll.manifest
12/01/2009  01:33 PM           330,905 libflame-x64-r3692.exp
12/01/2009  01:33 PM           573,048 libflame-x64-r3692.lib
               4 File(s)     20,812,155 bytes
               2 Dir(s)  85,156,536,320 bytes free
\end{Verbatim}

\noindent
%Here, the installation directories of {\tt c:$\bs$field$\bs$lib$\bs$libflame$\bs$lib}
%and {\tt c:$\bs$field$\bs$lib$\bs$libflame$\bs$dll} were specified in
%{\tt build$\bs$defs.mk} as the static and dynamic library install paths,
%respectively.
The user may then invoke the \installdll target to install the DLL files
to the directory specified by {\tt INSTALL\_PREFIX} in
{\tt build$\bs$defs.mk}:

\begin{Verbatim}[frame=single,framesep=2.5mm,xleftmargin=5mm,fontsize=\footnotesize]
C:\field\libflame-wc\windows>nmake install-dll

Microsoft (R) Program Maintenance Utility Version 9.00.30729.01
Copyright (C) Microsoft Corporation.  All rights reserved.

nmake: Installing .\dll\x64\debug\libflame-x64-r3692.dll to c:\field\lib\libflame\dll
nmake: Installing .\dll\x64\debug\libflame-x64-r3692.lib to c:\field\lib\libflame\dll
nmake: Installing .\dll\x64\debug\libflame-x64-r3692.exp to c:\field\lib\libflame\dll
\end{Verbatim}

If you haven't already run the \install target for a static library install,
you'll need to manually invoke the \installheaders target so that the
\libflame header files are copied to the install directory.




\section{Linking against \libflame}
\label{sec:linking-win}

\index{\libflame!for Microsoft Windows!linking against}

This section will show you how to link a Windows build of \libflame with
your existing application.
Let's assume that you've installed \libflame to
{\tt c:$\bs$field$\bs$lib$\bs$libflame}.
Let's also assume that you are building your application from the command
line.\footnote{
We acknowledge that most users will probably be using an integrated
development environment (IDE) to develop their programs.
However, just as \libflame only supports building from the command line,
we will only demonstrate how to link against the library using \nmake and
leave it up to the motivated user to learn how to link against \libflame
from within whatever IDE he wishes.}

In general, you should make the following changes to your application build
process:
\begin{itemize}
\item
{\bf Add the \libflame header directory to the include path of your compiler}.
Usually, this is done by with the {\tt /I} compiler option.
For example, if you configured \libflame r3692 with the ``x64'' build label,
and specified that \configurecmd use
{\tt c:$\bs$field$\bs$lib$\bs$libflame} as the
install prefix, then you would add
{\tt /Ic:$\bs$field$\bs$lib$\bs$libflame$\bs$include-x86-r3021} to the command
line when invoking the compiler.
Strictly speaking, this is only necessary when compiling source code files
that use \libflame symbols or APIs, but it is generally safe to use when compiling
all of your application's source code.
\item
{\bf Add \libflame to the link command that links your application.}
To link against \libflamens, you need to add
{\tt libflame-x64-r3692.lib} to your link command.
\end{itemize}

Now let's give a concrete example of these changes.
Suppose you've been building your application with an \nmake \makefile that
looks something like:

\begin{Verbatim}[frame=single,framesep=2.5mm,xleftmargin=5mm,commandchars=\#\{\},fontsize=\footnotesize]
SRC_PATH     = .
OBJ_PATH     = .
INC_PATH     = .

LIB_HOME     = c:\field\lib
BLAS_LIB     = $(LIB_HOME)\libblas.lib
LAPACK_LIB   = $(LIB_HOME)\liblapack.lib

CC           = cl.exe
LINKER       = link.exe
CFLAGS       = /nologo /O2 /I$(INC_PATH)
LDFLAGS      = /nologo \
               /LIBPATH:"C:\Program Files\Microsoft SDKs\Windows\v6.0A\Lib\x64" \
               /LIBPATH:"C:\Program Files (x86)\Microsoft Visual Studio 9.0\VC\lib\amd64" \
               /nodefaultlib:libcmt /nodefaultlib:libc /nodefaultlib:libmmt \
               msvcrt.lib

MYAPP_OBJS   = main.obj file.obj util.obj proc.obj
MYAPP_BIN    = my_app.exe

{$(SRC_PATH)}.c{$(OBJ_PATH)}.obj:
    $(CC) $(CFLAGS) /c $< /Fo$@

$(MYAPP): $(MYAPP_OBJS)
	$(LINKER) $(MYAPP_OBJS) /Fe$(MYAPP_BIN) $(LDFLAGS) $(LAPACK_LIB) $(BLAS_LIB)

clean:
    del /F /Q $(MYAPP_OBJS) $(MYAPP_BIN)
    del /F /Q *.manifest
\end{Verbatim}

\noindent
To link against \libflame, you should change your \makefile as follows:

\begin{Verbatim}[frame=single,framesep=2.5mm,xleftmargin=5mm,commandchars=\#\{\},fontsize=\footnotesize]
SRC_PATH     = .
OBJ_PATH     = .
INC_PATH     = .

LIB_HOME     = c:\field\lib
BLAS_LIB     = $(LIB_HOME)\libblas.lib
LAPACK_LIB   = $(LIB_HOME)\liblapack.lib

#textcolor{red}{FLAME_HOME   = c:\field\lib\libflame}
#textcolor{red}{FLAME_INC    = $(FLAME_HOME)\include-x64-r3692}
#textcolor{red}{FLAME_LIB    = $(FLAME_HOME)\lib\libflame-x64-r3692.lib}

CC           = cl.exe
LINKER       = link.exe
CFLAGS       = /nologo /O2 /I$(INC_PATH) #textcolor{red}{/I$(FLAME_INC)}
LDFLAGS      = /nologo \
               /LIBPATH:"C:\Program Files\Microsoft SDKs\Windows\v6.0A\Lib\x64" \
               /LIBPATH:"C:\Program Files (x86)\Microsoft Visual Studio 9.0\VC\lib\amd64" \
               /nodefaultlib:libcmt /nodefaultlib:libc /nodefaultlib:libmmt \
               msvcrt.lib

MYAPP_OBJS   = main.obj file.obj util.obj proc.obj
MYAPP_BIN    = my_app.exe

{$(SRC_PATH)}.c{$(OBJ_PATH)}.obj:
    $(CC) $(CFLAGS) /c $< /Fo$@

$(MYAPP): $(MYAPP_OBJS)
	$(LINKER) $(MYAPP_OBJS) /Fe$(MYAPP_BIN) $(LDFLAGS) #textcolor{red}{$(FLAME_LIB)} $(LAPACK_LIB) $(BLAS_LIB)

clean:
    del /F /Q $(MYAPP_BIN) $(MYAPP_OBJS)
    del /F /Q *.manifest
\end{Verbatim}

\noindent
The changes appear in red.

First, we define the locations of \libflame and the \libflame header directory.

Second, we include the location of the \libflame headers to the compilers'
command line options so that the C compiler will be able to perform type
checking against \libflame declarations and prototypes.

Finally, we add the \libflame library to the link command, making sure to
insert it before the LAPACK and BLAS libraries.

Note that we are linking against a static build of \libflamens.
In principle, the user may also link to a dynamically-linked copy of
\libflamens.
However, as mentioned previously, the DLL instantiation of \libflame is
considered experimental and likely to not link properly.
%which had its internal symbols resolved against external libraries, such
%as system libraries and the BLAS, when the library was built.
%Thus, we only continue to link against the BLAS and LAPACK libraries under the
%assumption that the application still makes use of these libraries directly.

%If you want to link your application
%to a static build of \libflamens, take a look at the default contents of
%{\tt linkargs.txt} to get an idea of what link options might be needed to
%successfully link your executable.
%Note that this applies only if you wish to link your application from
%the command line, as IDEs tend to supply the user with most of these link
%options automatically.




%Since you are building \libflame, you probably wish to use it in your
%application.
%This section will show you how to link \libflame with your existing application.
%
%Let's assume that you've installed \libflame to the default location in
%{\tt \$HOME/flame}.
%Let's also assume that you invoked the \installsymlinks target, giving
%you shorthand symbolic links to both \libflame and the directory containing
%header files.
%
%In general, you should make the following changes to your application build
%process:
%\begin{itemize}
%\item
%{\bf Add the \libflame header directory to the include path of your compiler}.
%Usually, this is done by with the {\tt -I} compiler option.
%For example, if you configured \libflame to use {\tt \$HOME/flame} as the
%install prefix, then you would add {\tt -I\$HOME/flame/include} to the command
%line when invoking the compiler.
%Strictly speaking, this is only necessary when compiling source code files
%that use \libflame symbols or APIs, but it is generally safe to use when compiling
%all of your application's source code.
%\item
%{\bf Add \libflame to the link command that links your application.}
%If you only wish to use the native \libflame API, then you only need to add
%{\tt libflame.a} to your link command.
%However, note that {\tt libflame.a} {\em must} appear in front of the
%LAPACK and BLAS libraries.
%This is because the linker only searches for symbols in the ``current''
%archive and those that appear further down in the link command.
%Placing \libflame after LAPACK or the BLAS will result in undefined symbol
%errors at link-time.
%\item
%{\bf Use the recommended linker flags detected by \configurens.}
%This topic was previously alluded to toward the end of Section
%\ref{sec:running-configure}.
%It is often the case that you must add various linker flags to the link
%command in order to properly link your application with \libflamens.
%This is usually the result of the compilers embedding certain low-level
%functions into the object code.
%These functions may only be resolved at link-time if the library in which
%they are defined is also provided to the linker.
%The list of linker flags that you will need is displayed when \configure
%finished and exits.
%After \configure is run, you may also find these linker flags in the
%\postconfigure script, as described near the end of
%Section \ref{sec:running-configure}.
%\end{itemize}


