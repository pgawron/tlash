\chapter{Developer Application Programming Interfaces}
\label{chapter:dev-apis}

This chapter describes APIs of interest to developers of \libflamens,
including advanced users seeking to extend existing functionality to suit
their own application.






\section{Locks}

\index{developer APIs!locks}

There are a few instances, most notably within the \libflame implementation
of SuperMatrix, where locks are needed to ensure that certain data structures
are updated synchronously by multiple threads.
\libflame abstracts the implementation of the locking mechanism from the user
by exporting a general interface that operates upon an internally defined
\flalock structure.
This structure contains all of the information needed to identify the actual
lock, however it may be defined by the implementation.
\libflame will define its implementation in terms of whichever multithreading
interface is enabled.
See Section \ref{sec:configure-options} for more information on how to specify
the type of multithreading at configure-time.

Note that this API provides only basic locking functionality.
The functions do {\em not} return any status value, and thus the caller
cannot check whether the function succeeded or not.
This was done to simplify the implementation, and also because our primary
application, SuperMatrix, did not require the ability to recover from the
kinds of errors that might occur beyond the control of the user, such as
system errors.


\subsection{API}

% --- FLA_Lock_init() ----------------------------------------------------------

\begin{flaspec}
\begin{verbatim}
void FLA_Lock_init( FLA_Lock* lock_ptr );
\end{verbatim}
\purpose{
Initialize the lock structure pointed to by \flalockns.
Upon successful return, the state of the lock becomes initialized and
unlocked.
}
\notes{
Attempting to initialize a lock that has already been initialized (and
not yet released) may result in undefined behavior.
}
\begin{checks}
\checkitem
\lockptr may not be \fnullns.
%\itemvsp
%\checkitem
% 
\end{checks}
\begin{params}
\parameter{\flalockp}{lock\_ptr}{A pointer to an \flalock structure.}
\end{params}
\end{flaspec}

% --- FLA_Lock_acquire() -------------------------------------------------------

\begin{flaspec}
\begin{verbatim}
void FLA_Lock_acquire( FLA_Lock* lock_ptr );
\end{verbatim}
\purpose{
Attempt to acquire the lock pointed to by \lockptrns.
If the lock is unavailable (if its state is already locked), the call
blocks and returns only upon successful acquisition of the lock.
}
\begin{checks}
\checkitem
\lockptr may not be \fnullns.
%\itemvsp
%\checkitem
% 
\end{checks}
\begin{params}
\parameter{\flalockp}{lock\_ptr}{A pointer to an \flalock structure.}
\end{params}
\end{flaspec}

% --- FLA_Lock_release() -------------------------------------------------------

\begin{flaspec}
\begin{verbatim}
void FLA_Lock_release( FLA_Lock* lock_ptr )
\end{verbatim}
\purpose{
Release the lock associated with the structure pointed to by \flalockns.
}
\notes{
Attempting to release a lock that is uninitialized or that has not yet
been acquired may result in undefined behavior.
}
\begin{checks}
\checkitem
\lockptr may not be \fnullns.
%\itemvsp
%\checkitem
% 
\end{checks}
\begin{params}
\parameter{\flalockp}{lock\_ptr}{A pointer to an \flalock structure.}
\end{params}
\end{flaspec}

% --- FLA_Lock_destroy() -------------------------------------------------------

\begin{flaspec}
\begin{verbatim}
void FLA_Lock_destroy( FLA_Lock* lock_ptr );
\end{verbatim}
\purpose{
Destroy the the lock structure pointed to by \flalockns.
This causes all system resources associated with the lock that had been
previously allocated by {\tt FLA\_Lock\_init()} to be freed.
Upon returning, the state of the lock becomes uninitialized.
}
\notes{
Attempting to destroy a lock that is currently in the locked state
may result in undefined behavior.
}
\begin{checks}
\checkitem
\lockptr may not be \fnullns.
%\itemvsp
%\checkitem
% 
\end{checks}
\begin{params}
\parameter{\flalockp}{lock\_ptr}{A pointer to an \flalock structure.}
\end{params}
\end{flaspec}






\section{Memory management}

\index{developer APIs!memory allocation}

% --- FLA_malloc() -------------------------------------------------------------

\begin{flaspec}
\begin{verbatim}
void* FLA_malloc( size_t size );
\end{verbatim}
\purpose{
Request a pointer to \size bytes of heap-allocated memory from the system.
Note that a value of zero for \size will guarantee that \fnull is returned.
}
\notes{
The programmer is encouraged to use {\tt FLA\_malloc()} instead of calling
{\tt malloc()} directly.
Using {\tt FLA\_malloc()} and {\tt FLA\_free()} allows \libflame to output
via standard error the balance of allocations and releases when
{\tt FLA\_Finalize()} is called, which provides a basic memory leak
detection.
}
\implnotes{
If \libflame was configured with {\tt --enable-memory-alignment={\em N}},
then memory will be allocated using {\tt posix\_memalign()} using {\em N}
as the alignment factor.
Otherwise, {\tt malloc()} is used, which typically only guarantees
memory alignment at 8-byte boundaries.
}
\caveats{
If by chance {\tt malloc()} (or {\tt posix\_memalign()}) fails to allocate
the requested number of bytes, the library raises an abort signal
and the program is ended.
This may seem like overkill, and it probably is.
But it ensures that this situation does not go unreported.
Besides, in the unlikely event that {\tt malloc()} does return a \fnull
pointer, it is most likely because the memory heap is exhausted, which
would prevent most programs from running correctly (or at all).
}
\rvalue{
A \voidp pointer to a heap-allocated region of memory \size bytes long.
}
\begin{params}
\parameter{\sizet}{size}{The number of bytes to allocate.}
\end{params}
\end{flaspec}

% --- FLA_realloc() ------------------------------------------------------------

\begin{flaspec}
\begin{verbatim}
void* FLA_realloc( void* old_ptr, size_t size );
\end{verbatim}
\purpose{
Request a reallocation of previously-allocated memory such that the new
region is \size bytes in length and contains the original contents of the
reigion pointed to by {\tt old\_ptr}.
}
\implnotes{
This function does not guarantee adherence to the library-wide memory
alignment factor set during configuration via the
{\tt --enable-memory-alignment={\em N}} option, if it was given.
We fundamentally cannot implement our own version of {\tt realloc()}
in user-space because we cannot know how much memory was allocated
at {\tt old\_ptr}.
This information is needed if the original contents are to be copied
over to the new memory region.
Thus, {\tt FLA\_realloc()} is implemented with {\tt realloc()}, which
guarantees only 8-byte memory alignment on most systems.
}
\rvalue{
A \voidp pointer to a heap-allocated region of memory \size bytes long.
}
\begin{params}
\parameter{\voidp}{old\_ptr}{A pointer to the region of memory the user wishes to reallocate to \size bytes.}
\parameter{\sizet}{size}{The number of bytes to allocate for the new region.}
\end{params}
\end{flaspec}

% --- FLA_free() ---------------------------------------------------------------

\begin{flaspec}
\begin{verbatim}
void FLA_free( void* ptr );
\end{verbatim}
\purpose{
Release a heap-allocated region of memory back to the system.
Note that passing a value of \fnull for \ptr will cause {\tt FLA\_free()}
to return immediately without performing any action.
}
\notes{
The programmer is encouraged to use {\tt FLA\_free()} instead of calling
{\tt free()} directly.
Using {\tt FLA\_malloc()} and {\tt FLA\_free()} allows \libflame to output
via standard error the balance of allocations and releases when
{\tt FLA\_Finalize()} is called, which provides a basic memory leak
detection.
}
\begin{params}
\parameter{\voidp}{ptr}{A pointer to a region of memory previously allocated by {\tt FLA\_malloc()} (or {\tt FLA\_realloc()}).}
\end{params}
\end{flaspec}






\section{Object creation}

\index{developer APIs!object creation}

% --- FLA_Obj_create_ext() -----------------------------------------------------

\begin{flaspec}
\begin{verbatim}
FLA_Error FLA_Obj_create_ext( FLA_Datatype datatype, FLA_Elemtype elemtype, dim_t m,
                              dim_t n, FLA_Obj* obj );
\end{verbatim}
\purpose{
Create a new object using an extended FLASH-aware interface.
Upon returning, \obj points to a valid heap-allocated $ m \by n $ object.
}
\notes{
The total size of the underlying allocated array depends on the value of
\elemtypens.
If the elements are requested to be \flascalarns, then the size of each
element is determined by the value of \datatypens.
If the elements are of type \flamatrixns, then each element is allocated
to store an \flaobjns.
}
\rvalue{
\flasuccess
}
\begin{params}
\parameter{\fladatatype}{datatype}{A constant corresponding to the numerical datatype requested.} 
\parameter{\flaelemtype}{elemtype}{A constant corresponding to the object element type requested.} 
\parameter{\dimt}{m}{The number of rows to be created in new object.}
\parameter{\dimt}{n}{The number of columns to be created in the new object.}
\parminout{\flaobjp}{obj}{A pointer to an uninitialized \flaobjns.}
                         {A pointer to a new \flaobj parameterized by {\tt m}, {\tt n},
                          \elemtypens, and \datatypens.}
\end{params}
\end{flaspec}






\section{SuperMatrix}
\label{sec:sm-dev}

\index{developer APIs!SuperMatrix}

% --- FLASH_Queue_init() -------------------------------------------------------

\begin{flaspec}
\begin{verbatim}
void FLASH_Queue_init( void );
\end{verbatim}
\purpose{
Initialize SuperMatrix.
This function is normally called from within {\tt FLA\_Init()}.
}
\end{flaspec}

% --- FLASH_Queue_finalize() ---------------------------------------------------

\begin{flaspec}
\begin{verbatim}
void FLASH_Queue_finalize( void );
\end{verbatim}
\purpose{
Finalize SuperMatrix.
This function is normally called from within {\tt FLA\_Finalize()}.
}
\end{flaspec}

% --- FLASH_Queue_get_thread_id() ----------------------------------------------

\begin{flaspec}
\begin{verbatim}
unsigned int FLASH_Queue_get_thread_id( void );
\end{verbatim}
\purpose{
Query the ID number of the calling thread.
The thread ID ranges from zero to $ t - 1 $ where $ t $ is the total number of
SuperMatrix threads, equal to the unsigned integer returned by
{\tt FLASH\_Queue\_get\_num\_threads()}.
}
\devnotes{
This function is not yet implemented.
}
\rvalue{
An unsigned integer representing the thread ID number of the calling thread.
}
\end{flaspec}

% --- FLASH_Queue_get_num_write_blocks() ---------------------------------------
%
%\begin{flaspec}
%\begin{verbatim}
%unsigned int FLASH_Queue_get_num_write_blocks( void );
%\end{verbatim}
%\purpose{
%Query the number of unique blocks (associated with the current set of enqueued
%tasks) that are scheduled to be overwritten during the computation.
%}
%\devnotes{
%This function is not used by the implementation.
%}
%\rvalue{
%An unsigned integer representing the number of blocks to be overwritten.
%}
%\end{flaspec}











\section{Control trees}
\label{sec:cntl-trees}


\subsection{Motivation}

\index{control trees!motivation}

While \libflame was in its early stages of development, we encountered a basic
and recurring problem:
coding blocked FLAME/C algorithms forced us to statically specify the
implementation used when invoking algorithmic subproblems.
Sometimes we chose to hard-code a function call an unblocked implementation
that was itself coded with the FLAME/C API.
Other times we chose to invoke external BLAS and/or unblocked LAPACK
implementations via FLAME wrappers.
And still other situations, such as those encountered when using FLASH and
algorithms-by-blocks, call for us to invoke yet another blocked
routine, creating more than one level of ``recursion'' in the overall
algorithm.

The problem becomes more unavoidable.
Consider that not all situations call for using the same set of algorithmic
variants.
Perhaps for small standalone instances of {\sc syrk}, variant 5 works
well, but when the {\sc syrk} operation is a subproblem within a larger
Cholesky factorization, then variant 2 is more appropriate.
How can we handle both of these cases while statically coding the
choice of implementation for blocked algorithm subproblems?
The most straightforward and naive solution would be to duplicate the
code as many times as there are different situations.
It is not difficult to see that this would quickly become a nightmare for
the maintainers of the library.

So, in summary, we wish to code our algorithms in such a manner that a
subproblem operation is specified {\em without} binding it to a
particular implementation, and in such a manner that we may only
maintain {\em one} copy of each blocked algorithm.



\subsection{The solution in \libflame}

\index{control trees!description}

The aforementioned problem is addressed in \libflame using a technique we
call {\em control trees}.
The general idea is simple: encode control information {\em a priori}
into a tree structure that is passed into algorithmic subproblems and
decoded by internal functions that remain hidden from the user-level API.
This approach has five separate but related aspects, all of which require
us to extend the original FLAME/C API in some way:

\begin{itemize}
\item
{\bf Define control tree node structures and API.}
Our solution is built upon a tree structure where internal nodes
encapsulate information used by blocked algorithms and leaf nodes
specify external implementations.
Specifically, all individual control tree node structures are defined
with blocksize and variant fields, which allow us to specify which
algorithmic variant to execute and what blocksize to use within that
algorithm.
Each structure will also have an matrix type field which will allow
us to handle both flat and arbitrary depth hierarchical matrices.
We define a control tree structure for each operation that we wish
to support (\flasyrkt for {\sc syrk}, \flacholt for {\sc chol}, etc.)
These structures contain the standard three fields and also some number
of fields which may contain pointers to child nodes.
The number and types of these fields depend on the operation for which
the control tree is being defined, but in general, the set of fields
present will be enough to handle all algorithmic variants provided by
\libflamens.
Lastly, we must define an API that allows the programmer to easily
create individual control tree structures and build trees up from child
nodes.
\item
{\bf Control trees created at library startup.}
Control trees are created dynamically via the structure creation
interface and stored on the heap.
The default set used internally by \libflame is instantiated at the time
that the library is initialized, with one class of trees being configured
for flat matrices, and another for hierarchical matrices.
Within each class, several different trees may be created for a given
operation, depending on the desired execution characteristics.
A different tree may created for problem sizes deemed to be ``small''
and those considered to be ``large'', which would vary as a function
of the blocksize and cache size.
Alternatively, if the tree is structured properly, the same tree
may be used for all problem sizes and shapes with only a very small
additional cost in overhead incurred from the extra levels of blocked
algorithms.
Pointers to the root nodes of the trees are stored in global variable
space.
The default set of control trees is destroyed at library shutdown, in
{\tt FLA\_Finalize()}.
\item
{\bf Control tree is selected within operation front-ends.}
The default set of control trees are used within operation ``front-end''
routines.
These routines are defined as user-level computational routines for Level-3
BLAS and LAPACK-level operations that are designed for ``global'' problems,
not subproblems.
In other words, front-ends are for end users only and are not called
internally by \libflame developers.
Examples of front-end routines are: \flasyrkns, \flatrsmns, \flacholns, and
\flaspdinvns.
The root node pointers are accessed via {\tt extern} declarations within the
files that define the front-end routines.
There, the appropriate tree is selected, if more than one exists, and the
root node pointer is passed down into the operation's internal ``back-end''.
\item
{\bf Internal back-ends handle parameters and decode trees.}
Internal back-end functions must be defined in order to parse and decode
the control tree for a particular operation.
The implementation handles cases where (1) computation should execute
immediately, such as for flat matrices or for the leaf matrices of
hierarchical matricies when SuperMatrix is disabled, (2) more recursion
is necessary, in order to handle arbitrary depth hierarchical matrices,
and (3) the library should enqueue tasks for parallel execution via
SuperMatrix.
The back-end also parses the parameters defined by the operation, such
as \sidens, \uplons, and \trans arguments, and then the appropriate
variant or external implementation is called depending on the \variant
field of the control tree node.
In the context of flat matrices, a \variant field equal to \flasubproblem
refers to a node that induces execution of an external implementation.
For hierarchical matricies, the \flasubproblem variant refers to nodes that
may or may not cause further recursion, and reuse of the control tree, to
reach the leaf levels of the hieararchy.
Otherwise, if a blocked algorithmic variant is called, the control tree is
passed on.
\item
{\bf Algorithmic subproblems invoke internal back-ends.}
The modified blocked algorithms feature a \cntl control tree pointer
argument instead of the integer \nbalg argument.
Recall that the pointer will refer to a control tree structure which
contains all the information necessary to specify the execution of the
blocked algorithmic variant, including the blocksize.
The statement which computes the blocksize is replaced by one which
uses a new routine, {\tt FLA\_Determine\_blocksize()}.
The subproblems are replaced with corresponding calls to the internal
back-end routine for the operation in question.
C preprocessor macros are used to access the fields within the \cntl
argument, specifically to extract the blocksize argument and the 
pointers to the child nodes of the current control tree node.
The child nodes are passed in as the last argument to the internal
back-ends, and recursion continues.
\end{itemize}



\subsection{Structure fields}

There are three fields common to every control tree, regardless of its type.

\begin{figure}[t]
\begin{center}
\begin{tabular}{|c|c|c|}
\hline
% --------------------------------------
{\sc gemm} & {\sc symm} & {\sc syrk} \\ \hline
% --------------------------------------
\begin{minipage}[t]{1.5in}
{\tiny
\begin{verbatim}
struct FLA_Gemm_s
{
    FLA_Exec           exec_type;
    int                variant;
    fla_blocksize_t*   blocksize;
    struct FLA_Gemm_s* sub_gemm;
};
typedef struct FLA_Gemm_s fla_gemm_t;
\end{verbatim}
}
\end{minipage}
&
\begin{minipage}[t]{1.5in}
{\tiny
\begin{verbatim}
struct FLA_Symm_s
{
    FLA_Exec           exec_type;
    int                variant;
    fla_blocksize_t*   blocksize;
    struct FLA_Symm_s* sub_symm;
    struct FLA_Gemm_s* sub_gemm1;
    struct FLA_Gemm_s* sub_gemm2;
};
typedef struct FLA_Symm_s fla_symm_t;
\end{verbatim}
}
\end{minipage}
&
\begin{minipage}[t]{1.5in}
{\tt \tiny
\begin{verbatim}
struct FLA_Syrk_s
{
    FLA_Exec           exec_type;
    int                variant;
    fla_blocksize_t*   blocksize;
    struct FLA_Syrk_s* sub_syrk;
    struct FLA_Gemm_s* sub_gemm;
};
typedef struct FLA_Syrk_s fla_syrk_t;
\end{verbatim}
}
\end{minipage}
\\
% --------------------------------------
 & & \\ \hline
% --------------------------------------
{\sc trsm} & {\sc chol} & {\sc trinv} \\ \hline
% --------------------------------------
\begin{minipage}[t]{1.5in}
{\tiny
\begin{verbatim}
struct FLA_Trsm_s
{
    FLA_Exec           exec_type;
    int                variant;
    fla_blocksize_t*   blocksize;
    struct FLA_Trsm_s* sub_trsm;
    struct FLA_Gemm_s* sub_gemm;
};
typedef struct FLA_Trsm_s fla_trsm_t;
\end{verbatim}
}
\end{minipage}
&
\begin{minipage}[t]{1.5in}
{\tiny
\begin{verbatim}
struct FLA_Chol_s
{
    FLA_Exec           exec_type;
    int                variant;
    fla_blocksize_t*   blocksize;
    struct FLA_Chol_s* sub_chol;
    struct FLA_Syrk_s* sub_syrk;
    struct FLA_Herk_s* sub_herk;
    struct FLA_Trsm_s* sub_trsm;
    struct FLA_Gemm_s* sub_gemm;
};
typedef struct FLA_Chol_s fla_chol_t;
\end{verbatim}
}
\end{minipage}
&
\begin{minipage}[t]{1.5in}
{\tt \tiny
\begin{verbatim}
struct FLA_Trinv_s
{
    FLA_Exec            exec_type;
    int                 variant;
    fla_blocksize_t*    blocksize;
    struct FLA_Trinv_s* sub_trinv;
    struct FLA_Gemm_s*  sub_gemm;
    struct FLA_Trmm_s*  sub_trmm;
    struct FLA_Trsm_s*  sub_trsm1;
    struct FLA_Trsm_s*  sub_trsm2;
};
typedef struct FLA_Trinv_s fla_trinv_t;
\end{verbatim}
}
\end{minipage}
\\
% --------------------------------------
 & & \\ \hline
% --------------------------------------
\end{tabular}
\end{center}
\caption{
Control tree structure definitions for various operations.
}
\label{fig:cntl-tree-structs}
\end{figure}


\begin{itemize}
\item
{\tt matrix\_type}.
The \matrixtype field denotes the type of matrices, flat or
hierarchical, that the control tree is to assume.
A matrix type of \flahier allows hierarchical matrices, where the
matrices may be of arbitrary depth.
A matrix type of \flaflat implies that the matrix operands will be flat
matrix objects.
Though \libflame objects contain the \elemtype field in each node in the
matrix hierarchy, the \matrixtype field is needed because the control trees
for flat and hierarchical matricies differ.
Control trees for flat matrices explicitly prescribe the execution for every
level of algorithmic recursion.
However, control trees for hierarchical matrices must allow recursion to
an arbitrary depth.
Thus, we leave it to the internal back-end to detect when the leaves of the
hierarchy are reached and stop recursion accordingly.
Note that the \matrixtype field of every node in the control tree must be
identical.
\item
{\tt variant}.
The \variant field specifies which algorithmic variant should execute.
Valid values are \flasubproblem and \flablockedvariantone
through \flablockedvarianttwentyns.
Note that the semantic meaning of \flasubproblem differs depending on whether
the tree is configured for flat or hierarchical matrices.
For control trees of matrix type \flaflat, \flasubproblem refers to a node
that invokes the operation's external implementation.
For trees of matrix type \flahier, \flasubproblem refers to a node that
causes the back-end to either recurse further if the bottom of the hierarchy
has not yet been reached, or execute (or enqueue) the subproblem otherwise.
Native unblocked FLAME algorithm implementations are not yet supported.
\item
{\tt blocksize}.
The \blocksize field specifies a structure which contains four blocksize
values.
This allows the user to specify different blocksizes depending on the
numerical datatype being used.
blocksize structures should be created with {\tt FLA\_Blocksize\_create()}.
\end{itemize}

Control tree nodes also contain fields which are unique to the operation
type.
Specifically, a control tree type is made unique by the number and type
of sub-tree fields it contains.
Each control tree structure has at least one pointer to another control
tree node.
These child nodes contain the relevant information for executing the
the operation's algorithmic subproblems.
For example, the \flasyrkt type allows for two child nodes, one for a
{\sc syrk} subproblem and one for a {\sc gemm} subproblem.
Figure \ref{fig:cntl-tree-structs} lists the code defining the
structure for the \flasyrkt type along a sample of structures
for other operations supported in \libflamens.

There are two circumstances under which you may leave a subproblem
field uninitialized.
\begin{itemize}
\item
{\bf The algorithmic variant does not need the full set of subproblems.}
Some algorithmic variants only use a subset of the subproblem fields made
available in the control tree structure.
For example, variants 5 and 6 of the {\sc syrk} algorithm only invoke
smaller {\sc syrk} subproblems, and thus do not need to perform any
{\sc gemm} subproblems.
In this case, the \subgemm field is not referenced by the runtime system
and thus it may safely be initialized to \fnullns.
\item
{\bf The control tree node's variant field is \flasubproblemns.}
The nodes at the ``leaves'' of the tree should contain a \variant field
with a special value: \flasubproblemns.
Every flat matrix control tree contains at least one leaf node that
indicates that blocked algorithm recursion should stop, and an
external implementation should be invoked for that node.
Likewise, every hierarchical matrix control tree contains at least one
``recurse'' node where further recursion in the algorithm-by-blocks is
performed if the matrix hierarchy requires it.
In both cases, {\em none} of the subproblem fields are referenced, and thus
they may all be safely initialized to \fnullns.
\end{itemize}

When a control tree contains more than one subproblem field for a given
operation type, the order of the subproblems matters.
Consider the {\sc symm} operation and its corresponding control tree
structure.
All ten variants of {\sc symm} contain one {\sc symm} subproblem, and
blocked variants 1 through 8 contain two {\sc gemm} subproblems.
So control tree nodes for blocked variants 1 through 8
must also initialize both {\sc gemm} fields.
But which {\sc gemm} field corresponds to which {\sc gemm} subproblem
instance in the {\sc symm} algorithm?
In situations like this, the mapping is simple: the \subgemmo field
corresponds the {\sc gemm} subproblem that occurs first in {\sc symm}
algorithm, while the \subgemmtw field corresponds to the {\sc gemm}
subproblem that occurs second.
This rule for disambiguating subproblems of identical operation types
must be observed for all cases where the algorithm contains more than
one subproblem of a particular operation.


\subsection{Control tree API}

\index{developer APIs!control trees}

First, we present the interface for creating and manipulating blocksize
structures, which are a necessary component of control tree nodes.

\subsubsection{Blocksize structures}

% --- FLA_Blocksize_create() ---------------------------------------------------

\begin{flaspec}
\begin{verbatim}
fla_blocksize_t* FLA_Blocksize_create( dim_t b_s,
                                       dim_t b_d,
                                       dim_t b_c,
                                       dim_t b_z );
\end{verbatim}
\purpose{
Create a structure containing a set of four blocksizes, one for each of the
numerical datatypes supported by \libflamens, and initialize the structure
fields according to the function arguments.
}
\ifacenotes{
Though the interface allows the programmer to set blocksize values for all
four datatypes, it is permissible to assign meaningful values to only the
fields that correspond to the datatypes that will be used by the application.
}
\notes{
Blocksize structures are allocated on the heap and should be released
with {\tt FLA\_Blocksize\_free()} when they are no longer needed.
}
\begin{checks}
\checkitem
None of the blocksize arguments may be zero.
%\itemvsp
%\checkitem
% 
\end{checks}
\rvalue{
A pointer to a heap-allocated \flablocksizet structure.
}
\begin{params}
\parameter{\dimt}{b\_s}{The blocksize to use for single precision real data.}
\parameter{\dimt}{b\_d}{The blocksize to use for double precision real data.}
\parameter{\dimt}{b\_c}{The blocksize to use for single precision complex data.}
\parameter{\dimt}{b\_z}{The blocksize to use for double precision complex data.}
\end{params}
\end{flaspec}

% --- FLA_Blocksize_set() ------------------------------------------------------

\begin{flaspec}
\begin{verbatim}
void FLA_Blocksize_set( fla_blocksize_t* bp,
                        dim_t            b_s,
                        dim_t            b_d,
                        dim_t            b_c,
                        dim_t            b_z );
\end{verbatim}
\purpose{
Set the individual fields in an existing blocksize structure.
}
\ifacenotes{
Providing a value of zero for one of the blocksize arguments causes the
function to leave the existing value unchanged for that particular
blocksize field.
}
\begin{checks}
\checkitem
\bp may not be \fnullns.
%\itemvsp
%\checkitem
% 
\end{checks}
\begin{params}
\parameter{\flablocksizet}{bp}{A pointer to an existing blocksize structure.}
\parameter{\dimt}{b\_s}{The blocksize to use for single precision real data.}
\parameter{\dimt}{b\_d}{The blocksize to use for double precision real data.}
\parameter{\dimt}{b\_c}{The blocksize to use for single precision complex data.}
\parameter{\dimt}{b\_z}{The blocksize to use for double precision complex data.}
\end{params}
\end{flaspec}

% --- FLA_Blocksize_scale() ----------------------------------------------------

\begin{flaspec}
\begin{verbatim}
void FLA_Blocksize_scale( fla_blocksize_t* bp,
                          double           factor );
\end{verbatim}
\purpose{
Scale the individual fields of an existing blocksize structure.
}
\implnotes{
The scaling occurs as follows: the blocksize fields are typecast to
\cdoublens, then multiplied by the scaling {\tt factor}, and finally
typecast back to \cint before being stored to the blocksize structure.
}
\begin{checks}
\checkitem
\bp may not be \fnullns.
%\itemvsp
%\checkitem
% 
\end{checks}
\begin{params}
\parameter{\flablocksizet}{bp}{A pointer to an existing blocksize structure.}
\parameter{\cdouble}{factor}{The scaling factor to apply to the blocksize values.}
\end{params}
\end{flaspec}

% --- FLA_Blocksize_create_copy() ----------------------------------------------

\begin{flaspec}
\begin{verbatim}
fla_blocksize_t* FLA_Blocksize_create_copy( fla_blocksize_t* bp );
\end{verbatim}
\purpose{
Create a copy of an existing blocksize structure.
}
\begin{checks}
\checkitem
\bp may not be \fnullns.
%\itemvsp
%\checkitem
% 
\end{checks}
\rvalue{
A pointer to a heap-allocated \flablocksizet structure.
}
\begin{params}
\parameter{\flablocksizet}{bp}{A pointer to an existing blocksize structure.}
\end{params}
\end{flaspec}

% --- FLA_Blocksize_free() -----------------------------------------------------

\begin{flaspec}
\begin{verbatim}
void FLA_Blocksize_free( fla_blocksize_t* bp );
\end{verbatim}
\purpose{
Release the memory allocated to a blocksize structure.
}
\notes{
{\tt FLA\_Blocksize\_free()} should only be used with pointers to blocksize
structures that were allocated with {\tt FLA\_Blocksize\_create()} or
{\tt FLA\_Blocksize\_create\_copy()}.
}
\begin{checks}
\checkitem
\bp may not be \fnullns.
%\itemvsp
%\checkitem
% 
\end{checks}
\begin{params}
\parameter{\flablocksizet}{bp}{A pointer to an existing blocksize structure.}
\end{params}
\end{flaspec}

% --- FLA_Blocksize_extract() --------------------------------------------------

\begin{flaspec}
\begin{verbatim}
dim_t FLA_Blocksize_extract( FLA_Datatype     datatype,
                             fla_blocksize_t* bp );
\end{verbatim}
\purpose{
Extract the value associated with a specific numerical datatype from a
blocksize structure.
}
\begin{checks}
\checkitem
The value of \datatype must refer to a floating-point datatype.
\itemvsp
\checkitem
\bp may not be \fnullns.
\end{checks}
\rvalue{
An unsigned integer value of type \dimtns.
}
\begin{params}
\parameter{\fladatatype}{datatype}{A constant corresponding to the numerical datatype requested.} 
\parameter{\flablocksizet}{bp}{A pointer to an existing blocksize structure.}
\end{params}
\end{flaspec}

% --- FLA_Query_blocksizes() ---------------------------------------------------

\begin{flaspec}
\begin{verbatim}
fla_blocksize_t* FLA_Query_blocksizes( FLA_Dimension dim_tag );
\end{verbatim}
\purpose{
Query the library for a reasonable set of blocksizes.
The user must specify how the blocksizes are chosen by specifying a
dimension {\em tag}.
Valid tag values are \fladimensionmns, \fladimensionkns,
\fladimensionnns, and \fladimensionminns.
If \libflame was configured with {\tt --enable-goto-interfaces},
the first three values correspond to architecture-specific blocksizes
associated with the $ m $, $ k $, and $ n $ dimensions of the inner-most
matrix-matrix multiplication kernel in GotoBLAS.
Otherwise, these three constants are associated with default values that
may not be optimal.
The last tag, \fladimensionminns, will cause {\tt FLA\_Query\_blocksizes()}
to return the smallest of the $ m $, $ k $, and $ n $ blocksizes.
If unsure, use \fladimensionminns.
}
\notes{
This function dynamically allocates memory for the structure in which
the blocksizes are returned.
It is the user's responsibility to deallocate this structure with
{\tt FLA\_Blocksize\_free()} when it is no longer needed.
}
\rvalue{
A pointer to a heap-allocated \flablocksizet structure.
}
\begin{params}
\parameter{\fladimension}{dim\_tag}{A constant specifying how to choose the blocksize.} 
\end{params}
\end{flaspec}

% --- FLA_Query_blocksize() ----------------------------------------------------

\begin{flaspec}
\begin{verbatim}
dim_t FLA_Query_blocksize( FLA_Datatype  datatype,
                           FLA_Dimension dim_tag );
\end{verbatim}
\purpose{
Query the library for a reasonable blocksize for a specific datatype.
The behavior of this function is similar to that of
{\tt FLA\_Query\_blocksizes()},
except that only a single \dimt scalar (for the datatype in question)
is returned intead of a pointer to an entire \flablocksizet structure.
}
\notes{
The values returned by this function are the same as those attainable
by calling {\tt FLA\_Query\_blocksizes()} and then using
{\tt FLA\_Blocksize\_extract()} to extract the blocksize for \datatypens.
}
\begin{checks}
\checkitem
The value of \datatype must refer to a floating-point datatype.
%\itemvsp
%\checkitem
% 
\end{checks}
\rvalue{
An unsigned integer value of type \dimtns.
}
\begin{params}
\parameter{\fladatatype}{datatype}{A constant corresponding to the numerical datatype requested.} 
\parameter{\fladimension}{dim\_tag}{A constant specifying how to choose the blocksize.} 
\end{params}
\end{flaspec}

% --- FLA_Determine_blocksize() ------------------------------------------------

\begin{flaspec}
\begin{verbatim}
dim_t FLA_Determine_blocksize( FLA_Obj          A_unproc,
                               FLA_Quadrant     to_dir,
                               fla_blocksize_t* bp );
\end{verbatim}
\purpose{
Determine the blocksize given the contents of a blocksize structure and
the current state of the matrix partitioning.
If the blocksize is larger than the dimension of {\tt A\_unproc},
the dimension of {\tt A\_unproc} is returned instead.
In this case, the dimension in question, be it the length or width, is
determined by the value of {\tt to\_dir}.
Specifically, if {\tt to\_dir} denotes vertical movement, the length of
{\tt A\_unproc} is used, and for lateral movement, the width is used.
If {\tt to\_dir} denotes diagonal movement, then the minimum dimension
(as would be returned by {\tt FLA\_Obj\_min\_dim()}) is used.
}
\begin{checks}
\checkitem
\bp may not be \fnullns.
%\itemvsp
%\checkitem
% 
\end{checks}
\rvalue{
An unsigned integer value of type \dimtns.
}
\begin{params}
\parameter{\flaobj}{A\_unproc}{An \flaobj view into the unprocessed portion of a matrix being tracked by a blocked algorithm.}
\parameter{\flaquadrant}{to\_dir}{The direction in which the algorithm is moving through the parent matrix of {\tt A\_unproc}.}
\parameter{\flablocksizet}{bp}{A pointer to an existing blocksize structure.}
\end{params}
\end{flaspec}




The remainder of this subsection describes the functions that create and
initialize control tree nodes for each of the supported linear algebra
operations.
These functions share the same first three arguments, which correspond
to the fields described in the previous subsection.
Note that you should always invoke these interfaces for the leaves of
the trees first, and then use the pointers returned from those routines
in the initialization of higher internal nodes.
Simply put, you cannot create a non-leaf node until you have created
and initialized its children nodes.



\subsubsection{Level-3 BLAS operations}

% --- FLA_Cntl_gemm_obj_create() -----------------------------------------------

\begin{flaspec}
\begin{verbatim}
fla_gemm_t* FLA_Cntl_gemm_obj_create( FLA_Matrix_type  matrix_type,
                                      int              variant,
                                      fla_blocksize_t* blocksize,
                                      fla_gemm_t*      sub_gemm );
\end{verbatim}
\purpose{
Create a structure representing a node in a control tree for a general
matrix-matrix multiplication ({\sc gemm}) operation and initialize its
fields according to the function arguments.
}
\notes{
If \variant is \flasubproblemns, none of the pointer arguments
are used and thus they may be safely set to \fnullns.
%The \blocksize and \subgemm arguments may be \fnull if \variant is
%\flasubproblemns.
}
\begin{checks}
\checkitem
If \variant is not \flasubproblemns, then it must be one of
\flablockedvariantone through \flablockedvariantsixns.
%\itemvsp
%\checkitem
% 
\end{checks}
\rvalue{
A pointer to a heap-allocated \flagemmt structure.
}
\begin{params}
\parameter{\matrixtype}{matrix\_type}{The type of matrix (flat or hierarchical) to support in the control tree in which the node will be used.}
\parameter{\int}{variant}{A constant value indicating the choice of variant for executing the computation associated with the control tree node being created.}
\parameter{\flablocksizet}{blocksize}{A pointer to a blocksize structure to be used for the node being created.}
\parameter{\flahemmt}{sub\_gemm}{A pointer to the node to be used for the {\sc gemm} subproblem.}
\end{params}
\end{flaspec}

% --- FLA_Cntl_hemm_obj_create() -----------------------------------------------

\begin{flaspec}
\begin{verbatim}
fla_hemm_t* FLA_Cntl_hemm_obj_create( FLA_Matrix_type  matrix_type,
                                      int              variant,
                                      fla_blocksize_t* blocksize,
                                      fla_hemm_t*      sub_hemm,
                                      fla_gemm_t*      sub_gemm1,
                                      fla_gemm_t*      sub_gemm2 );
\end{verbatim}
\purpose{
Create a structure representing a node in a control tree for a Hermitian
matrix-matrix multiplication ({\sc hemm}) operation and initialize its
fields according to the function arguments.
}
\notes{
If \variant is \flasubproblemns, none of the pointer arguments
are used and thus they may be safely set to \fnullns.
Even if \variant specifies a blocked variant, some algorithms contain
fewer subproblems and thus do not use every subproblem field
argument.
In such cases, these arguments may be safely set to \fnullns.
Please refer to the blocked algorithmic variant implementations
to determine which subproblem fields are unused. 
%If \variant is \flablockedvariantnine or \flablockedvarianttenns, the
%\subgemmo and \subgemmtw arguments may be \fnullns.
%If \variant is \flasubproblemns, the \blocksizens, \subhemmns,
%\subgemmons, and \subgemmtw arguments may be \fnullns.
}
\begin{checks}
\checkitem
If \variant is not \flasubproblemns, then it must be one of
\flablockedvariantone through \flablockedvarianttenns.
%\itemvsp
%\checkitem
% 
\end{checks}
\rvalue{
A pointer to a heap-allocated \flahemmt structure.
}
\begin{params}
\parameter{\matrixtype}{matrix\_type}{The type of matrix (flat or hierarchical) to support in the control tree in which the node will be used.}
\parameter{\int}{variant}{A constant value indicating the choice of variant for executing the computation associated with the control tree node being created.}
\parameter{\flablocksizet}{blocksize}{A pointer to a blocksize structure to be used for the node being created.}
\parameter{\flahemmt}{sub\_hemm}{A pointer to the node to be used for the {\sc hemm} subproblem.}
\parameter{\flagemmt}{sub\_gemm1}{A pointer to the node to be used for the first {\sc gemm} subproblem.}
\parameter{\flagemmt}{sub\_gemm2}{A pointer to the node to be used for the second {\sc gemm} subproblem.}
\end{params}
\end{flaspec}

% --- FLA_Cntl_herk_obj_create() -----------------------------------------------

\begin{flaspec}
\begin{verbatim}
fla_herk_t* FLA_Cntl_herk_obj_create( FLA_Matrix_type  matrix_type,
                                      int              variant,
                                      fla_blocksize_t* blocksize,
                                      fla_herk_t*      sub_herk,
                                      fla_gemm_t*      sub_gemm );
\end{verbatim}
\purpose{
Create a structure representing a node in a control tree for a Hermitian
rank-k update ({\sc herk}) operation and initialize its
fields according to the function arguments.
}
\notes{
If \variant is \flasubproblemns, none of the pointer arguments
are used and thus they may be safely set to \fnullns.
Even if \variant specifies a blocked variant, some algorithms contain
fewer subproblems and thus do not use every subproblem field
argument.
In such cases, these arguments may be safely set to \fnullns.
Please refer to the blocked algorithmic variant implementations
to determine which subproblem fields are unused. 
%If \variant is \flablockedvariantfive or \flablockedvariantsixns, the
%\subgemm argument may be \fnullns.
%If \variant is \flasubproblemns, the \blocksizens, \subherkns,
%and \subgemm arguments may be \fnullns.
}
\begin{checks}
\checkitem
If \variant is not \flasubproblemns, then it must be one of
\flablockedvariantone through \flablockedvariantsixns.
%\itemvsp
%\checkitem
% 
\end{checks}
\rvalue{
A pointer to a heap-allocated \flaherkt structure.
}
\begin{params}
\parameter{\matrixtype}{matrix\_type}{The type of matrix (flat or hierarchical) to support in the control tree in which the node will be used.}
\parameter{\int}{variant}{A constant value indicating the choice of variant for executing the computation associated with the control tree node being created.}
\parameter{\flablocksizet}{blocksize}{A pointer to a blocksize structure to be used for the node being created.}
\parameter{\flaherkt}{sub\_herk}{A pointer to the node to be used for the {\sc herk} subproblem.}
\parameter{\flagemmt}{sub\_gemm}{A pointer to the node to be used for the {\sc gemm} subproblem.}
\end{params}
\end{flaspec}

% --- FLA_Cntl_her2k_obj_create() ----------------------------------------------

\begin{flaspec}
\begin{verbatim}
fla_her2k_t* FLA_Cntl_her2k_obj_create( FLA_Matrix_type  matrix_type,
                                        int              variant,
                                        fla_blocksize_t* blocksize,
                                        fla_her2k_t*     sub_her2k,
                                        fla_gemm_t*      sub_gemm1,
                                        fla_gemm_t*      sub_gemm2 );
\end{verbatim}
\purpose{
Create a structure representing a node in a control tree for a Hermitian
rank-2k update ({\sc her2k}) operation and initialize its
fields according to the function arguments.
}
\notes{
If \variant is \flasubproblemns, none of the pointer arguments
are used and thus they may be safely set to \fnullns.
Even if \variant specifies a blocked variant, some algorithms contain
fewer subproblems and thus do not use every subproblem field
argument.
In such cases, these arguments may be safely set to \fnullns.
Please refer to the blocked algorithmic variant implementations
to determine which subproblem fields are unused. 
%If \variant is \flablockedvariantnine or \flablockedvarianttenns, the
%\subgemmo and \subgemmtw arguments may be \fnullns.
%If \variant is \flasubproblemns, the \blocksizens, \subhertkns,
%\subgemmons, and \subgemmtw arguments may be \fnullns.
}
\begin{checks}
\checkitem
If \variant is not \flasubproblemns, then it must be one of
\flablockedvariantone through \flablockedvarianttenns.
%\itemvsp
%\checkitem
% 
\end{checks}
\rvalue{
A pointer to a heap-allocated \flahertkt structure.
}
\begin{params}
\parameter{\matrixtype}{matrix\_type}{The type of matrix (flat or hierarchical) to support in the control tree in which the node will be used.}
\parameter{\int}{variant}{A constant value indicating the choice of variant for executing the computation associated with the control tree node being created.}
\parameter{\flablocksizet}{blocksize}{A pointer to a blocksize structure to be used for the node being created.}
\parameter{\flahertkt}{sub\_her2k}{A pointer to the node to be used for the {\sc her2k} subproblem.}
\parameter{\flagemmt}{sub\_gemm1}{A pointer to the node to be used for the first {\sc gemm} subproblem.}
\parameter{\flagemmt}{sub\_gemm2}{A pointer to the node to be used for the second {\sc gemm} subproblem.}
\end{params}
\end{flaspec}

% --- FLA_Cntl_symm_obj_create() -----------------------------------------------

\begin{flaspec}
\begin{verbatim}
fla_symm_t* FLA_Cntl_symm_obj_create( FLA_Matrix_type  matrix_type,
                                      int              variant,
                                      fla_blocksize_t* blocksize,
                                      fla_symm_t*      sub_symm,
                                      fla_gemm_t*      sub_gemm1,
                                      fla_gemm_t*      sub_gemm2 );
\end{verbatim}
\purpose{
Create a structure representing a node in a control tree for a symmetric
matrix-matrix multiplication ({\sc symm}) operation and initialize its
fields according to the function arguments.
}
\notes{
If \variant is \flasubproblemns, none of the pointer arguments
are used and thus they may be safely set to \fnullns.
Even if \variant specifies a blocked variant, some algorithms contain
fewer subproblems and thus do not use every subproblem field
argument.
In such cases, these arguments may be safely set to \fnullns.
Please refer to the blocked algorithmic variant implementations
to determine which subproblem fields are unused. 
%If \variant is \flablockedvariantnine or \flablockedvarianttenns, the
%\subgemmo and \subgemmtw arguments may be \fnullns.
%If \variant is \flasubproblemns, the \blocksizens, \subsymmns,
%\subgemmons, and \subgemmtw arguments may be \fnullns.
}
\begin{checks}
\checkitem
If \variant is not \flasubproblemns, then it must be one of
\flablockedvariantone through \flablockedvarianttenns.
%\itemvsp
%\checkitem
% 
\end{checks}
\rvalue{
A pointer to a heap-allocated \flasymmt structure.
}
\begin{params}
\parameter{\matrixtype}{matrix\_type}{The type of matrix (flat or hierarchical) to support in the control tree in which the node will be used.}
\parameter{\int}{variant}{A constant value indicating the choice of variant for executing the computation associated with the control tree node being created.}
\parameter{\flablocksizet}{blocksize}{A pointer to a blocksize structure to be used for the node being created.}
\parameter{\flasymmt}{sub\_symm}{A pointer to the node to be used for the {\sc symm} subproblem.}
\parameter{\flagemmt}{sub\_gemm1}{A pointer to the node to be used for the first {\sc gemm} subproblem.}
\parameter{\flagemmt}{sub\_gemm2}{A pointer to the node to be used for the second {\sc gemm} subproblem.}
\end{params}
\end{flaspec}

% --- FLA_Cntl_syrk_obj_create() -----------------------------------------------

\begin{flaspec}
\begin{verbatim}
fla_syrk_t* FLA_Cntl_syrk_obj_create( FLA_Matrix_type  matrix_type,
                                      int              variant,
                                      fla_blocksize_t* blocksize,
                                      fla_syrk_t*      sub_syrk,
                                      fla_gemm_t*      sub_gemm );
\end{verbatim}
\purpose{
Create a structure representing a node in a control tree for a symmetric
rank-k update ({\sc syrk}) operation and initialize its
fields according to the function arguments.
}
\notes{
If \variant is \flasubproblemns, none of the pointer arguments
are used and thus they may be safely set to \fnullns.
Even if \variant specifies a blocked variant, some algorithms contain
fewer subproblems and thus do not use every subproblem field
argument.
In such cases, these arguments may be safely set to \fnullns.
Please refer to the blocked algorithmic variant implementations
to determine which subproblem fields are unused. 
%If \variant is \flablockedvariantfive or \flablockedvariantsixns, the
%\subgemm argument may be \fnullns.
%If \variant is \flasubproblemns, the \blocksizens, \subherkns,
%and \subgemm arguments may be \fnullns.
}
\begin{checks}
\checkitem
If \variant is not \flasubproblemns, then it must be one of
\flablockedvariantone through \flablockedvariantsixns.
%\itemvsp
%\checkitem
% 
\end{checks}
\rvalue{
A pointer to a heap-allocated \flasyrkt structure.
}
\begin{params}
\parameter{\matrixtype}{matrix\_type}{The type of matrix (flat or hierarchical) to support in the control tree in which the node will be used.}
\parameter{\int}{variant}{A constant value indicating the choice of variant for executing the computation associated with the control tree node being created.}
\parameter{\flablocksizet}{blocksize}{A pointer to a blocksize structure to be used for the node being created.}
\parameter{\flasyrkt}{sub\_syrk}{A pointer to the node to be used for the {\sc syrk} subproblem.}
\parameter{\flagemmt}{sub\_gemm}{A pointer to the node to be used for the {\sc gemm} subproblem.}
\end{params}
\end{flaspec}

% --- FLA_Cntl_syr2k_obj_create() ----------------------------------------------

\begin{flaspec}
\begin{verbatim}
fla_syr2k_t* FLA_Cntl_syr2k_obj_create( FLA_Matrix_type  matrix_type,
                                        int              variant,
                                        fla_blocksize_t* blocksize,
                                        fla_syr2k_t*     sub_syr2k,
                                        fla_gemm_t*      sub_gemm1,
                                        fla_gemm_t*      sub_gemm2 );
\end{verbatim}
\purpose{
Create a structure representing a node in a control tree for a symmetric
rank-2k update ({\sc syr2k}) operation and initialize its
fields according to the function arguments.
}
\notes{
If \variant is \flasubproblemns, none of the pointer arguments
are used and thus they may be safely set to \fnullns.
Even if \variant specifies a blocked variant, some algorithms contain
fewer subproblems and thus do not use every subproblem field
argument.
In such cases, these arguments may be safely set to \fnullns.
Please refer to the blocked algorithmic variant implementations
to determine which subproblem fields are unused. 
%If \variant is \flablockedvariantnine or \flablockedvarianttenns, the
%\subgemmo and \subgemmtw arguments may be \fnullns.
%If \variant is \flasubproblemns, the \blocksizens, \subsyrtkns,
%\subgemmons, and \subgemmtw arguments may be \fnullns.
}
\begin{checks}
\checkitem
If \variant is not \flasubproblemns, then it must be one of
\flablockedvariantone through \flablockedvarianttenns.
%\itemvsp
%\checkitem
% 
\end{checks}
\rvalue{
A pointer to a heap-allocated \flasyrtkt structure.
}
\begin{params}
\parameter{\matrixtype}{matrix\_type}{The type of matrix (flat or hierarchical) to support in the control tree in which the node will be used.}
\parameter{\int}{variant}{A constant value indicating the choice of variant for executing the computation associated with the control tree node being created.}
\parameter{\flablocksizet}{blocksize}{A pointer to a blocksize structure to be used for the node being created.}
\parameter{\flasyrtkt}{sub\_syr2k}{A pointer to the node to be used for the {\sc syr2k} subproblem.}
\parameter{\flagemmt}{sub\_gemm1}{A pointer to the node to be used for the first {\sc gemm} subproblem.}
\parameter{\flagemmt}{sub\_gemm2}{A pointer to the node to be used for the second {\sc gemm} subproblem.}
\end{params}
\end{flaspec}

% --- FLA_Cntl_trmm_obj_create() -----------------------------------------------

\begin{flaspec}
\begin{verbatim}
fla_trmm_t* FLA_Cntl_trmm_obj_create( FLA_Matrix_type  matrix_type,
                                      int              variant,
                                      fla_blocksize_t* blocksize,
                                      fla_trmm_t*      sub_trmm,
                                      fla_gemm_t*      sub_gemm );
\end{verbatim}
\purpose{
Create a structure representing a node in a control tree for a triangular
matrix-matrix multiplication ({\sc trmm}) operation and initialize its
fields according to the function arguments.
}
\notes{
If \variant is \flasubproblemns, none of the pointer arguments
are used and thus they may be safely set to \fnullns.
Even if \variant specifies a blocked variant, some algorithms contain
fewer subproblems and thus do not use every subproblem field
argument.
In such cases, these arguments may be safely set to \fnullns.
Please refer to the blocked algorithmic variant implementations
to determine which subproblem fields are unused. 
%If \variant is \flablockedvariantthree or \flablockedvariantfourns, the
%\subgemm argument may be \fnullns.
%If \variant is \flasubproblemns, the \blocksizens, \subtrmmns,
%and \subgemm arguments may be \fnullns.
}
\begin{checks}
\checkitem
If \variant is not \flasubproblemns, then it must be one of
\flablockedvariantone through \flablockedvariantfourns.
%\itemvsp
%\checkitem
% 
\end{checks}
\rvalue{
A pointer to a heap-allocated \flatrmmt structure.
}
\begin{params}
\parameter{\matrixtype}{matrix\_type}{The type of matrix (flat or hierarchical) to support in the control tree in which the node will be used.}
\parameter{\int}{variant}{A constant value indicating the choice of variant for executing the computation associated with the control tree node being created.}
\parameter{\flablocksizet}{blocksize}{A pointer to a blocksize structure to be used for the node being created.}
\parameter{\flatrmmt}{sub\_trmm}{A pointer to the node to be used for the {\sc trmm} subproblem.}
\parameter{\flagemmt}{sub\_gemm}{A pointer to the node to be used for the {\sc gemm} subproblem.}
\end{params}
\end{flaspec}

% --- FLA_Cntl_trsm_obj_create() -----------------------------------------------

\begin{flaspec}
\begin{verbatim}
fla_trsm_t* FLA_Cntl_trsm_obj_create( FLA_Matrix_type  matrix_type,
                                      int              variant,
                                      fla_blocksize_t* blocksize,
                                      fla_trsm_t*      sub_trsm,
                                      fla_gemm_t*      sub_gemm );
\end{verbatim}
\purpose{
Create a structure representing a node in a control tree for a triangular
solve with multiple right-hand sides ({\sc trsm}) operation and initialize its
fields according to the function arguments.
}
\notes{
If \variant is \flasubproblemns, none of the pointer arguments
are used and thus they may be safely set to \fnullns.
Even if \variant specifies a blocked variant, some algorithms contain
fewer subproblems and thus do not use every subproblem field
argument.
In such cases, these arguments may be safely set to \fnullns.
Please refer to the blocked algorithmic variant implementations
to determine which subproblem fields are unused. 
%If \variant is \flablockedvariantthree or \flablockedvariantfourns, the
%\subgemm argument may be \fnullns.
%If \variant is \flasubproblemns, the \blocksizens, \subtrsmns,
%and \subgemm arguments may be \fnullns.
}
\begin{checks}
\checkitem
If \variant is not \flasubproblemns, then it must be one of
\flablockedvariantone through \flablockedvariantfourns.
%\itemvsp
%\checkitem
% 
\end{checks}
\rvalue{
A pointer to a heap-allocated \flatrsmt structure.
}
\begin{params}
\parameter{\matrixtype}{matrix\_type}{The type of matrix (flat or hierarchical) to support in the control tree in which the node will be used.}
\parameter{\int}{variant}{A constant value indicating the choice of variant for executing the computation associated with the control tree node being created.}
\parameter{\flablocksizet}{blocksize}{A pointer to a blocksize structure to be used for the node being created.}
\parameter{\flatrsmt}{sub\_trsm}{A pointer to the node to be used for the {\sc trsm} subproblem.}
\parameter{\flagemmt}{sub\_gemm}{A pointer to the node to be used for the {\sc gemm} subproblem.}
\end{params}
\end{flaspec}



\subsubsection{LAPACK-level operations}


% --- FLA_Cntl_chol_obj_create() -----------------------------------------------

\begin{flaspec}
\begin{verbatim}
fla_chol_t* FLA_Cntl_chol_obj_create( FLA_Matrix_type  matrix_type,
                                      int              variant,
                                      fla_blocksize_t* blocksize,
                                      fla_chol_t*      sub_chol,
                                      fla_syrk_t*      sub_syrk,
                                      fla_herk_t*      sub_herk,
                                      fla_trsm_t*      sub_trsm,
                                      fla_gemm_t*      sub_gemm );
\end{verbatim}
\purpose{
Create a structure representing a node in a control tree for a Cholesky
factorization ({\sc chol}) operation and initialize its
fields according to the function arguments.
}
\notes{
If \variant is \flasubproblemns, none of the pointer arguments
are used and thus they may be safely set to \fnullns.
Even if \variant specifies a blocked variant, some algorithms contain
fewer subproblems and thus do not use every subproblem field
argument.
In such cases, these arguments may be safely set to \fnullns.
Please refer to the blocked algorithmic variant implementations
to determine which subproblem fields are unused. 
}
\begin{checks}
\checkitem
If \variant is not \flasubproblemns, then it must be one of
\flablockedvariantone through \flablockedvariantthreens.
%\itemvsp
%\checkitem
% 
\end{checks}
\rvalue{
A pointer to a heap-allocated \flacholt structure.
}
\begin{params}
\parameter{\matrixtype}{matrix\_type}{The type of matrix (flat or hierarchical) to support in the control tree in which the node will be used.}
\parameter{\int}{variant}{A constant value indicating the choice of variant for executing the computation associated with the control tree node being created.}
\parameter{\flablocksizet}{blocksize}{A pointer to a blocksize structure to be used for the node being created.}
\parameter{\flacholt}{sub\_chol}{A pointer to the node to be used for the {\sc chol} subproblem.}
\parameter{\flasyrkt}{sub\_syrk}{A pointer to the node to be used for the {\sc sryk} subproblem.}
\parameter{\flasyrkt}{sub\_herk}{A pointer to the node to be used for the {\sc herk} subproblem.}
\parameter{\flatrsmt}{sub\_trsm}{A pointer to the node to be used for the {\sc trsm} subproblem.}
\parameter{\flatrsmt}{sub\_gemm}{A pointer to the node to be used for the {\sc gemm} subproblem.}
\end{params}
\end{flaspec}

% --- FLA_Cntl_lu_obj_create() -------------------------------------------------

\begin{flaspec}
\begin{verbatim}
fla_lu_t* FLA_Cntl_lu_obj_create( FLA_Matrix_type  matrix_type,
                                  int              variant,
                                  fla_blocksize_t* blocksize,
                                  fla_lu_t*        sub_lu,
                                  fla_gemm_t*      sub_gemm1,
                                  fla_gemm_t*      sub_gemm2,
                                  fla_gemm_t*      sub_gemm3,
                                  fla_trsm_t*      sub_trsm1,
                                  fla_trsm_t*      sub_trsm2 );
\end{verbatim}
\purpose{
Create a structure representing a node in a control tree for an LU
factorization ({\sc lu}) operation and initialize its
fields according to the function arguments.
}
\notes{
If \variant is \flasubproblemns, none of the pointer arguments
are used and thus they may be safely set to \fnullns.
Even if \variant specifies a blocked variant, some algorithms contain
fewer subproblems and thus do not use every subproblem field
argument.
In such cases, these arguments may be safely set to \fnullns.
Please refer to the blocked algorithmic variant implementations
to determine which subproblem fields are unused. 
}
\begin{checks}
\checkitem
If \variant is not \flasubproblemns, then it must be one of
\flablockedvariantone through \flablockedvariantthreens.
%\itemvsp
%\checkitem
% 
\end{checks}
\rvalue{
A pointer to a heap-allocated \flalut structure.
}
\begin{params}
\parameter{\matrixtype}{matrix\_type}{The type of matrix (flat or hierarchical) to support in the control tree in which the node will be used.}
\parameter{\int}{variant}{A constant value indicating the choice of variant for executing the computation associated with the control tree node being created.}
\parameter{\flablocksizet}{blocksize}{A pointer to a blocksize structure to be used for the node being created.}
\parameter{\flacholt}{sub\_chol}{A pointer to the node to be used for the {\sc chol} subproblem.}
\parameter{\flasyrkt}{sub\_syrk}{A pointer to the node to be used for the {\sc sryk} subproblem.}
\parameter{\flasyrkt}{sub\_herk}{A pointer to the node to be used for the {\sc herk} subproblem.}
\parameter{\flatrsmt}{sub\_trsm}{A pointer to the node to be used for the {\sc trsm} subproblem.}
\parameter{\flatrsmt}{sub\_gemm}{A pointer to the node to be used for the {\sc gemm} subproblem.}
\end{params}
\end{flaspec}

% --- FLA_Cntl_qrut_obj_create() -----------------------------------------------

\begin{flaspec}
\begin{verbatim}
fla_qrut_t* FLA_Cntl_qrut_obj_create( FLA_Matrix_type  matrix_type,
                                      int              variant,
                                      fla_blocksize_t* blocksize,
                                      fla_qrut_t*      sub_qrut,
                                      fla_trmm_t*      sub_trmm1,
                                      fla_trmm_t*      sub_trmm2,
                                      fla_gemm_t*      sub_gemm1,
                                      fla_gemm_t*      sub_gemm2,
                                      fla_trsm_t*      sub_trsm,
                                      fla_axpy_t*      sub_axpy,
                                      fla_copy_t*      sub_copy );
\end{verbatim}
\purpose{
Create a structure representing a node in a control tree for a QR
factorization via the UT transform ({\sc qrut}) operation and initialize its
fields according to the function arguments.
}
\notes{
If \variant is \flasubproblemns, none of the pointer arguments
are used and thus they may be safely set to \fnullns.
Even if \variant specifies a blocked variant, some algorithms contain
fewer subproblems and thus do not use every subproblem field
argument.
In such cases, these arguments may be safely set to \fnullns.
Please refer to the blocked algorithmic variant implementations
to determine which subproblem fields are unused. 
}
\begin{checks}
\checkitem
If \variant is not \flasubproblemns, then it must be
\flablockedvariantonens.
%\itemvsp
%\checkitem
% 
\end{checks}
\rvalue{
A pointer to a heap-allocated \flaqrutt structure.
}
\begin{params}
\parameter{\matrixtype}{matrix\_type}{The type of matrix (flat or hierarchical) to support in the control tree in which the node will be used.}
\parameter{\int}{variant}{A constant value indicating the choice of variant for executing the computation associated with the control tree node being created.}
\parameter{\flablocksizet}{blocksize}{A pointer to a blocksize structure to be used for the node being created.}
\parameter{\flaqrutt}{sub\_qrut}{A pointer to the node to be used for the {\sc qrut} subproblem.}
\parameter{\flatrmmt}{sub\_trmm1}{A pointer to the node to be used for the first {\sc trmm} subproblem.}
\parameter{\flatrmmt}{sub\_trmm2}{A pointer to the node to be used for the second {\sc trmm} subproblem.}
\parameter{\flagemmt}{sub\_gemm1}{A pointer to the node to be used for the first {\sc gemm} subproblem.}
\parameter{\flagemmt}{sub\_gemm2}{A pointer to the node to be used for the second {\sc gemm} subproblem.}
\parameter{\flagemmt}{sub\_trsm}{A pointer to the node to be used for the {\sc trsm} subproblem.}
\parameter{\flagemmt}{sub\_axpy}{A pointer to the node to be used for the {\sc axpy} subproblem.}
\parameter{\flagemmt}{sub\_copy}{A pointer to the node to be used for the {\sc copy} subproblem.}
\end{params}
\end{flaspec}

% --- FLA_Cntl_lq_obj_create() -------------------------------------------------

\begin{flaspec}
\begin{verbatim}
fla_lqut_t* FLA_Cntl_lqut_obj_create( FLA_Matrix_type  matrix_type,
                                      int              variant,
                                      fla_blocksize_t* blocksize,
                                      fla_lqut_t*      sub_lqut,
                                      fla_trmm_t*      sub_trmm1,
                                      fla_trmm_t*      sub_trmm2,
                                      fla_gemm_t*      sub_gemm1,
                                      fla_gemm_t*      sub_gemm2,
                                      fla_trsm_t*      sub_trsm,
                                      fla_axpy_t*      sub_axpy,
                                      fla_copy_t*      sub_copy );
\end{verbatim}
\purpose{
Create a structure representing a node in a control tree for a LQ
factorization via the UT transform ({\sc lqut}) operation and initialize its
fields according to the function arguments.
}
\notes{
If \variant is \flasubproblemns, none of the pointer arguments
are used and thus they may be safely set to \fnullns.
Even if \variant specifies a blocked variant, some algorithms contain
fewer subproblems and thus do not use every subproblem field
argument.
In such cases, these arguments may be safely set to \fnullns.
Please refer to the blocked algorithmic variant implementations
to determine which subproblem fields are unused. 
}
\begin{checks}
\checkitem
If \variant is not \flasubproblemns, then it must be
\flablockedvariantonens.
%\itemvsp
%\checkitem
% 
\end{checks}
\rvalue{
A pointer to a heap-allocated \flalqutt structure.
}
\begin{params}
\parameter{\matrixtype}{matrix\_type}{The type of matrix (flat or hierarchical) to support in the control tree in which the node will be used.}
\parameter{\int}{variant}{A constant value indicating the choice of variant for executing the computation associated with the control tree node being created.}
\parameter{\flablocksizet}{blocksize}{A pointer to a blocksize structure to be used for the node being created.}
\parameter{\flalqt}{sub\_lqut}{A pointer to the node to be used for the {\sc lqut} subproblem.}
\parameter{\flatrmmt}{sub\_trmm1}{A pointer to the node to be used for the first {\sc trmm} subproblem.}
\parameter{\flatrmmt}{sub\_trmm2}{A pointer to the node to be used for the second {\sc trmm} subproblem.}
\parameter{\flagemmt}{sub\_gemm1}{A pointer to the node to be used for the first {\sc gemm} subproblem.}
\parameter{\flagemmt}{sub\_gemm2}{A pointer to the node to be used for the second {\sc gemm} subproblem.}
\parameter{\flagemmt}{sub\_trsm}{A pointer to the node to be used for the {\sc trsm} subproblem.}
\parameter{\flagemmt}{sub\_axpy}{A pointer to the node to be used for the {\sc axpy} subproblem.}
\parameter{\flagemmt}{sub\_copy}{A pointer to the node to be used for the {\sc copy} subproblem.}
\end{params}
\end{flaspec}

% --- FLA_Cntl_trinv_obj_create() ----------------------------------------------

\begin{flaspec}
\begin{verbatim}
fla_trinv_t* FLA_Cntl_trinv_obj_create( FLA_Matrix_type  matrix_type,
                                        int              variant,
                                        fla_blocksize_t* blocksize,
                                        fla_trinv_t*     sub_trinv,
                                        fla_trmm_t*      sub_trmm,
                                        fla_trsm_t*      sub_trsm1,
                                        fla_trsm_t*      sub_trsm2,
                                        fla_gemm_t*      sub_gemm );
\end{verbatim}
\purpose{
Create a structure representing a node in a control tree for a triangular
matrix inversion ({\sc trinv}) operation and initialize its
fields according to the function arguments.
}
\notes{
If \variant is \flasubproblemns, none of the pointer arguments
are used and thus they may be safely set to \fnullns.
Even if \variant specifies a blocked variant, some algorithms contain
fewer subproblems and thus do not use every subproblem field
argument.
In such cases, these arguments may be safely set to \fnullns.
Please refer to the blocked algorithmic variant implementations
to determine which subproblem fields are unused. 
}
\begin{checks}
\checkitem
If \variant is not \flasubproblemns, then it must be one of
\flablockedvariantone through \flablockedvariantfourns.
%\itemvsp
%\checkitem
% 
\end{checks}
\rvalue{
A pointer to a heap-allocated \flatrinvt structure.
}
\begin{params}
\parameter{\matrixtype}{matrix\_type}{The type of matrix (flat or hierarchical) to support in the control tree in which the node will be used.}
\parameter{\int}{variant}{A constant value indicating the choice of variant for executing the computation associated with the control tree node being created.}
\parameter{\flablocksizet}{blocksize}{A pointer to a blocksize structure to be used for the node being created.}
\parameter{\flacholt}{sub\_trinv}{A pointer to the node to be used for the {\sc trinv} subproblem.}
\parameter{\flasyrkt}{sub\_trmm}{A pointer to the node to be used for the {\sc trmm} subproblem.}
\parameter{\flasyrkt}{sub\_trsm1}{A pointer to the node to be used for the first {\sc trsm} subproblem.}
\parameter{\flatrsmt}{sub\_trsm2}{A pointer to the node to be used for the second {\sc trsm} subproblem.}
\parameter{\flatrsmt}{sub\_gemm}{A pointer to the node to be used for the {\sc gemm} subproblem.}
\end{params}
\end{flaspec}

% --- FLA_Cntl_spdinv_obj_create() ---------------------------------------------

\begin{flaspec}
\begin{verbatim}
fla_spdinv_t* FLA_Cntl_spdinv_obj_create( FLA_Matrix_type  matrix_type,
                                          int              variant,
                                          fla_blocksize_t* blocksize,
                                          fla_chol_t*      sub_chol,
                                          fla_trinv_t*     sub_trinv,
                                          fla_ttmm_t*      sub_ttmm );
\end{verbatim}
\purpose{
Create a structure representing a node in a control tree for a symmetric
(or Hermitian) positive definite matrix inversion ({\sc spdinv}) operation
and initialize its fields according to the function arguments.
}
\notes{
Since {\sc spdinv} is implemented as the sequence of its three constituent
suboperations, {\sc chol}, {\sc trinv}, and {\sc ttmm} without any matrix
partitioning at the {\sc spdinv} level, the \variant field is not used.
Also, the {\sc spdinv} front-end interprets the \blocksize field as the
cutoff at which to switch from external routines to internal \libflame
variants.
}
%\begin{checks}
%\checkitem
%If \variant is not \flasubproblemns, then it must be one of
%%\itemvsp
%%\checkitem
%% 
%\end{checks}
\rvalue{
A pointer to a heap-allocated \flaspdinvt structure.
}
\begin{params}
\parameter{\matrixtype}{matrix\_type}{The type of matrix (flat or hierarchical) to support in the control tree in which the node will be used.}
\parameter{\int}{variant}{Not referenced.}
\parameter{\flablocksizet}{blocksize}{A pointer to a blocksize structure to be used for the node being created. Note that the front-end interprets these values as the cutoffs at which to switch from external implementations to internal \libflame variants.}
\parameter{\flacholt}{sub\_chol}{A pointer to the node to be used for the {\sc chol} suboperation.}
\parameter{\flatrinvt}{sub\_trinv}{A pointer to the node to be used for the {\sc trinv} suboperation.}
\parameter{\flattmmt}{sub\_ttmm}{A pointer to the node to be used for the {\sc ttmm} suboperation.}
\end{params}
\end{flaspec}

% --- FLA_Cntl_hess_obj_create() -----------------------------------------------
%
%\begin{flaspec}
%\begin{verbatim}
%fla_hess_t* FLA_Cntl_hess_obj_create( FLA_Matrix_type  matrix_type,
%                                      int              variant,
%                                      fla_blocksize_t* blocksize,
%                                      fla_hess_t*      sub_hess,
%                                      fla_trmm_t*      sub_trmm1,
%                                      fla_trmm_t*      sub_trmm2,
%                                      fla_trmm_t*      sub_trmm3,
%                                      fla_trmm_t*      sub_trmm4,
%                                      fla_gemm_t*      sub_gemm1,
%                                      fla_gemm_t*      sub_gemm2,
%                                      fla_gemm_t*      sub_gemm3 );
%\end{verbatim}
%\purpose{
%Create a structure representing a node in a control tree for a reduction
%to upper Hessenberg form ({\sc hess}) operation and initialize its
%fields according to the function arguments.
%}
%\notes{
%If \variant is \flasubproblemns, none of the pointer arguments
%are used and thus they may be safely set to \fnullns.
%Even if \variant specifies a blocked variant, some algorithms contain
%fewer subproblems and thus do not use every subproblem field
%argument.
%In such cases, these arguments may be safely set to \fnullns.
%Please refer to the blocked algorithmic variant implementations
%to determine which subproblem fields are unused. 
%}
%\devnotes{
%The algorithmic variant implementations for reduction to upper
%Hessenberg form do not exist.
%If this operation is needed, use the external wrapper routine
%{\tt FLA\_Hess\_blk\_external()}.
%}
%\begin{checks}
%\checkitem
%If \variant must be \flasubproblem since no blocked variants
%for the operation exist yet.
%%\itemvsp
%%\checkitem
%% 
%\end{checks}
%\rvalue{
%A pointer to a heap-allocated \flahesst structure.
%}
%\begin{params}
%\parameter{\matrixtype}{matrix\_type}{The type of matrix (flat or hierarchical) to support in the control tree in which the node will be used.}
%\parameter{\int}{variant}{A constant value indicating the choice of variant for executing the computation associated with the control tree node being created.}
%\parameter{\flablocksizet}{blocksize}{A pointer to a blocksize structure to be used for the node being created.}
%\parameter{\flacholt}{sub\_hess}{A pointer to the node to be used for the {\sc hess} subproblem.}
%\parameter{\flasyrkt}{sub\_trmm1}{A pointer to the node to be used for the first {\sc trmm} subproblem.}
%\parameter{\flasyrkt}{sub\_trmm2}{A pointer to the node to be used for the second {\sc trmm} subproblem.}
%\parameter{\flasyrkt}{sub\_trmm3}{A pointer to the node to be used for the third {\sc trmm} subproblem.}
%\parameter{\flasyrkt}{sub\_trmm4}{A pointer to the node to be used for the fourth {\sc trmm} subproblem.}
%\parameter{\flatrsmt}{sub\_gemm1}{A pointer to the node to be used for the first {\sc gemm} subproblem.}
%\parameter{\flatrsmt}{sub\_gemm2}{A pointer to the node to be used for the second {\sc gemm} subproblem.}
%\parameter{\flatrsmt}{sub\_gemm3}{A pointer to the node to be used for the third {\sc gemm} subproblem.}
%\end{params}
%\end{flaspec}

% --- FLA_Cntl_ttmm_obj_create() -----------------------------------------------

\begin{flaspec}
\begin{verbatim}
fla_ttmm_t* FLA_Cntl_ttmm_obj_create( FLA_Matrix_type  matrix_type,
                                      int              variant,
                                      fla_blocksize_t* blocksize,
                                      fla_ttmm_t*      sub_ttmm,
                                      fla_syrk_t*      sub_syrk,
                                      fla_herk_t*      sub_herk,
                                      fla_trmm_t*      sub_trmm,
                                      fla_gemm_t*      sub_gemm );
\end{verbatim}
\purpose{
Create a structure representing a node in a control tree for a
triangular-transpose
matrix multiply ({\sc ttmm}) operation and initialize its
fields according to the function arguments.
}
\notes{
If \variant is \flasubproblemns, none of the pointer arguments
are used and thus they may be safely set to \fnullns.
Even if \variant specifies a blocked variant, some algorithms contain
fewer subproblems and thus do not use every subproblem field
argument.
In such cases, these arguments may be safely set to \fnullns.
Please refer to the blocked algorithmic variant implementations
to determine which subproblem fields are unused. 
}
\begin{checks}
\checkitem
If \variant is not \flasubproblemns, then it must be one of
\flablockedvariantone through \flablockedvariantthreens.
%\itemvsp
%\checkitem
% 
\end{checks}
\rvalue{
A pointer to a heap-allocated \flattmmt structure.
}
\begin{params}
\parameter{\matrixtype}{matrix\_type}{The type of matrix (flat or hierarchical) to support in the control tree in which the node will be used.}
\parameter{\int}{variant}{A constant value indicating the choice of variant for executing the computation associated with the control tree node being created.}
\parameter{\flablocksizet}{blocksize}{A pointer to a blocksize structure to be used for the node being created.}
\parameter{\flattmmt}{sub\_ttmm}{A pointer to the node to be used for the {\sc ttmm} subproblem.}
\parameter{\flasyrkt}{sub\_syrk}{A pointer to the node to be used for the {\sc sryk} subproblem.}
\parameter{\flasyrkt}{sub\_herk}{A pointer to the node to be used for the {\sc herk} subproblem.}
\parameter{\flatrsmt}{sub\_trmm}{A pointer to the node to be used for the {\sc trmm} subproblem.}
\parameter{\flatrsmt}{sub\_gemm}{A pointer to the node to be used for the {\sc gemm} subproblem.}
\end{params}
\end{flaspec}

% --- FLA_Cntl_sylv_obj_create() -----------------------------------------------

\begin{flaspec}
\begin{verbatim}
fla_sylv_t* FLA_Cntl_sylv_obj_create( FLA_Matrix_type  matrix_type,
                                      int              variant,
                                      fla_blocksize_t* blocksize,
                                      fla_sylv_t*      sub_sylv1,
                                      fla_sylv_t*      sub_sylv2,
                                      fla_sylv_t*      sub_sylv3,
                                      fla_gemm_t*      sub_gemm1,
                                      fla_gemm_t*      sub_gemm2,
                                      fla_gemm_t*      sub_gemm3,
                                      fla_gemm_t*      sub_gemm4,
                                      fla_gemm_t*      sub_gemm5,
                                      fla_gemm_t*      sub_gemm6,
                                      fla_gemm_t*      sub_gemm7,
                                      fla_gemm_t*      sub_gemm8 );
\end{verbatim}
\purpose{
Create a structure representing a node in a control tree for a triangular
Sylvester equation solve ({\sc sylv}) operation and initialize its
fields according to the function arguments.
}
\notes{
If \variant is \flasubproblemns, none of the pointer arguments
are used and thus they may be safely set to \fnullns.
Even if \variant specifies a blocked variant, some algorithms contain
fewer subproblems and thus do not use every subproblem field
argument.
In such cases, these arguments may be safely set to \fnullns.
Please refer to the blocked algorithmic variant implementations
to determine which subproblem fields are unused. 
}
\begin{checks}
\checkitem
If \variant is not \flasubproblemns, then it must be one of
\flablockedvariantone through \flablockedvarianteightteenns.
%\itemvsp
%\checkitem
% 
\end{checks}
\rvalue{
A pointer to a heap-allocated \flasylvt structure.
}
\begin{params}
\parameter{\matrixtype}{matrix\_type}{The type of matrix (flat or hierarchical) to support in the control tree in which the node will be used.}
\parameter{\int}{variant}{A constant value indicating the choice of variant for executing the computation associated with the control tree node being created.}
\parameter{\flablocksizet}{blocksize}{A pointer to a blocksize structure to be used for the node being created.}
\parameter{\flasylvt}{sub\_sylv1}{A pointer to the node to be used for the first {\sc sylv} subproblem.}
\parameter{\flasylvt}{sub\_sylv2}{A pointer to the node to be used for the second {\sc sylv} subproblem.}
\parameter{\flasylvt}{sub\_sylv3}{A pointer to the node to be used for the third {\sc sylv} subproblem.}
\parameter{\flagemmt}{sub\_gemm1}{A pointer to the node to be used for the first {\sc gemm} subproblem.}
\parameter{\flagemmt}{sub\_gemm2}{A pointer to the node to be used for the second {\sc gemm} subproblem.}
\parameter{\flagemmt}{sub\_gemm3}{A pointer to the node to be used for the third {\sc gemm} subproblem.}
\parameter{\flagemmt}{sub\_gemm4}{A pointer to the node to be used for the fourth {\sc gemm} subproblem.}
\parameter{\flagemmt}{sub\_gemm5}{A pointer to the node to be used for the fifth {\sc gemm} subproblem.}
\parameter{\flagemmt}{sub\_gemm6}{A pointer to the node to be used for the sixth {\sc gemm} subproblem.}
\parameter{\flagemmt}{sub\_gemm7}{A pointer to the node to be used for the seventh {\sc gemm} subproblem.}
\parameter{\flagemmt}{sub\_gemm8}{A pointer to the node to be used for the eighth {\sc gemm} subproblem.}
\end{params}
\end{flaspec}




\subsubsection{Miscellaneous operations}

% --- FLA_Cntl_swap_obj_create() -----------------------------------------------

\begin{flaspec}
\begin{verbatim}
fla_swap_t* FLA_Cntl_swap_obj_create( FLA_Matrix_type  matrix_type,
                                      int              variant,
                                      fla_blocksize_t* blocksize,
                                      fla_swap_t*      sub_swap );
\end{verbatim}
\purpose{
Create a structure representing a node in a control tree for a matrix swap
({\sc swap}) operation and initialize its fields according to the function
arguments.
}
\notes{
If \variant is \flasubproblemns, none of the pointer arguments
are used and thus they may be safely set to \fnullns.
}
\begin{checks}
\checkitem
If \variant is not \flasubproblemns, then it must be either
\flablockedvariantone or \flablockedvarianttwons.
%\itemvsp
%\checkitem
% 
\end{checks}
\rvalue{
A pointer to a heap-allocated \flaswapts structure.
}
\begin{params}
\parameter{\matrixtype}{matrix\_type}{The type of matrix (flat or hierarchical) to support in the control tree in which the node will be used.}
\parameter{\int}{variant}{A constant value indicating the choice of variant for executing the computation associated with the control tree node being created.}
\parameter{\flablocksizet}{blocksize}{A pointer to a blocksize structure to be used for the node being created.}
\parameter{\flaswapt}{sub\_swap}{A pointer to the node to be used for the {\sc swap} subproblem.}
\end{params}
\end{flaspec}

% --- FLA_Cntl_transpose_obj_create() ------------------------------------------

\begin{flaspec}
\begin{verbatim}
fla_transpose_t* FLA_Cntl_transpose_obj_create( FLA_Matrix_type  matrix_type,
                                                int              variant,
                                                fla_blocksize_t* blocksize,
                                                fla_trans_t*     sub_trans,
                                                fla_swap_t*      sub_swap );
\end{verbatim}
\purpose{
Create a structure representing a node in a control tree for a matrix
transposition ({\sc transpose}) operation and initialize its
fields according to the function arguments.
}
\notes{
If \variant is \flasubproblemns, none of the pointer arguments
are used and thus they may be safely set to \fnullns.
}
\begin{checks}
\checkitem
If \variant is not \flasubproblemns, then it must be either
\flablockedvariantone or \flablockedvarianttwons.
%\itemvsp
%\checkitem
% 
\end{checks}
\rvalue{
A pointer to a heap-allocated \flatransposet structure.
}
\begin{params}
\parameter{\matrixtype}{matrix\_type}{The type of matrix (flat or hierarchical) to support in the control tree in which the node will be used.}
\parameter{\int}{variant}{A constant value indicating the choice of variant for executing the computation associated with the control tree node being created.}
\parameter{\flablocksizet}{blocksize}{A pointer to a blocksize structure to be used for the node being created.}
\parameter{\flatransposet}{sub\_trans}{A pointer to the node to be used for the {\sc transpose} subproblem.}
\parameter{\flaswapt}{sub\_swap}{A pointer to the node to be used for the {\sc swap} subproblem.}
\end{params}
\end{flaspec}

% --- FLA_Cntl_obj_free() ------------------------------------------------------

\begin{flaspec}
\begin{verbatim}
void FLA_Cntl_obj_free( void* cntl );
\end{verbatim}
\purpose{
Release the memory allocated for a structure representing a node in a control
tree.
}
\notes{
{\tt FLA\_Cntl\_obj\_free()} should only be used with pointers to control tree
structures that were allocated with the {\tt FLA\_Cntl\_*\_create()} routines.
}
\begin{params}
\parameter{\voidp}{cntl}{A pointer to the node to be freed.}
\end{params}
\end{flaspec}




\subsection{Default control trees}

The default control trees are created when \libflame is initialized via
{\tt FLA\_Init()}.
The subroutines in which the actual creation and initialization takes
place are named according to the operation name and execution type.
For example, control trees for a Cholesky factorization that is to be
executed sequentially with conventinal storage are initialized
in the subroutine {\tt FLA\_Chol\_cntl\_init()}.
Likewise, control trees for a triangular matrix inversion that is to be
executed sequentiall with hierarchial storage are initialized
in {\tt FLASH\_Trinv\_cntl\_init()}.
Figure \ref{fig:cntl-init} shows examples of these routines for
the Cholesky factorizatoin with hierarchical storage, which gives the
reader an idea of how control trees should be initialized.

\begin{figure}[tbp]
\begin{center}
\begin{tabular}{|c|c|}
\hline
% --------------------------------------
{\sc gemm} with hierarchical storage  & {\sc syrk} with hierarchical storage \\ \hline
% --------------------------------------
\begin{minipage}[t]{3in}
{\tiny
\begin{verbatim}
#include "FLAME.h"

fla_gemm_t*      flash_gemm_cntl_blas;
fla_gemm_t*      flash_gemm_cntl_mm_op;
fla_gemm_t*      flash_gemm_cntl_mp_pb;
fla_gemm_t*      flash_gemm_cntl_pm_bp;
fla_gemm_t*      flash_gemm_cntl_op_bp;
fla_gemm_t*      flash_gemm_cntl_pb_bb;
fla_gemm_t*      flash_gemm_cntl_bp_bb;
fla_blocksize_t* flash_gemm_bsize;

void FLASH_Gemm_cntl_init()
{
    // Set gemm blocksize based on hierarchical storage.
    flash_gemm_bsize      = FLA_Blocksize_create( 1, 1, 1, 1 );

    // Create a control tree node that executes a gemm subproblem.
    flash_gemm_cntl_blas  = FLA_Cntl_gemm_obj_create( FLA_EXECUTE_DREF,
                                                      FLA_UNBLOCKED_EXTERN,
                                                      NULL,
                                                      NULL );

    // Create control trees for situations where one dimension is large.
    flash_gemm_cntl_pb_bb = FLA_Cntl_gemm_obj_create( FLA_EXECUTE_ONLY,
                                                      FLA_BLOCKED_VARIANT1,
                                                      flash_gemm_bsize,
                                                      flash_gemm_cntl_blas );
    flash_gemm_cntl_bp_bb = FLA_Cntl_gemm_obj_create( FLA_EXECUTE_ONLY,
                                                      FLA_BLOCKED_VARIANT3,
                                                      flash_gemm_bsize,
                                                      flash_gemm_cntl_blas );

    // Create control trees for situations where two dimensions are large.
    flash_gemm_cntl_op_bp = FLA_Cntl_gemm_obj_create( FLA_EXECUTE_ONLY,
                                                      FLA_BLOCKED_VARIANT1,
                                                      flash_gemm_bsize,
                                                      flash_gemm_cntl_bp_bb );
    flash_gemm_cntl_mp_pb = FLA_Cntl_gemm_obj_create( FLA_EXECUTE_ONLY,
                                                      FLA_BLOCKED_VARIANT5,
                                                      flash_gemm_bsize,
                                                      flash_gemm_cntl_pb_bb );
    flash_gemm_cntl_pm_bp = FLA_Cntl_gemm_obj_create( FLA_EXECUTE_ONLY,
                                                      FLA_BLOCKED_VARIANT5,
                                                      flash_gemm_bsize,
                                                      flash_gemm_cntl_bp_bb );

    // Create control trees for situations where all dimensions are large.
    flash_gemm_cntl_mm_op = FLA_Cntl_gemm_obj_create( FLA_EXECUTE_ONLY,
                                                      FLA_BLOCKED_VARIANT5,
                                                      flash_gemm_bsize,
                                                      flash_gemm_cntl_op_bp );
}
\end{verbatim}
}
\end{minipage}
&
\begin{minipage}[t]{3in}
{\tt \tiny
\begin{verbatim}
#include "FLAME.h"

extern fla_gemm_t* flash_gemm_cntl_pb_bb;

fla_syrk_t*        flash_syrk_cntl_blas;
fla_syrk_t*        flash_syrk_cntl_ip;
fla_syrk_t*        flash_syrk_cntl_op;
fla_syrk_t*        flash_syrk_cntl_sq;
fla_blocksize_t*   flash_syrk_bsize;

void FLASH_Syrk_cntl_init()
{
    // Set syrk blocksize based on hierarchical storage.
    flash_syrk_bsize      = FLA_Blocksize_create( 1, 1, 1, 1 );

    // Create a control tree that assumes A is a b x b block.
    flash_syrk_cntl_blas  = FLA_Cntl_syrk_obj_create( FLA_EXECUTE_DREF,
                                                      FLA_UNBLOCKED_EXTERN,
                                                      NULL,
                                                      NULL,
                                                      NULL );

    // Create a control tree that assumes A * A' forms an inner panel product.
    flash_syrk_cntl_ip    = FLA_Cntl_syrk_obj_create( FLA_EXECUTE_ONLY,
                                                      FLA_BLOCKED_VARIANT5,
                                                      flash_syrk_bsize,
                                                      flash_syrk_cntl_blas,
                                                      NULL );

    // Create a control tree that assumes A * A' forms an outer panel product.
    flash_syrk_cntl_op    = FLA_Cntl_syrk_obj_create( FLA_EXECUTE_ONLY,
                                                      FLA_BLOCKED_VARIANT2,
                                                      flash_syrk_bsize,
                                                      flash_syrk_cntl_blas,
                                                      flash_gemm_cntl_pb_bb );

    // Create a control tree that assumes A is large.
    flash_syrk_cntl_sq    = FLA_Cntl_syrk_obj_create( FLA_EXECUTE_ONLY,
                                                      FLA_BLOCKED_VARIANT5,
                                                      flash_syrk_bsize,
                                                      flash_syrk_cntl_op,
                                                      NULL );
}
\end{verbatim}
}
\end{minipage}
\\
% --------------------------------------
 & \\ \hline
% --------------------------------------
{\sc trsm} with hierarchical storage & {\sc chol} with hierarchical storage \\ \hline
% --------------------------------------
\begin{minipage}[t]{3in}
{\tiny
\begin{verbatim}
#include "FLAME.h"

extern fla_gemm_t* flash_gemm_cntl_op_bp;

fla_trsm_t*        flash_trsm_cntl_blas;
fla_trsm_t*        flash_trsm_cntl_bp;
fla_trsm_t*        flash_trsm_cntl_mp;
fla_trsm_t*        flash_trsm_cntl_mm;
fla_blocksize_t*   flash_trsm_bsize;

void FLASH_Trsm_cntl_init()
{
    // Set trsm blocksize based on hierarchical storage.
    flash_trsm_bsize      = FLA_Blocksize_create( 1, 1, 1, 1 );

    // Create a control tree that assumes A and B are b x b blocks.
    flash_trsm_cntl_blas  = FLA_Cntl_trsm_obj_create( FLA_EXECUTE_DREF,
                                                      FLA_UNBLOCKED_EXTERN,
                                                      NULL,
                                                      NULL,
                                                      NULL );

    // Create a control tree that assumes A is a block and B is a panel.
    flash_trsm_cntl_bp    = FLA_Cntl_trsm_obj_create( FLA_EXECUTE_ONLY,
                                                      FLA_BLOCKED_VARIANT3,
                                                      flash_trsm_bsize,
                                                      flash_trsm_cntl_blas,
                                                      NULL );

    // Create a control tree that assumes A is large and B is a panel.
    flash_trsm_cntl_mp    = FLA_Cntl_trsm_obj_create( FLA_EXECUTE_ONLY,
                                                      FLA_BLOCKED_VARIANT2,
                                                      flash_trsm_bsize,
                                                      flash_trsm_cntl_blas,
                                                      flash_gemm_cntl_op_bp );

    // Create a control tree that assumes A and B are both large.
    flash_trsm_cntl_mm    = FLA_Cntl_trsm_obj_create( FLA_EXECUTE_ONLY,
                                                      FLA_BLOCKED_VARIANT3,
                                                      flash_trsm_bsize,
                                                      flash_trsm_cntl_mp,
                                                      NULL );
}
\end{verbatim}
}
\end{minipage}
&
\begin{minipage}[t]{3in}
{\tt \tiny
\begin{verbatim}
#include "FLAME.h"

extern fla_herk_t* flash_herk_cntl_op;
extern fla_syrk_t* flash_syrk_cntl_op;
extern fla_trsm_t* flash_trsm_cntl_bp;

fla_chol_t*        flash_chol_cntl_lapack;
fla_chol_t*        flash_chol_cntl;
fla_blocksize_t*   flash_chol_bsize;

void FLASH_Chol_cntl_init()
{
    // Set blocksize based on hierarchical storage.
    flash_chol_bsize       = FLA_Blocksize_create( 1, 1, 1, 1 );

    // Create a control tree that assumes A is a b x b block.
    flash_chol_cntl_lapack = FLA_Cntl_chol_obj_create( FLA_EXECUTE_DREF,
                                                       FLA_UNBLOCKED_EXTERN,
                                                       NULL,
                                                       NULL,
                                                       NULL,
                                                       NULL,
                                                       NULL,
                                                       NULL );

    // Create a control tree that assumes A is large.
    flash_chol_cntl        = FLA_Cntl_chol_obj_create( FLA_EXECUTE_ONLY,
                                                       FLA_BLOCKED_VARIANT3,
                                                       flash_chol_bsize,
                                                       flash_chol_cntl_lapack,
                                                       flash_syrk_cntl_op,
                                                       flash_herk_cntl_op,
                                                       flash_trsm_cntl_bp,
                                                       NULL );
}
\end{verbatim}
}
\end{minipage}
\\
% --------------------------------------
 & \\ \hline
% --------------------------------------
\end{tabular}
\end{center}
\caption{
Control tree initialization subroutines for various operations.
}
\label{fig:cntl-init}
\end{figure}





\subsection{Operation front-ends}

Once the library has been initialized, the default set of control trees
are ready to use.
Figure \ref{fig:cntl-front-ends} shows examples of some front-end
routines found in \libflamens.
This illustrates how control trees are used at the highest level.

\begin{figure}[t]
\begin{center}
\begin{tabular}{|c|c|}
\hline
% --------------------------------------
{\sc chol} front-end for conventional storage & {\sc chol} front-end for hierarchical storage \\ \hline
% --------------------------------------
\begin{minipage}[t]{3in}
{\tiny
\begin{verbatim}
#include "FLAME.h"

extern fla_chol_t* fla_chol_cntl;
extern fla_chol_t* fla_chol_cntl2;

FLA_Error FLA_Chol( FLA_Uplo uplo, FLA_Obj A )
{
  FLA_Error r_val;

  // Check parameters.
  if ( FLA_Check_error_level() >= FLA_MIN_ERROR_CHECKING )
    FLA_Chol_check( uplo, A );

  // Invoke FLA_Chol_internal() with the a control tree for sequential
  // execution with conventional storage.
  r_val = FLA_Chol_internal( uplo, A, fla_chol_cntl2 );

  return r_val;
}
\end{verbatim}
}
\end{minipage}
&
\begin{minipage}[t]{3in}
{\tt \tiny
\begin{verbatim}
#include "FLAME.h"

#ifdef FLA_ENABLE_SUPERMATRIX
extern fla_chol_t* flash_sm_chol_cntl;
#endif
extern fla_chol_t* flash_chol_cntl;

FLA_Error FLASH_Chol( FLA_Uplo uplo, FLA_Obj A )
{
  FLA_Error r_val;

  // Check parameters.
  if ( FLA_Check_error_level() >= FLA_MIN_ERROR_CHECKING )
    FLA_Chol_check( uplo, A );

  // Begin a parallel region.
  FLASH_Queue_begin();

  // Enqueue tasks via a SuperMatrix-aware control tree.
  r_val = FLA_Chol_internal( uplo, A, flash_sm_chol_cntl );

  // End the parallel region.
  FLASH_Queue_end();

  return r_val;
}
\end{verbatim}
}
\end{minipage}
\\
% --------------------------------------
 & \\ \hline
% --------------------------------------
\end{tabular}
\end{center}
\caption{
Front-end routines for Cholesky factorization with conventional storage (left)
and hierarchical storage (right).
Notice that {\tt FLASH\_Chol()} is capable of invoking both sequential and
parallel (SuperMatrix) execution types.
}
\label{fig:cntl-front-ends}
\end{figure}





\subsection{Internal back-ends}

\index{developer APIs!internal back-ends}

The \libflame front-ends and algorithmic variant implementations both
directly invoke internal back-end functions.
It is here that the control tree is decoded and used to determine
how execution will proceed with respect to variant and execution
type.
Figure \ref{fig:cntl-internal-back-ends} shows the internal routines
for Cholesky factorization.

\begin{figure}[t]
\begin{center}
\begin{tabular}{|c|c|}
\hline
% --------------------------------------
{\sc chol} internal back-end function  & {\sc chol} back-end subroutine for lower triangular case \\ \hline
% --------------------------------------
\begin{minipage}[t]{3in}
{\tiny
\begin{verbatim}
#include "FLAME.h"

FLA_Error FLA_Chol_internal( FLA_Uplo uplo, FLA_Obj A, fla_chol_t* cntl )
{
    FLA_Error r_val = FLA_SUCCESS;
    
#ifdef FLA_ENABLE_INTERNAL_ERROR_CHECKING
    FLA_Chol_internal_check( uplo, A, cntl );
#endif

    if      ( FLA_Cntl_matrix_type( cntl ) == FLA_HIER &&
              FLA_Obj_elemtype( A ) == FLA_MATRIX &&
              FLA_Cntl_variant( cntl ) == FLA_SUBPROBLEM )
    {
        // Recurse
        r_val = FLA_Chol_internal( uplo,
                                   *FLASH_OBJ_PTR_AT( A ),
                                   flash_chol_cntl );
    }
    else if ( FLA_Cntl_matrix_type( cntl ) == FLA_HIER &&
              FLA_Obj_elemtype( A ) == FLA_SCALAR &&
              FLASH_Queue_get_enabled( ) )
    {
        // Enqueue
        ENQUEUE_FLASH_Chol( uplo, A, cntl );
    }
    else
    {
        if ( FLA_Cntl_matrix_type( cntl ) == FLA_HIER &&
             FLA_Obj_elemtype( A ) == FLA_SCALAR &&
             !FLASH_Queue_get_enabled( ) )
        {
            // Execute leaf
            cntl = flash_chol_cntl_lapack;
        }
        // Parameter combinations
        if      ( uplo == FLA_LOWER_TRIANGULAR )
        {
            r_val = FLA_Chol_l( A, cntl );
        }
        else if ( uplo == FLA_UPPER_TRIANGULAR )
        {
            r_val = FLA_Chol_u( A, cntl );
        }
    }

    return r_val;
}
\end{verbatim}
}
\end{minipage}
&
\begin{minipage}[t]{3in}
{\tt \tiny
\begin{verbatim}
#include "FLAME.h"

FLA_Error FLA_Chol_l( FLA_Obj A, fla_chol_t* cntl )
{
    FLA_Error r_val = FLA_SUCCESS;

    if      ( FLA_Cntl_variant( cntl ) == FLA_SUBPROBLEM )
    {
        r_val = FLA_Chol_unb_external( FLA_LOWER_TRIANGULAR, A, cntl );
    }
#ifdef FLA_ENABLE_NON_CRITICAL_CODE
    else if ( FLA_Cntl_variant( cntl ) == FLA_BLOCKED_VARIANT1 )
    {
        r_val = FLA_Chol_l_blk_var1( A, cntl );
    }
    else if ( FLA_Cntl_variant( cntl ) == FLA_BLOCKED_VARIANT2 )
    {
        r_val = FLA_Chol_l_blk_var2( A, cntl );
    }
#endif
    else if ( FLA_Cntl_variant( cntl ) == FLA_BLOCKED_VARIANT3 )
    {
        r_val = FLA_Chol_l_blk_var3( A, cntl );
    }

    return r_val;
}
\end{verbatim}
}
\end{minipage}
\\
% --------------------------------------
 & \\ \hline
% --------------------------------------
\end{tabular}
\end{center}
\caption{
Internal back-end function for Cholesky factorization (left) and helper
subroutine to handle the lower triangular case (right).
}
\label{fig:cntl-internal-back-ends}
\end{figure}


Interfaces for the supported operation back-ends follow.



\subsubsection{Level-3 BLAS operations}

% --- FLA_Gemm_internal() ------------------------------------------------------

\begin{flaspec}
\begin{verbatim}
void FLA_Gemm_internal( FLA_Trans transa, FLA_Trans transb, FLA_Obj alpha,
                        FLA_Obj A, FLA_Obj B, FLA_Obj beta, FLA_Obj C, 
                        fla_gemm_t* cntl );
\end{verbatim}
\purpose{
Perform a {\sc gemm} operation on $ A $, $ B $, and $ C $ according to the
parameters specified by control tree node \cntlns.
}
\begin{checks}
\checkitem
\cntl must not be \fnullns.
%\itemvsp
%\checkitem
% 
\end{checks}
\moreinfo{
This function's interface is similar to that of \flagemmns.
Please see the description for \flagemm for further details.
}
\end{flaspec}

% --- FLA_Hemm_internal() ------------------------------------------------------

\begin{flaspec}
\begin{verbatim}
void FLA_Hemm_internal( FLA_Side side, FLA_Uplo uplo, FLA_Obj alpha,
                        FLA_Obj A, FLA_Obj B, FLA_Obj beta, FLA_Obj C, 
                        fla_hemm_t* cntl );
\end{verbatim}
\purpose{
Perform a {\sc hemm} operation on $ A $, $ B $, and $ C $ according to the
parameters specified by control tree node \cntlns.
}
\begin{checks}
\checkitem
\cntl must not be \fnullns.
%\itemvsp
%\checkitem
% 
\end{checks}
\moreinfo{
This function's interface is similar to that of \flahemmns.
Please see the description for \flahemm for further details.
}
\end{flaspec}

% --- FLA_Herk_internal() ------------------------------------------------------

\begin{flaspec}
\begin{verbatim}
void FLA_Herk_internal( FLA_Uplo uplo, FLA_Trans trans, FLA_Obj alpha,
                        FLA_Obj A, FLA_Obj beta, FLA_Obj C, 
                        fla_herk_t* cntl );
\end{verbatim}
\purpose{
Perform a {\sc herk} operation on $ A $ and $ C $ according to the
parameters specified by control tree node \cntlns.
}
\begin{checks}
\checkitem
\cntl must not be \fnullns.
%\itemvsp
%\checkitem
% 
\end{checks}
\moreinfo{
This function's interface is similar to that of \flaherkns.
Please see the description for \flaherk for further details.
}
\end{flaspec}

% --- FLA_Her2k_internal() -----------------------------------------------------

\begin{flaspec}
\begin{verbatim}
void FLA_Her2k_internal( FLA_Uplo uplo, FLA_Trans trans, FLA_Obj alpha,
                         FLA_Obj A, FLA_Obj B, FLA_Obj beta, FLA_Obj C, 
                         fla_her2k_t* cntl );
\end{verbatim}
\purpose{
Perform a {\sc her2k} operation on $ A $, $ B $, and $ C $ according to the
parameters specified by control tree node \cntlns.
}
\begin{checks}
\checkitem
\cntl must not be \fnullns.
%\itemvsp
%\checkitem
% 
\end{checks}
\moreinfo{
This function's interface is similar to that of \flahertkns.
Please see the description for \flahertk for further details.
}
\end{flaspec}

% --- FLA_Symm_internal() ------------------------------------------------------

\begin{flaspec}
\begin{verbatim}
void FLA_Symm_internal( FLA_Side side, FLA_Uplo uplo, FLA_Obj alpha,
                        FLA_Obj A, FLA_Obj B, FLA_Obj beta, FLA_Obj C, 
                        fla_symm_t* cntl );
\end{verbatim}
\purpose{
Perform a {\sc symm} operation on $ A $, $ B $, and $ C $ according to the
parameters specified by control tree node \cntlns.
}
\begin{checks}
\checkitem
\cntl must not be \fnullns.
%\itemvsp
%\checkitem
% 
\end{checks}
\moreinfo{
This function's interface is similar to that of \flasymmns.
Please see the description for \flasymm for further details.
}
\end{flaspec}

% --- FLA_Syrk_internal() ------------------------------------------------------

\begin{flaspec}
\begin{verbatim}
void FLA_Syrk_internal( FLA_Uplo uplo, FLA_Trans trans, FLA_Obj alpha,
                        FLA_Obj A, FLA_Obj beta, FLA_Obj C, 
                        fla_syrk_t* cntl );
\end{verbatim}
\purpose{
Perform a {\sc syrk} operation on $ A $ and $ C $ according to the
parameters specified by control tree node \cntlns.
}
\begin{checks}
\checkitem
\cntl must not be \fnullns.
%\itemvsp
%\checkitem
% 
\end{checks}
\moreinfo{
This function's interface is similar to that of \flasyrkns.
Please see the description for \flasyrk for further details.
}
\end{flaspec}

% --- FLA_Syr2k_internal() -----------------------------------------------------

\begin{flaspec}
\begin{verbatim}
void FLA_Syr2k_internal( FLA_Uplo uplo, FLA_Trans trans, FLA_Obj alpha,
                         FLA_Obj A, FLA_Obj B, FLA_Obj beta, FLA_Obj C, 
                         fla_syr2k_t* cntl );
\end{verbatim}
\purpose{
Perform a {\sc syr2k} operation on $ A $, $ B $, and $ C $ according to the
parameters specified by control tree node \cntlns.
}
\begin{checks}
\checkitem
\cntl must not be \fnullns.
%\itemvsp
%\checkitem
% 
\end{checks}
\moreinfo{
This function's interface is similar to that of \flasyrtkns.
Please see the description for \flasyrtk for further details.
}
\end{flaspec}

% --- FLA_Trmm_internal() ------------------------------------------------------

\begin{flaspec}
\begin{verbatim}
void FLA_Trmm_internal( FLA_Side side, FLA_Uplo uplo, FLA_Trans trans,
                        FLA_Diag diag, FLA_Obj alpha, FLA_Obj A, FLA_Obj B,  
                        fla_trmm_t* cntl );
\end{verbatim}
\purpose{
Perform a {\sc trmm} operation on $ A $ and $ B $ according to the
parameters specified by control tree node \cntlns.
}
\begin{checks}
\checkitem
\cntl must not be \fnullns.
%\itemvsp
%\checkitem
% 
\end{checks}
\moreinfo{
This function's interface is similar to that of \flatrmmns.
Please see the description for \flatrmm for further details.
}
\end{flaspec}

% --- FLA_Trsm_internal() ------------------------------------------------------

\begin{flaspec}
\begin{verbatim}
void FLA_Trsm_internal( FLA_Side side, FLA_Uplo uplo, FLA_Trans trans, FLA_Diag diag,
                        FLA_Obj alpha, FLA_Obj A, FLA_Obj B,  
                        fla_trsm_t* cntl );
\end{verbatim}
\purpose{
Perform a {\sc trsm} operation on $ A $ and $ B $ according to the
parameters specified by control tree node \cntlns.
}
\begin{checks}
\checkitem
\cntl must not be \fnullns.
%\itemvsp
%\checkitem
% 
\end{checks}
\moreinfo{
This function's interface is similar to that of \flatrsmns.
Please see the description for \flatrsm for further details.
}
\end{flaspec}










\subsubsection{LAPACK operations}

% --- FLA_Chol_internal() ------------------------------------------------------

\begin{flaspec}
\begin{verbatim}
FLA_Error FLA_Chol_internal( FLA_Uplo uplo, FLA_Obj A, fla_chol_t* cntl );
\end{verbatim}
\purpose{
Perform a {\sc chol} operation on $ A $ according to the
parameters specified by control tree node \cntlns.
}
\begin{checks}
\checkitem
\cntl must not be \fnullns.
%\itemvsp
%\checkitem
% 
\end{checks}
\moreinfo{
This function's interface is similar to that of \flacholns.
Please see the description for \flachol for further details.
}
\end{flaspec}

% --- FLA_Trinv_internal() -----------------------------------------------------

\begin{flaspec}
\begin{verbatim}
FLA_Error FLA_Trinv_internal( FLA_Uplo uplo, FLA_Diag diag, FLA_Obj A,
                              fla_trinv_t* cntl );
\end{verbatim}
\purpose{
Perform a {\sc trinv} operation on $ A $ according to the
parameters specified by control tree node \cntlns.
}
\begin{checks}
\checkitem
\cntl must not be \fnullns.
%\itemvsp
%\checkitem
% 
\end{checks}
\moreinfo{
This function's interface is similar to that of \flatrinvns.
Please see the description for \flatrinv for further details.
}
\end{flaspec}

% --- FLA_Ttmm_internal() ------------------------------------------------------

\begin{flaspec}
\begin{verbatim}
void FLA_Ttmm_internal( FLA_Uplo uplo, FLA_Obj A, fla_ttmm_t* cntl );
\end{verbatim}
\purpose{
Perform a {\sc ttmm} operation on $ A $ according to the
parameters specified by control tree node \cntlns.
}
\begin{checks}
\checkitem
\cntl must not be \fnullns.
%\itemvsp
%\checkitem
% 
\end{checks}
\moreinfo{
This function's interface is similar to that of \flattmmns.
Please see the description for \flattmm for further details.
}
\end{flaspec}

% --- FLA_SPDinv_internal() ----------------------------------------------------

\begin{flaspec}
\begin{verbatim}
void FLA_SPDinv_internal( FLA_Uplo uplo, FLA_Obj A, fla_spdinv_t* cntl );
\end{verbatim}
\purpose{
Perform a {\sc spdinv} operation on $ A $ according to the
parameters specified by control tree node \cntlns.
}
\begin{checks}
\checkitem
\cntl must not be \fnullns.
%\itemvsp
%\checkitem
% 
\end{checks}
\moreinfo{
This function's interface is similar to that of \flaspdinvns.
Please see the description for \flaspdinv for further details.
}
\end{flaspec}

% --- FLA_Hess_internal() ------------------------------------------------------
%
%\begin{flaspec}
%\begin{verbatim}
%void FLA_Hess_internal( FLA_Obj A, FLA_Obj t, int ilo, int ihi,
%                        fla_hess_t* cntl );
%\end{verbatim}
%\purpose{
%Perform a {\sc hess} operation on $ A $ according to the
%parameters specified by control tree node \cntlns.
%}
%\begin{checks}
%\checkitem
%\cntl must not be \fnullns.
%%\itemvsp
%%\checkitem
%% 
%\end{checks}
%\moreinfo{
%This function's interface is similar to that of \flahessns.
%Please see the description for \flahess for further details.
%}
%\end{flaspec}

% --- FLA_LU_nopiv_internal() --------------------------------------------------

\begin{flaspec}
\begin{verbatim}
FLA_Error FLA_LU_nopiv_internal( FLA_Obj A, fla_lu_t* cntl );
\end{verbatim}
\purpose{
Perform a {\sc lunopiv} operation on $ A $ according to the
parameters specified by control tree node \cntlns.
}
\begin{checks}
\checkitem
\cntl must not be \fnullns.
%\itemvsp
%\checkitem
% 
\end{checks}
\moreinfo{
This function's interface is similar to that of \flalunopivns.
Please see the description for \flalunopiv for further details.
}
\end{flaspec}

% --- FLA_LU_piv_internal() ----------------------------------------------------

\begin{flaspec}
\begin{verbatim}
FLA_Error FLA_LU_piv_internal( FLA_Obj A, FLA_Obj p, fla_lu_t* cntl );
\end{verbatim}
\purpose{
Perform a {\sc lupiv} operation on $ A $ according to the
parameters specified by control tree node \cntlns.
}
\begin{checks}
\checkitem
\cntl must not be \fnullns.
%\itemvsp
%\checkitem
% 
\end{checks}
\moreinfo{
This function's interface is similar to that of \flalunopivns.
Please see the description for \flalunopiv for further details.
}
\end{flaspec}

% --- FLA_QR_UT_internal() -----------------------------------------------------

\begin{flaspec}
\begin{verbatim}
void FLA_QR_UT_internal( FLA_Obj A, FLA_Obj T, fla_qrut_t* cntl );
\end{verbatim}
\purpose{
Perform a {\sc qrut} operation on $ A $ according to the
parameters specified by control tree node \cntlns.
}
\begin{checks}
\checkitem
\cntl must not be \fnullns.
%\itemvsp
%\checkitem
% 
\end{checks}
\moreinfo{
This function's interface is similar to that of \flaqrutns.
Please see the description for \flaqrut for further details.
}
\end{flaspec}

% --- FLA_LQ_UT_internal() -----------------------------------------------------

\begin{flaspec}
\begin{verbatim}
void FLA_LQ_UT_internal( FLA_Obj A, FLA_Obj T, fla_lq_t* cntl );
\end{verbatim}
\purpose{
Perform a {\sc lqut} operation on $ A $ according to the
parameters specified by control tree node \cntlns.
}
\begin{checks}
\checkitem
\cntl must not be \fnullns.
%\itemvsp
%\checkitem
% 
\end{checks}
\moreinfo{
This function's interface is similar to that of \flalqutns.
Please see the description for \flalqut for further details.
}
\end{flaspec}

% --- FLA_Sylv_internal() ------------------------------------------------------

\begin{flaspec}
\begin{verbatim}
void FLA_Sylv_internal( FLA_Trans transa, FLA_Trans transb, FLA_Obj isgn,
                        FLA_Obj A, FLA_Obj B, FLA_Obj C, FLA_Obj scale,
                        fla_sylv_t* cntl );
\end{verbatim}
\purpose{
Perform a {\sc sylv} operation on $ A $, $ B $, and $ C $ according to the
parameters specified by control tree node \cntlns.
}
\begin{checks}
\checkitem
\cntl must not be \fnullns.
%\itemvsp
%\checkitem
% 
\end{checks}
\moreinfo{
This function's interface is similar to that of \flasylvns.
Please see the description for \flasylv for further details.
}
\end{flaspec}













\subsection{Algorithmic variants}

The algorithmic variants in \libflame are coded differently than earlier
incarnations of FLAME/C. 
Figure \ref{fig:noncntl-cntl-chol-code} illustrates these differences
for the blocked FLAME/C implementation of algorithmic variant 3 of
Cholesky factorization.
The top-left code example shows what the code might look like when
the programmer used unblocked FLAME variants to perform subproblems.
The top-right code is similar, except that it uses blocked variants.
Note that this code uses the same algorithmic blocksize for it subproblems
as it does for matrix partitioning within the Cholesky algorithm.
The bottom-left code again shows how the FLAME group used to present
its codes during lectures and presentations, where the routines
{\tt FLA\_Chol()}, {\tt FLA\_Trsm()}, and {\tt FLA\_Syrk()} were
wrappers to external implementations of those operations.\footnote{
Notice that this class of function names is now reserved for the user-level
front-end interfaces documented in Sections \ref{sec:blas3-front-ends} and
\ref{sec:lapack-front-ends}, and the corresponding external routines are now
explicitly named as {\tt FLA\_Chol\_unb\_external()},
{\tt FLA\_Trsm\_external()}, and {\tt FLA\_Syrk\_external()}.
}
Finally, the bottom-right code shows how algorithms are now coded within
\libflamens, using control trees.
The most notable difference between this code and the others is
that the subproblems invoke internal back-end routines for the
operation in question rather than statically specifying an unblocked,
blocked, or external implementation.
%All of the details of the implementation for a particular subproblem
%are specified within its corresponding control tree node.

%A partial but representative list of algorithmic variants is given in Section
%\ref{sec:algorithmic-variants}.

\begin{figure}[tbp]
\begin{center}
\begin{tabular}{|c|c|}
\hline
% --------------------------------------
calling unblocked FLAME/C variants & calling blocked FLAME/C variants \\ \hline
% --------------------------------------
\begin{minipage}[t]{3in}
{\tiny
\begin{verbatim}
FLA_Error FLA_Chol_l_blk_var3( FLA_Obj A, dim_t nb_alg )
{
  FLA_Obj ATL,   ATR,      A00, A01, A02,
          ABL,   ABR,      A10, A11, A12,
                           A20, A21, A22;
  dim_t b;
  int value;

  FLA_Part_2x2( A,    &ATL, &ATR,
                      &ABL, &ABR,     0, 0, FLA_TL );

  while ( FLA_Obj_length( ATL ) < FLA_Obj_length( A ) )
  {
    b = min( FLA_Obj_length( ABR ), nb_alg );

    FLA_Repart_2x2_to_3x3( ATL, /**/ ATR,       &A00, /**/ &A01, &A02,
                        /* ************* */   /* ******************** */
                                                &A10, /**/ &A11, &A12,
                           ABL, /**/ ABR,       &A20, /**/ &A21, &A22,
                           b, b, FLA_BR );
    /*-------------------------------------------------------------------*/

    value = FLA_Chol_l_unb_var2( A11 );

    if ( value != FLA_SUCCESS )
      return ( FLA_Obj_length( A00 ) + value );

    FLA_Trsm_rlt_unb_var3( FLA_NONUNIT_DIAG,
                           FLA_ONE, A11, A21 );

    FLA_Syrk_ln_unb_var2( FLA_MINUS_ONE, A21, FLA_ONE, A22 );

    /*-------------------------------------------------------------------*/
    FLA_Cont_with_3x3_to_2x2( &ATL, /**/ &ATR,       A00, A01, /**/ A02,
                                                     A10, A11, /**/ A12,
                            /* ************** */  /* ****************** */
                              &ABL, /**/ &ABR,       A20, A21, /**/ A22,
                              FLA_TL );
  }

  return value;
}
\end{verbatim}
}
\end{minipage}
&
\begin{minipage}[t]{3in}
{\tt \tiny
\begin{verbatim}
FLA_Error FLA_Chol_l_blk_var3( FLA_Obj A, dim_t nb_alg )
{
  FLA_Obj ATL,   ATR,      A00, A01, A02,
          ABL,   ABR,      A10, A11, A12,
                           A20, A21, A22;
  dim_t b;
  int value;

  FLA_Part_2x2( A,    &ATL, &ATR,
                      &ABL, &ABR,     0, 0, FLA_TL );

  while ( FLA_Obj_length( ATL ) < FLA_Obj_length( A ) )
  {
    b = min( FLA_Obj_length( ABR ), nb_alg );

    FLA_Repart_2x2_to_3x3( ATL, /**/ ATR,       &A00, /**/ &A01, &A02,
                        /* ************* */   /* ******************** */
                                                &A10, /**/ &A11, &A12,
                           ABL, /**/ ABR,       &A20, /**/ &A21, &A22,
                           b, b, FLA_BR );
    /*-------------------------------------------------------------------*/

    value = FLA_Chol_l_blk_var2( A11, nb_alg );

    if ( value != FLA_SUCCESS )
      return ( FLA_Obj_length( A00 ) + value );

    FLA_Trsm_rlt_blk_var3( FLA_NONUNIT_DIAG,
                           FLA_ONE, A11, A21, nb_alg );

    FLA_Syrk_ln_blk_var2( FLA_MINUS_ONE, A21, FLA_ONE, A22, nb_alg );

    /*-------------------------------------------------------------------*/
    FLA_Cont_with_3x3_to_2x2( &ATL, /**/ &ATR,       A00, A01, /**/ A02,
                                                     A10, A11, /**/ A12,
                            /* ************** */  /* ****************** */
                              &ABL, /**/ &ABR,       A20, A21, /**/ A22,
                              FLA_TL );
  }

  return value;
}
\end{verbatim}
}
\end{minipage}
\\
% --------------------------------------
 & \\ \hline
% --------------------------------------
calling unblocked external implementations & using control trees \\ \hline
% --------------------------------------
\begin{minipage}[t]{3in}
{\tiny
\begin{verbatim}
FLA_Error FLA_Chol_l_blk_var3( FLA_Obj A, dim_t nb_alg )
{
  FLA_Obj ATL,   ATR,      A00, A01, A02,
          ABL,   ABR,      A10, A11, A12,
                           A20, A21, A22;
  dim_t b;
  int value;

  FLA_Part_2x2( A,    &ATL, &ATR,
                      &ABL, &ABR,     0, 0, FLA_TL );

  while ( FLA_Obj_length( ATL ) < FLA_Obj_length( A ) )
  {
    b = min( FLA_Obj_length( ABR ), nb_alg );

    FLA_Repart_2x2_to_3x3( ATL, /**/ ATR,       &A00, /**/ &A01, &A02,
                        /* ************* */   /* ******************** */
                                                &A10, /**/ &A11, &A12,
                           ABL, /**/ ABR,       &A20, /**/ &A21, &A22,
                           b, b, FLA_BR );
    /*-------------------------------------------------------------------*/

    value = FLA_Chol( FLA_LOWER_TRIANGULAR, A11 );

    if ( value != FLA_SUCCESS )
      return ( FLA_Obj_length( A00 ) + value );

    FLA_Trsm( FLA_RIGHT, FLA_LOWER_TRIANGULAR,
              FLA_TRANSPOSE, FLA_NONUNIT_DIAG,
              FLA_ONE, A11, A21 );

    FLA_Syrk( FLA_LOWER_TRIANGULAR, FLA_NO_TRANSPOSE,
              FLA_MINUS_ONE, A21, FLA_ONE, A22 );

    /*-------------------------------------------------------------------*/
    FLA_Cont_with_3x3_to_2x2( &ATL, /**/ &ATR,       A00, A01, /**/ A02,
                                                     A10, A11, /**/ A12,
                            /* ************** */  /* ****************** */
                              &ABL, /**/ &ABR,       A20, A21, /**/ A22,
                              FLA_TL );
  }

  return value;
}
\end{verbatim}
}
\end{minipage}
&
\begin{minipage}[t]{3in}
{\tt \tiny
\begin{verbatim}
FLA_Error FLA_Chol_l_blk_var3( FLA_Obj A, fla_chol_t* cntl )
{
  FLA_Obj ATL,   ATR,      A00, A01, A02,
          ABL,   ABR,      A10, A11, A12,
                           A20, A21, A22;
  dim_t b;
  int value;

  FLA_Part_2x2( A,    &ATL, &ATR,
                      &ABL, &ABR,     0, 0, FLA_TL );

  while ( FLA_Obj_length( ATL ) < FLA_Obj_length( A ) )
  {
    b = FLA_Determine_blocksize( ABR, FLA_BR, FLA_Cntl_blocksize( cntl ) );

    FLA_Repart_2x2_to_3x3( ATL, /**/ ATR,       &A00, /**/ &A01, &A02,
                        /* ************* */   /* ******************** */
                                                &A10, /**/ &A11, &A12,
                           ABL, /**/ ABR,       &A20, /**/ &A21, &A22,
                           b, b, FLA_BR );
    /*-------------------------------------------------------------------*/

    value = FLA_Chol_internal( FLA_LOWER_TRIANGULAR, A11,
                               FLA_Cntl_sub_chol( cntl ) );

    if ( value != FLA_SUCCESS )
      return ( FLA_Obj_length( A00 ) + value );

    FLA_Trsm_internal( FLA_RIGHT, FLA_LOWER_TRIANGULAR,
                       FLA_TRANSPOSE, FLA_NONUNIT_DIAG,
                       FLA_ONE, A11, A21,
                       FLA_Cntl_sub_trsm( cntl ) );

    FLA_Syrk_internal( FLA_LOWER_TRIANGULAR, FLA_NO_TRANSPOSE,
                       FLA_MINUS_ONE, A21, FLA_ONE, A22,
                       FLA_Cntl_sub_syrk( cntl ) );

    /*-------------------------------------------------------------------*/
    FLA_Cont_with_3x3_to_2x2( &ATL, /**/ &ATR,       A00, A01, /**/ A02,
                                                     A10, A11, /**/ A12,
                            /* ************** */  /* ****************** */
                              &ABL, /**/ &ABR,       A20, A21, /**/ A22,
                              FLA_TL );
  }

  return value;
}
\end{verbatim}
}
\end{minipage}
\\
% --------------------------------------
 & \\ \hline
% --------------------------------------
\end{tabular}
\end{center}
\caption{
A Cholesky factorization algorithm (variant 3) implemented using various
deprecated methods for statically encoding the subproblem implementation
(top-left, top-right, bottom-left) and the more modern style using control
trees (bottom-right).
}
\label{fig:noncntl-cntl-chol-code}
\end{figure}






\section{Parameter and error checking}

\index{developer APIs!parameter and error checking}





\subsection{Linear algebra parameters}

% --- FLA_Check_valid_side() ---------------------------------------------------

\begin{flaspec}
\begin{verbatim}
FLA_Error FLA_Check_valid_side( FLA_Side side );
\end{verbatim}
\purpose{
Confirm that \side is one of the following values defined for the
\flaside type:
\flaleftns, \flarightns, \flatopns, \flabottomns.
}
\rvalue{
\flasuccess if \side is valid;
\flainvalidside otherwise.
}
%\begin{params}
%\parameter{\flaside}{side}{A constant value to check against.}
%\end{params}
\end{flaspec}

% --- FLA_Check_valid_uplo() ---------------------------------------------------

\begin{flaspec}
\begin{verbatim}
FLA_Error FLA_Check_valid_uplo( FLA_Uplo uplo );
\end{verbatim}
\purpose{
Confirm that \uplo is one of the following values defined for the
\flauplo type:
\flalowertriangularns, \flauppertriangularns.
}
\rvalue{
\flasuccess if \uplo is valid;
\flainvaliduplo otherwise.
}
%\begin{params}
%\parameter{\flauplo}{uplo}{A constant value to check against.}
%\end{params}
\end{flaspec}

% --- FLA_Check_valid_trans() --------------------------------------------------

\begin{flaspec}
\begin{verbatim}
FLA_Error FLA_Check_valid_trans( FLA_Trans trans );
\end{verbatim}
\purpose{
Confirm that \trans is one of the following values defined for the
\flatrans type:
\flanotranspose, \flatransposens, \flaconjtransposens, \flaconjnotransposens.
}
\rvalue{
\flasuccess if \trans is valid;
\flainvalidtrans otherwise.
}
%\begin{params}
%\parameter{\flatrans}{trans}{A constant value to check against.}
%\end{params}
\end{flaspec}

% --- FLA_Check_valid_real_trans() ---------------------------------------------

\begin{flaspec}
\begin{verbatim}
FLA_Error FLA_Check_valid_real_trans( FLA_Trans trans );
\end{verbatim}
\purpose{
Confirm that \trans is either \flanotranspose or \flatransposens.
}
\notes{
This check is typically used with \trans arguments that are expected to be
applied to real matrices.
}
\rvalue{
\flasuccess if the \trans argument is either \flanotranspose or
\flatransposens.
\flainvalidrealtrans otherwise.
}
\end{flaspec}

% --- FLA_Check_valid_complex_trans() ------------------------------------------

\begin{flaspec}
\begin{verbatim}
FLA_Error FLA_Check_valid_complex_trans( FLA_Trans trans );
\end{verbatim}
\purpose{
Confirm that \trans is either \flanotranspose or \flaconjtransposens.
}
\notes{
This check is typically used with \trans arguments that are expected to be
applied to Hermitian matrices.
}
\rvalue{
\flasuccess if the \trans argument is either \flanotranspose or
\flaconjtransposens.
\flainvalidcomplextrans otherwise.
}
\end{flaspec}

% --- FLA_Check_valid_blas_trans() ---------------------------------------------

\begin{flaspec}
\begin{verbatim}
FLA_Error FLA_Check_valid_blas_trans( FLA_Trans trans );
\end{verbatim}
\purpose{
Confirm that \trans is either \flanotransposens, \flatransposens, or
\flaconjtransposens.
}
\notes{
This check is typically used with \trans arguments that are expected to be
applied to general matrices.
Valid values correspond to those supported by the BLAS interface, and
thus \flaconjnotranspose is not allowed.
}
\rvalue{
\flasuccess if the \trans argument is either \flanotransposens,
\flatransposens, or \flaconjtransposens.
\flainvalidblastrans otherwise.
}
\end{flaspec}

% --- FLA_Check_valid_diag() ---------------------------------------------------

\begin{flaspec}
\begin{verbatim}
FLA_Error FLA_Check_valid_diag( FLA_Diag diag );
\end{verbatim}
\purpose{
Confirm that \diag is one of the following values defined for the
\fladiag type:
\flanonunitdiagns, \flaunitdiagns, \flazerodiagns.
}
\rvalue{
\flasuccess if \diag is valid;
\flainvaliddiag otherwise.
}
%\begin{params}
%\parameter{\fladiag}{diag}{A constant value to check against.}
%\end{params}
\end{flaspec}

% --- FLA_Check_valid_conj() ---------------------------------------------------

\begin{flaspec}
\begin{verbatim}
FLA_Error FLA_Check_valid_conj( FLA_Conj conj );
\end{verbatim}
\purpose{
Confirm that \conj is one of the following values defined for the
\flaconj type:
\flanoconjugatens, \flaconjugatens.
}
\rvalue{
\flasuccess if \conj is valid;
\flainvalidconj otherwise.
}
%\begin{params}
%\parameter{\flaconj}{conj}{A constant value to check against.}
%\end{params}
\end{flaspec}

% --- FLA_Check_valid_direct() -------------------------------------------------

\begin{flaspec}
\begin{verbatim}
FLA_Error FLA_Check_valid_direct( FLA_Direct direct );
\end{verbatim}
\purpose{
Confirm that \direct is one of the following values defined for the
\fladirect type:
\flaforwardns, \flabackwardns.
}
\rvalue{
\flasuccess if \direct is valid;
\flainvaliddirect otherwise.
}
%\begin{params}
%\parameter{\fladirect}{direct}{A constant value to check against.}
%\end{params}
\end{flaspec}

% --- FLA_Check_valid_storev() -------------------------------------------------

\begin{flaspec}
\begin{verbatim}
FLA_Error FLA_Check_valid_storev( FLA_Store storev );
\end{verbatim}
\purpose{
Confirm that \storev is one of the following values defined for the
\flastore type:
\flacolumnwisens, \flarowwisens.
}
\rvalue{
\flasuccess if \storev is valid;
\flainvalidstorev otherwise.
}
%\begin{params}
%\parameter{\flastore}{storev}{A constant value to check against.}
%\end{params}
\end{flaspec}

% --- FLA_Check_valid_quadrant() -----------------------------------------------

\begin{flaspec}
\begin{verbatim}
FLA_Error FLA_Check_valid_quadrant( FLA_Quadrant quadrant );
\end{verbatim}
\purpose{
Confirm that \quadrant is one of the following values defined for the
\flaquadrant type:
\flatlns, \flatrns, \flablns, \flabrns.
}
\rvalue{
\flasuccess if \quadrant is valid;
\flainvalidquadrant otherwise.
}
\end{flaspec}




\subsection{Datatypes}

% --- FLA_Check_valid_datatype() -----------------------------------------------

\begin{flaspec}
\begin{verbatim}
FLA_Error FLA_Check_valid_datatype( FLA_Datatype datatype );
\end{verbatim}
\purpose{
Confirm that \datatype is one of the following values defined for the
\fladatatype type:
\flafloatns, \fladoublens, \flacomplexns, \fladoublecomplexns,
\flaintns, \flaconstantns.
}
\rvalue{
\flasuccess if \datatype is valid;
\flainvaliddatatype otherwise.
}
%\begin{params}
%\parameter{\fladatatype}{datatype}{A constant value to check against.}
%\end{params}
\end{flaspec}

% --- FLA_Check_valid_object_datatype() ----------------------------------------

\begin{flaspec}
\begin{verbatim}
FLA_Error FLA_Check_valid_object_datatype( FLA_Obj A );
\end{verbatim}
\purpose{
Confirm that the datatype of $ A $ is one of the following values
defined for the \fladatatype type:
\flaintns, \flafloatns, \fladoublens, \flacomplexns, \fladoublecomplexns,
\flaconstantns.
}
\rvalue{
\flasuccess if the datatype of $ A $ is valid;
\flainvaliddatatype otherwise.
}
%\begin{params}
%\parameter{\flaobj}{A}{An \flaobj to check.}
%\end{params}
\end{flaspec}

% --- FLA_Check_floating_datatype() --------------------------------------------

\begin{flaspec}
\begin{verbatim}
FLA_Error FLA_Check_floating_datatype( FLA_Datatype datatype );
\end{verbatim}
\purpose{
Confirm that \datatype refers to one of the following floating point
type values defined for \fladatatypens:
\flafloatns, \fladoublens, \flacomplexns, \fladoublecomplexns,
\flaconstantns.
}
\notes{
Though it is a distinct type, \flaconstant is polymorphic and thus may
be considered floating point for the purposes of this function.
}
\rvalue{
\flasuccess if \datatype is floating point;
\flainvalidfloatingdatatype otherwise.
}
%\begin{params}
%\parameter{\fladatatype}{datatype}{A constant value to check against.}
%\end{params}
\end{flaspec}

% --- FLA_Check_int_datatype() -------------------------------------------------

\begin{flaspec}
\begin{verbatim}
FLA_Error FLA_Check_int_datatype( FLA_Datatype datatype );
\end{verbatim}
\purpose{
Confirm that \datatype refers to one of the following integer
type values defined for \fladatatypens:
\flaintns, \flaconstantns.
}
\notes{
Though it is a distinct type, \flaconstant is polymorphic and thus may
be considered integer for the purposes of this function.
}
\rvalue{
\flasuccess if \datatype is integer;
\flainvalidintegerdatatype otherwise.
}
%\begin{params}
%\parameter{\fladatatype}{datatype}{A constant value to check against.}
%\end{params}
\end{flaspec}

% --- FLA_Check_real_datatype() ------------------------------------------------

\begin{flaspec}
\begin{verbatim}
FLA_Error FLA_Check_real_datatype( FLA_Datatype datatype );
\end{verbatim}
\purpose{
Confirm that \datatype refers to one of the following real numerical
type values defined for \fladatatypens:
\flafloatns, \fladoublens, \flaconstantns.
}
\notes{
Though it is a distinct type, \flaconstant is polymorphic and thus may
be considered real for the purposes of this function.
}
\rvalue{
\flasuccess if \datatype is real;
\flainvalidrealdatatype otherwise.
}
%\begin{params}
%\parameter{\fladatatype}{datatype}{A constant value to check against.}
%\end{params}
\end{flaspec}

% --- FLA_Check_complex_datatype() ---------------------------------------------

\begin{flaspec}
\begin{verbatim}
FLA_Error FLA_Check_complex_datatype( FLA_Datatype datatype );
\end{verbatim}
\purpose{
Confirm that \datatype refers to one of the following complex numerical
type values defined for \fladatatypens:
\flacomplexns, \fladoublecomplexns, \flaconstantns.
}
\notes{
Though it is a distinct type, \flaconstant is polymorphic and thus may
be considered complex for the purposes of this function.
}
\rvalue{
\flasuccess if \datatype is complex;
\flainvalidcomplexdatatype otherwise.
}
%\begin{params}
%\parameter{\fladatatype}{datatype}{A constant value to check against.}
%\end{params}
\end{flaspec}

% --- FLA_Check_nonconstant_datatype() -----------------------------------------

\begin{flaspec}
\begin{verbatim}
FLA_Error FLA_Check_nonconstant_datatype( FLA_Datatype datatype );
\end{verbatim}
\purpose{
Confirm that \datatype is one of the following non-constant values
defined for the \fladatatype type:
\flafloatns, \fladoublens, \flacomplexns, \fladoublecomplexns,
\flaintns.
}
\notes{
This function is similar to {\tt FLA\_Check\_valid\_datatype()},
except that it does not allow \flaconstantns.
}
\rvalue{
\flasuccess if \datatype specifies a non-constant datatype;
\flainvalidnonconstantdatatype otherwise.
}
\end{flaspec}

% --- FLA_Check_floating_object() ----------------------------------------------

\begin{flaspec}
\begin{verbatim}
FLA_Error FLA_Check_floating_object( FLA_Obj A );
\end{verbatim}
\purpose{
Confirm that the datatype of $ A $ is one of the following floating point
type values defined for \fladatatypens:
\flafloatns, \fladoublens, \flacomplexns, \fladoublecomplexns,
\flaconstantns.
}
\notes{
Though it is a distinct type, \flaconstant is polymorphic and thus may
be considered floating point for the purposes of this function.
}
\rvalue{
\flasuccess if the datatype of $ A $ is floating point;
\flaobjectnotfloatingpoint otherwise.
}
%\begin{params}
%\parameter{\flaobj}{A}{An \flaobj to check.}
%\end{params}
\end{flaspec}

% --- FLA_Check_int_object() ---------------------------------------------------

\begin{flaspec}
\begin{verbatim}
FLA_Error FLA_Check_int_object( FLA_Obj A );
\end{verbatim}
\purpose{
Confirm that the datatype of $ A $ is one of the following integer
type values defined for \fladatatypens:
\flaintns, \flaconstantns.
}
\notes{
Though it is a distinct type, \flaconstant is polymorphic and thus may
be considered integer for the purposes of this function.
}
\rvalue{
\flasuccess if the datatype of $ A $ is integer;
\flaobjectnotinteger otherwise.
}
%\begin{params}
%\parameter{\flaobj}{A}{An \flaobj to check.}
%\end{params}
\end{flaspec}

% --- FLA_Check_real_object() --------------------------------------------------

\begin{flaspec}
\begin{verbatim}
FLA_Error FLA_Check_real_object( FLA_Obj A );
\end{verbatim}
\purpose{
Confirm that the datatype of $ A $ is one of the following real numerical
type values defined for \fladatatypens:
\flafloatns, \fladoublens, \flaconstantns.
}
\notes{
Though it is a distinct type, \flaconstant is polymorphic and thus may
be considered real for the purposes of this function.
}
\rvalue{
\flasuccess if the datatype of $ A $ is real;
\flaobjectnotreal otherwise.
}
%\begin{params}
%\parameter{\flaobj}{A}{An \flaobj to check.}
%\end{params}
\end{flaspec}

% --- FLA_Check_complex_object() -----------------------------------------------

\begin{flaspec}
\begin{verbatim}
FLA_Error FLA_Check_complex_object( FLA_Obj A );
\end{verbatim}
\purpose{
Confirm that the datatype of $ A $ is one of the following complex numerical
type values defined for \fladatatypens:
\flacomplexns, \fladoublecomplexns, \flaconstantns.
}
\notes{
Though it is a distinct type, \flaconstant is polymorphic and thus may
be considered complex for the purposes of this function.
}
\rvalue{
\flasuccess if the datatype of $ A $ is complex;
\flaobjectnotcomplex otherwise.
}
%\begin{params}
%\parameter{\flaobj}{A}{An \flaobj to check.}
%\end{params}
\end{flaspec}

% --- FLA_Check_nonconstant_object() -------------------------------------------

\begin{flaspec}
\begin{verbatim}
FLA_Error FLA_Check_nonconstant_object( FLA_Obj A );
\end{verbatim}
\purpose{
Confirm that the datatype of $ A $ is one of the following non-constant values
defined for the \fladatatype type:
\flaintns, \flafloatns, \fladoublens, \flacomplexns, \fladoublecomplexns.
}
\rvalue{
\flasuccess if the datatype of $ A $ is a non-constant datatype;
\flaobjectnotnonconstant otherwise.
}
\end{flaspec}

% --- FLA_Check_identical_object_datatype() ------------------------------------

\begin{flaspec}
\begin{verbatim}
FLA_Error FLA_Check_identical_object_datatype( FLA_Obj A, FLA_Obj B );
\end{verbatim}
\purpose{
Confirm that $ A $ and $ B $ have identical datatypes.
}
\notes{
This function enforces literal equality between the datatype fields of
$ A $ and $ B $.
}
\rvalue{
\flasuccess if $ A $ and $ B $ have identical datatypes;
\flaobjectdatatypesnotequal otherwise.
}
\end{flaspec}

% --- FLA_Check_consistent_object_datatype() -----------------------------------

\begin{flaspec}
\begin{verbatim}
FLA_Error FLA_Check_consistent_object_datatype( FLA_Obj A, FLA_Obj B );
\end{verbatim}
\purpose{
Confirm that the datatype of $ A $ is consistent with the datatype of $ B $.
}
\notes{
This function is similar to {\tt FLA\_Check\_identical\_object\_datatype()},
except that it considers objects of datatype \flaconstant to be consistent
with all other datatypes.
}
\rvalue{
\flasuccess if the datatype of $ A $ is equal to the datatype of $ B $;
\flainconsistentdatatypes otherwise.
}
\end{flaspec}

% --- FLA_Check_consistent_datatype() ------------------------------------------

\begin{flaspec}
\begin{verbatim}
FLA_Error FLA_Check_consistent_datatype( FLA_Datatype datatype, FLA_Obj A );
\end{verbatim}
\purpose{
Confirm that \datatype is consistent with the datatype of $ A $.
}
\notes{
This function is similar to {\tt FLA\_Check\_consistent\_object\_datatype()}
except that it takes one datatype value and one object as its arguments
instead of two objects.
}
\rvalue{
\flasuccess if \datatype is equal to the datatype of $ A $;
\flainconsistentdatatypes otherwise.
}
\end{flaspec}

% --- FLA_Check_identical_object_precision() -----------------------------------

\begin{flaspec}
\begin{verbatim}
FLA_Error FLA_Check_identical_object_precision( FLA_Obj A, FLA_Obj B );
\end{verbatim}
\purpose{
Confirm that the numerical precision of the datatype of $ A $ matches the
numerical precision of the datatype of $ B $.
}
\notes{
The function first verifies that both $ A $ and $ B $ are floating point
objects.
If one or both objects are not floating point, \flaobjectnotfloatingpoint
is returned.
}
\rvalue{
\flasuccess if the datatype precision of $ A $ is equal to the datatype
precision of $ B $;
\flainconsistentobjectprecision otherwise.
}
%\begin{params}
%\parameter{\flaobj}{A}{An \flaobj to check.}
%\parameter{\flaobj}{B}{An \flaobj to check.}
%\end{params}
\end{flaspec}

% --- FLA_Check_conj_and_datatype() --------------------------------------------

\begin{flaspec}
\begin{verbatim}
FLA_Error FLA_Check_conj_and_datatype( FLA_Conj conj, FLA_Obj A );
\end{verbatim}
\purpose{
Confirm, if $ A $ is real, that \conj is not \flaconjugatens.
}
\rvalue{
\flasuccess if the \conj argument is not in conflict with the complexness of $ A $;
\flainvalidconjgivendatatype otherwise.
}
\end{flaspec}

% --- FLA_Check_conj_trans_and_datatype() --------------------------------------

\begin{flaspec}
\begin{verbatim}
FLA_Error FLA_Check_conj_trans_and_datatype( FLA_Trans trans, FLA_Obj A );
\end{verbatim}
\purpose{
Confirm, if $ A $ is real, that \trans is neither \flaconjtranspose
nor \flaconjnotransposens.
}
\rvalue{
\flasuccess if the \trans argument is not in conflict with the complexness of $ A $;
\flainvalidtransgivendatatype otherwise.
}
\end{flaspec}




\subsection{Element types}

% --- FLA_Check_valid_elemtype() -----------------------------------------------

\begin{flaspec}
\begin{verbatim}
FLA_Error FLA_Check_valid_elemtype( FLA_Elemtype elemtype );
\end{verbatim}
\purpose{
Confirm that \elemtype is one of the following values defined for the
\flaelemtype type:
\flascalarns, \flamatrixns.
}
\rvalue{
\flasuccess if \elemtype is valid;
\flainvalidelemtype otherwise.
}
\end{flaspec}

% --- FLA_Check_object_scalar_elemtype() ---------------------------------------

\begin{flaspec}
\begin{verbatim}
FLA_Error FLA_Check_object_scalar_elemtype( FLA_Obj A );
\end{verbatim}
\purpose{
Confirm that the element type of $ A $ is \flascalarns.
}
\rvalue{
\flasuccess if the element type of $ A $ is \flascalarns;
\flaobjectnotscalarelemtype otherwise.
}
\end{flaspec}

% --- FLA_Check_object_matrix_elemtype() ---------------------------------------

\begin{flaspec}
\begin{verbatim}
FLA_Error FLA_Check_object_matrix_elemtype( FLA_Obj A );
\end{verbatim}
\purpose{
Confirm that the element type of $ A $ is \flamatrixns.
}
\rvalue{
\flasuccess if the element type of $ A $ is \flamatrixns;
\flaobjectnotmatrixelemtype otherwise.
}
\end{flaspec}




\subsection{Object dimensions}

% --- FLA_Check_square() -------------------------------------------------------

\begin{flaspec}
\begin{verbatim}
FLA_Error FLA_Check_square( FLA_Obj A );
\end{verbatim}
\purpose{
Confirm that $ A $ is square.
}
\rvalue{
\flasuccess if $ A $ is square;
\flaobjectnotsquare otherwise.
}
\end{flaspec}

% --- FLA_Check_if_scalar() ----------------------------------------------------

\begin{flaspec}
\begin{verbatim}
FLA_Error FLA_Check_if_scalar( FLA_Obj A );
\end{verbatim}
\purpose{
Confirm that $ A $ is a scalar (ie: that $ A $ is $ 1 \by 1 $).
}
\rvalue{
\flasuccess if $ A $ is a scalar;
\flaobjectnotscalar otherwise.
}
\end{flaspec}

% --- FLA_Check_if_vector() ----------------------------------------------------

\begin{flaspec}
\begin{verbatim}
FLA_Error FLA_Check_if_vector( FLA_Obj A );
\end{verbatim}
\purpose{
Confirm that $ A $ is a vector (ie: that $ A $ is $ n \by 1 $ or $ 1 \by n $
for $ n \ge 0 $).
}
\rvalue{
\flasuccess if $ A $ is a vector;
\flaobjectnotvector otherwise.
}
\end{flaspec}

% --- FLA_Check_conformal_dims() -----------------------------------------------

\begin{flaspec}
\begin{verbatim}
FLA_Error FLA_Check_conformal_dims( FLA_Trans trans, FLA_Obj A, FLA_Obj B );
\end{verbatim}
\purpose{
Confirm that $ A $ and $ B $ have conformal dimensions.
If \trans is \flatranspose or \flaconjtransposens, then the function
confirms that $ A $ and $ B^T $ have conformal dimensions.
}
\rvalue{
\flasuccess if $ A $ and $ B $ have conformal dimensions;
\flanonconformaldimensions otherwise.
}
\end{flaspec}

% --- FLA_Check_matrix_matrix_dims() -------------------------------------------

\begin{flaspec}
\begin{verbatim}
FLA_Error FLA_Check_matrix_matrix_dims( FLA_Trans transa, FLA_Trans transb, 
                                        FLA_Obj A, FLA_Obj B, FLA_Obj C );
\end{verbatim}
\purpose{
Confirm that $ A $, $ B $, and $ C $ have conformal dimensions suitable
for a matrix-matrix operation of the form $ C := C + AB $ where
$ C $ is $ m \by n $ and $ A $ and $ B $ are $ m \by k $ and $ k \by n $,
respectively, after optionally transpositions, per \transa and \transbns.
}
\rvalue{
\flasuccess if $ A $, $ B $, and $ C $ have conformal dimensions;
\flanonconformaldimensions otherwise.
}
\end{flaspec}

% --- FLA_Check_matrix_vector_dims() -------------------------------------------

\begin{flaspec}
\begin{verbatim}
FLA_Error FLA_Check_matrix_vector_dims( FLA_Trans trans, FLA_Obj A, FLA_Obj x, FLA_Obj y );
\end{verbatim}
\purpose{
Confirm that $ A $, $ x $, and $ y $ have conformal dimensions suitable
for a matrix-vector operation of the form $ y := y + Ax $ where
$ A $ is optionally transposed, per \transns.
}
\rvalue{
\flasuccess if $ A $, $ x $, and $ y $ have conformal dimensions;
\flanonconformaldimensions otherwise.
}
\end{flaspec}

% --- FLA_Check_equal_vector_lengths() -----------------------------------------

\begin{flaspec}
\begin{verbatim}
FLA_Error FLA_Check_equal_vector_lengths( FLA_Obj x, FLA_Obj y );
\end{verbatim}
\purpose{
Confirm that $ x $ and $ y $, which are assumed to be vectors, have equal
lengths.
}
\notes{
This function works as expected if one or both arguments are row vectors.
That is, ``length'' for the purposes of this function refers to the length
of the vector, not the number of rows in the object.
}
\rvalue{
\flasuccess if $ x $ and $ y $ have equal lengths;
\flaunequalvectorlengths otherwise.
}
\end{flaspec}

% --- FLA_Check_vector_length() ------------------------------------------------

\begin{flaspec}
\begin{verbatim}
FLA_Error FLA_Check_vector_length( FLA_Obj x, dim_t expected_length );
\end{verbatim}
\purpose{
Confirm that $ x $, which is assumed to be a column vector, is of length
{\tt expected\_length}.
}
\notes{
This function checks only the number of rows of $ x $.
Therefore, this function will not work as expected with row vectors.
}
\rvalue{
\flasuccess if the number of rows in $ x $ is {\tt expected\_length};
\flainvalidvectorlength otherwise.
}
\end{flaspec}

% --- FLA_Check_vector_length_min() --------------------------------------------

\begin{flaspec}
\begin{verbatim}
FLA_Error FLA_Check_vector_length_min( FLA_Obj x, dim_t min_length );
\end{verbatim}
\purpose{
Confirm that $ x $, which is assumed to be a column vector, is at least
{\tt min\_length} in length.
}
\notes{
This function checks only the number of rows of $ x $.
Therefore, this function will not work as expected with row vectors.
}
\rvalue{
\flasuccess if the number of rows in $ x $ is at least {\tt min\_length};
\flavectorlengthbelowmin otherwise.
}
\end{flaspec}

% --- FLA_Check_object_dims() --------------------------------------------------

\begin{flaspec}
\begin{verbatim}
FLA_Error FLA_Check_object_dims( FLA_Trans trans, dim_t m, dim_t n, FLA_Obj A );
\end{verbatim}
\purpose{
Confirm that matrix $ A $ is $ m \by n $.
If \trans is \flatranspose or \flaconjtransposens, then the function
will instead confirm that $ A $ is $ n \by m $.
}
\rvalue{
\flasuccess if the dimensions of $ A $ are identical to those specified;
\flaspecifiedobjdimmismatch otherwise.
}
\end{flaspec}

% --- FLA_Check_submatrix_dims_and_offset() ------------------------------------

\begin{flaspec}
\begin{verbatim}
FLA_Error FLA_Check_submatrix_dims_and_offset( int m, int n, int i, int j, FLA_Obj A );
\end{verbatim}
\purpose{
Confirm that the $ m \by n $ submatrix that has its top-left element located
at row-column offset $ (i,j) $ (indexed from zero) does not exceed the bounds
of matrix $ A $.
In other words, the following constraints are enforced:
\begin{itemize}
\item $ i \le m(A) $
\itemvsp
\item $ j \le n(A) $
\itemvsp
\item $ i + m \le m(A) $
\itemvsp
\item $ j + n \le n(A) $
\end{itemize}
where $ m( A ) $ and $ n( A ) $ denote the number or rows and number of
columns in $ A $, respectively.
}
\notes{
Strictly speaking, only the last two constraints are needed.
However, we first check against the first two constraints to allow us to
distinguish between situations where the offsets are invalid (in which case
the value of the matrix dimensions are moot), and situations where the
offsets are valid, but the submatrix dimensions places the submatrix beyond
the bounds of $ A $.
}
\rvalue{
\flasuccess if the specified submatrix is within the bounds of $ A $;
otherwise, \flainvalidsubmatrixdims if one of the first two contraints
is violated, and \flainvalidsubmatrixoffset if the first two constraints
are met but one of the last two constraints is violated.
}
\end{flaspec}

% --- FLA_Check_adjacent_objects_2x2() -----------------------------------------

\begin{flaspec}
\begin{verbatim}
FLA_Error FLA_Check_adjacent_objects_2x2( FLA_Obj ATL, FLA_Obj ATR,
                                          FLA_Obj ABL, FLA_Obj ABR );
\end{verbatim}
\purpose{
Confirm that views $ A_{TL} $, $ A_{TR} $, $ A_{BL} $, and $ A_{BR} $ are
vertically and horizontally aligned and adjacent, and that views that are
vertically or horizontally adjacent have dimensions appropriately
matched for them to form quadrants of a single larger view of the
base object.
}
\rvalue{
\flasuccess if the four object views are aligned and adjacent, and have
matching dimensions;
otherwise, one of the following, depending on the mismatch:
\begin{itemize}
\item \flaobjectsnotverticallyadj
\itemvsp
\item \flaobjectsnotverticallyaligned
\itemvsp
\item \flaobjectsnothorizontallyadj
\itemvsp
\item \flaobjectsnothorizontallyaligned
\itemvsp
\item \flaadjacentobjectdimmismatch
\end{itemize}
}
\end{flaspec}

% --- FLA_Check_adjacent_objects_2x1() -----------------------------------------

\begin{flaspec}
\begin{verbatim}
FLA_Error FLA_Check_adjacent_objects_2x1( FLA_Obj AT,
                                          FLA_Obj AB );
\end{verbatim}
\purpose{
Confirm that views $ A_T $ and $ A_B $ are vertically
and aligned and adjacent, and that the views
have column dimensions appropriately
matched for them to form a vertical panel a single larger view of the
base object.
}
\rvalue{
\flasuccess if the two object views are vertically aligned and adjacent,
and have matching column dimensions;
otherwise, one of the following, depending on the mismatch:
\begin{itemize}
\item \flaobjectsnotverticallyadj
\itemvsp
\item \flaobjectsnotverticallyaligned
\itemvsp
\item \flaadjacentobjectdimmismatch
\end{itemize}
}
\end{flaspec}

% --- FLA_Check_adjacent_objects_1x2() -----------------------------------------

\begin{flaspec}
\begin{verbatim}
FLA_Error FLA_Check_adjacent_objects_1x2( FLA_Obj AL, FLA_Obj AR );
\end{verbatim}
\purpose{
Confirm that views $ A_L $ and $ A_R $ are horizontally
and aligned and adjacent, and that the views
have row dimensions appropriately
matched for them to form a horizontal panel a single larger view of the
base object.
}
\rvalue{
\flasuccess if the two object views are horizontally aligned and adjacent,
and have matching row dimensions;
otherwise, one of the following, depending on the mismatch:
\begin{itemize}
\item \flaobjectsnothorizontallyadj
\itemvsp
\item \flaobjectsnothorizontallyaligned
\itemvsp
\item \flaadjacentobjectdimmismatch
\end{itemize}
}
\end{flaspec}




\subsection{UNIX file I/O}

% --- FLA_Check_file_descriptor() ----------------------------------------------

\begin{flaspec}
\begin{verbatim}
FLA_Error FLA_Check_file_descriptor( int fd );
\end{verbatim}
\purpose{
Confirm that \fdns, which is assumped to have been returned from
the UNIX/Linux {\tt open()} function, is a valid file descriptor.
}
\notes{
The UNIX/Linux {\tt open()} function returns $ -1 $ when it fails to
successfully open a file.
}
\rvalue{
\flasuccess if \fd is valid;
\flaopenreturnederror otherwise.
}
\end{flaspec}

% --- FLA_Check_close_result() -------------------------------------------------

\begin{flaspec}
\begin{verbatim}
FLA_Error FLA_Check_close_result( int r_val );
\end{verbatim}
\purpose{
Confirm that \rvalns, which is assumed to have been returned
from {\tt close()}, does not indicate that an error has occurred.
}
\notes{
The UNIX/Linux {\tt close()} function returns $ -1 $ when it fails to
successfully close a file.
}
\rvalue{
\flasuccess if \rval indicates no error;
\flaclosereturnederror otherwise.
}
\end{flaspec}

% --- FLA_Check_lseek_result() -------------------------------------------------

\begin{flaspec}
\begin{verbatim}
FLA_Error FLA_Check_lseek_result( off_t requested_offset, int r_val );
\end{verbatim}
\purpose{
Confirm that \rvalns, which is assumed to have been returned
from {\tt lseek()}, does not indicate that an error has occurred.
It is further assumed that {\tt requested\_offset} was the byte offset
passed to {\tt lseek()} when {\tt lseek()} returned \rvalns.
}
\notes{
The UNIX/Linux {\tt lseek()} function returns the resulting byte offset
relative to the beginning of the file.
If {\tt lseek()} is unsuccessful, it returns $ -1 $.
}
\rvalue{
\flasuccess if \rval indicates no error;
\flalseekreturnederror otherwise.
}
\end{flaspec}

% --- FLA_Check_read_result() --------------------------------------------------

\begin{flaspec}
\begin{verbatim}
FLA_Error FLA_Check_read_result( size_t requested_size, ssize_t r_val );
\end{verbatim}
\purpose{
Confirm that \rvalns, which is assumed to have been returned
from {\tt read()}, does not indicate that an error has occurred.
}
\notes{
Under normal circumstances, the UNIX/Linux {\tt read()} function
returns the number of bytes successfully read into the destination
buffer, where zero indicates no bytes were read due to attempting to
read at end of file.
It is not considered an error for this return value to be less than
the requested number of bytes.
If an actual error occurs, {\tt read()} returns $ -1 $.
Currently {\tt FLA\_Check\_read\_result()} only returns an error
code if {\tt read()} returns $ -1 $.
Thus, the {\tt requested\_size} argument is not referenced, but may be
used in the future to provide warnings.
}
\rvalue{
\flasuccess if \rval indicates no error;
\flareadreturnederror otherwise.
}
\end{flaspec}

% --- FLA_Check_write_result() -------------------------------------------------

\begin{flaspec}
\begin{verbatim}
FLA_Error FLA_Check_write_result( size_t requested_size, ssize_t r_val );
\end{verbatim}
\purpose{
Confirm that \rvalns, which is assumed to have been returned
from {\tt write()}, does not indicate that an error has occurred.
}
\notes{
Under normal circumstances, the UNIX/Linux {\tt write()} function
returns the number of bytes successfully written from the source
buffer, where zero indicates no bytes were written.
It is not considered an error for this return value to be less than
the requested number of bytes.
If an actual error occurs, {\tt write()} returns $ -1 $.
Currently {\tt FLA\_Check\_write\_result()} only returns an error
code if {\tt write()} returns $ -1 $.
Thus, the {\tt requested\_size} argument is not referenced, but may be
used in the future to provide warnings.
}
\rvalue{
\flasuccess if \rval indicates no error;
\flawritereturnederror otherwise.
}
\end{flaspec}

% --- FLA_Check_unlink_result() ------------------------------------------------

\begin{flaspec}
\begin{verbatim}
FLA_Error FLA_Check_unlink_result( int r_val );
\end{verbatim}
\purpose{
Confirm that \rvalns, which is assumed to have been returned
from {\tt unlink()}, does not indicate that an error has occurred.
}
\notes{
The UNIX/Linux {\tt unlink()} function returns $ -1 $ when it fails to
successfully delete a file from the filesystem.
}
\rvalue{
\flasuccess if \rval indicates no error;
\flaunlinkreturnederror otherwise.
}
\end{flaspec}




\subsection{Operation-specific errors}

% --- FLA_Check_chol_failure() -------------------------------------------------

\begin{flaspec}
\begin{verbatim}
FLA_Error FLA_Check_chol_failure( FLA_Error r_val );
\end{verbatim}
\purpose{
Confirm that \rvalns, which is assumped to have been
returned from the an LAPACK-compatible Cholesky factorization routine,
does not indicate that the input matrix was found to be non-symmetric
positive definite (or non-Hermitian positive definite).
}
\notes{
For the purposes of this function, ``LAPACK-compatible'' means the
Cholesky factorization routine returns the row/column offset (indexing
from one) of the diagonal entry that was found to be negative.
}
\rvalue{
\flasuccess if \rval is valid;
\flacholfailedmatrixnotspd otherwise.
}
\end{flaspec}

% --- FLA_FLA_Check_block_householder_transform() ------------------------------

\begin{flaspec}
\begin{verbatim}
FLA_Error FLA_Check_block_householder_transform( FLA_Store storev,
                                                 FLA_Obj A, FLA_Obj S );
\end{verbatim}
\purpose{
Confirm that matrix the dimensions of $ A $ are compatible with the dimensions
of a block Householder matrix $ S $.
Specifically, if \storev is \flacolumnwisens, then the number of columns of
$ A $ must match the order of $ S $.
Otherwise, if \storev is \flarowwisens, then the number of rows of $ A $
must match the order of $ S $.
}
\rvalue{
\flasuccess if the dimensions of $ A $ and $ S $ are compatible;
\flablockhousehdimmismatch otherwise.
}
\end{flaspec}

% --- FLA_Check_hess_indices() -------------------------------------------------
%
%\begin{flaspec}
%\begin{verbatim}
%FLA_Error FLA_Check_hess_indices( FLA_Obj A, int ilo, int ihi );
%\end{verbatim}
%\purpose{
%Confirm that the indices \ilo and \ihi are reasonable values for a
%reduction to upper Hessenberg operation.
%}
%\notes{
%Any of the following conditions will cause the function to return
%an \flainvalidhessenbergindices error value:
%\begin{itemize}
%\item $ n( A ) = 0 $ and $ \ilo \neq 1 $ and $ \ihi \neq 1 $
%\itemvsp
%\item $ \ilo < 1 $ or $ n( A ) < \ilo $
%\itemvsp
%\item $ \ihi < 1 $ or $ n( A ) < \ihi $
%\itemvsp
%\item $ \ihi < \ilo $
%\end{itemize}
%where $ n( A ) $ denotes the number of columns in matrix $ A $.
%}
%\rvalue{
%\flasuccess if \ilo and \ihi are valid indices for a reduction to upper Hessenberg operation;
%\flainvalidhessenbergindices otherwise.
%}
%\end{flaspec}

% --- FLA_Check_sylv_matrix_dims() ---------------------------------------------

\begin{flaspec}
\begin{verbatim}
FLA_Error FLA_Check_sylv_matrix_dims( FLA_Obj A, FLA_Obj B, FLA_Obj C );
\end{verbatim}
\purpose{
Confirm that $ A $, $ B $, and $ C $ have conformal dimensions suitable
for a triangular Sylvester equation solve of the form $ AX + XB = C $ where
$ A $ and $ B $ are $ m \by m $ and $ n \by n $, respectively, and $ X $ and
$ C $ are $ m \by n $.
}
\rvalue{
\flasuccess if $ A $, $ B $, and $ C $ have conformal dimensions;
\flanonconformaldimensions otherwise.
}
\end{flaspec}

% --- FLA_Check_valid_isgn_value() ---------------------------------------------

\begin{flaspec}
\begin{verbatim}
FLA_Error FLA_Check_valid_isgn_value( FLA_Obj isgn );
\end{verbatim}
\purpose{
Confirm that \isgn is either \flaone or \flaminusonens.
}
\notes{
This function currently compares \isgn against \flaone and \flaminusone
with the function {\tt FLA\_Obj\_is()}, which creates a stronger constraint
than just comparing the {\em values} contained within the objects.
}
\rvalue{
\flasuccess if \isgn is valid;
\flainvalidside otherwise.
}
\end{flaspec}




\subsection{Other system errors}

% --- FLA_Check_pthread_create_result() ----------------------------------------

\begin{flaspec}
\begin{verbatim}
FLA_Error FLA_Check_pthread_create_result( int r_val );
\end{verbatim}
\purpose{
Confirm that \rvalns, which is assumped to have been
returned from the POSIX {\tt pthread\_create()} function, does not indicate
that an error has occurred.
}
\rvalue{
\flasuccess if \rval indicates no error;
\flapthreadcreatereturnederror otherwise.
}
\end{flaspec}

% --- FLA_Check_pthread_join_result() ------------------------------------------

\begin{flaspec}
\begin{verbatim}
FLA_Error FLA_Check_pthread_join_result( int r_val );
\end{verbatim}
\purpose{
Confirm that \rvalns, which is assumped to have been
returned from the POSIX {\tt pthread\_join()} function, does not indicate
that an error has occurred.
}
\rvalue{
\flasuccess if \rval indicates no error;
\flapthreadjoinreturnederror otherwise.
}
\end{flaspec}

% --- FLA_Check_malloc_pointer() -----------------------------------------------

\begin{flaspec}
\begin{verbatim}
FLA_Error FLA_Check_malloc_pointer( void* ptr );
\end{verbatim}
\purpose{
Confirm that \ptrns, which is assumped to have been returned by a memory
allocation function such as {\tt malloc()}, is not a \fnull pointer.
}
\notes{
This routine is similar in behavior to {\tt FLA\_Check\_null\_pointer()}.
This only difference is that this function is used specifically to check
the validity of pointers returned by {\tt malloc()}, and thus is set up
to return a {\tt malloc()}-specific error message.
}
\rvalue{
\flasuccess if \ptr is not \fnullns;
\flamallocreturnednullpointer otherwise.
}
\end{flaspec}

% --- FLA_Check_posix_memalign_failure() ---------------------------------------

\begin{flaspec}
\begin{verbatim}
FLA_Error FLA_Check_posix_memalign_failure( int r_val );
\end{verbatim}
\purpose{
Confirm that \rvalns, which is assumped to have been returned by
the POSIX {\tt posix\_memalign()} function, does not indicate that an error
has occurred.
}
\notes{
Unlike {\tt malloc()}, {\tt posix\_memalign()} returns an integer
value to indicate success (zero) or failure (a non-zero value), and the
pointer to the requested memory region is obtained from the function by
providing the pointer's address as an argument, which allows
{\tt posix\_memalign()} to set the pointer value directly.
}
\rvalue{
\flasuccess if \rval is valid;
\flaposixmemalignfailed otherwise.
}
\end{flaspec}




\subsection{Misc. errors}

% --- FLA_Check_valid_pivot_type() ---------------------------------------------

\begin{flaspec}
\begin{verbatim}
FLA_Error FLA_Check_valid_pivot_type( FLA_Pivot_type ptype );
\end{verbatim}
\purpose{
Confirm that \ptype is one of the following values defined for the
\flapivottype type:
\flanativepivotsns, \flalapackpivotsns.
}
\rvalue{
\flasuccess if \ptype is valid;
\flainvalidconj otherwise.
}
\end{flaspec}

% --- FLA_Check_pivot_vector_length() ------------------------------------------

\begin{flaspec}
\begin{verbatim}
FLA_Error FLA_Check_pivot_vector_length( FLA_Obj ipiv );
\end{verbatim}
\purpose{
Confirm that the number of rows in \ipiv is less than or equal to
the current length of the interal pivoting work buffer.
If the length of \ipiv exceeds the current length of the work buffer,
then the work buffer is reallocated to the length of \ipivns.
The length of this internal buffer begins as
{\tt FLA\_MAX\_LU\_PIVOT\_LENGTH}.
If a reallocation takes place, the function confirms that the
reallocated pointer is not \fnullns.
}
\rvalue{
\flasuccess if no reallocation was needed, or if a reallocation took place
and the returned pointer is valid;
\flanullpointer otherwise.
}
\end{flaspec}

% --- FLA_Check_divide_by_zero() -----------------------------------------------

\begin{flaspec}
\begin{verbatim}
FLA_Error FLA_Check_divide_by_zero( FLA_Obj alpha );
\end{verbatim}
\purpose{
Confirm that $ \alpha $, which is assumed to be a potential denominator
in a future floating point division operation, is non-zero.
}
\rvalue{
\flasuccess if $ \alpha $ is non-zero;
\fladividebyzero otherwise.
}
\end{flaspec}

% --- FLA_Check_blocksize_value() ----------------------------------------------

\begin{flaspec}
\begin{verbatim}
FLA_Error FLA_Check_blocksize_value( dim_t b );
\end{verbatim}
\purpose{
Confirm that $ b $ is a valid blocksize.
}
\notes{
Since \dimt is a type of unsigned integer, the only invalid value that $ b $
might take on is zero.
Thus, this function simply confirms that $ b $ is non-zero.
}
\rvalue{
\flasuccess if $ b $ is valid (non-zero);
\flainvalidblocksizevalue otherwise.
}
\end{flaspec}

% --- FLA_Check_blocksize_object() ---------------------------------------------

\begin{flaspec}
\begin{verbatim}
FLA_Error FLA_Check_blocksize_object( FLA_Datatype datatype, fla_blocksize_t* bp );
\end{verbatim}
\purpose{
Confirm that the blocksize field associated with \datatype that resides
within the structure pointed to by \bp is valid.
}
\notes{
Similar to {\tt FLA\_Check\_blocksize\_value()}, this function only confirms
that the blocksize value is non-zero.
}
\rvalue{
\flasuccess if the blocksize field associated with \datatype is valid
(non-zero);
\flainvalidblocksizevalue otherwise.
}
\end{flaspec}

% --- FLA_Check_null_pointer() -------------------------------------------------

\begin{flaspec}
\begin{verbatim}
FLA_Error FLA_Check_null_pointer( void* ptr );
\end{verbatim}
\purpose{
Confirm that \ptr is not a \fnull pointer.
}
\rvalue{
\flasuccess if \ptr is not \fnullns;
\flanullpointer otherwise.
}
\end{flaspec}

% --- FLA_Check_base_buffer_mismatch() -----------------------------------------

\begin{flaspec}
\begin{verbatim}
FLA_Error FLA_Check_base_buffer_mismatch( FLA_Obj A, FLA_Obj B );
\end{verbatim}
\purpose{
Confirm that views $ A $ and $ B $ refer to the same underlying
object.
}
\notes{
This check is performed by comparing the addresses of the views' base
objects.
}
\rvalue{
\flasuccess if $ A $ and $ B $ refer to the same underlying object;
\flaobjectbasebuffermismatch otherwise.
}
\end{flaspec}

% --- FLA_Check_num_threads() --------------------------------------------------

\begin{flaspec}
\begin{verbatim}
FLA_Error FLA_Check_num_threads( unsigned int n_threads );
\end{verbatim}
\purpose{
Confirm that \nthreads is at least one.
}
\notes{
Since \nthreads is declared as an {\tt unsigned int}, the only
invalid value that \nthreads may take on is zero.
}
\rvalue{
\flasuccess if \nthreads is at least one;
\flaencounterednonpositiventhreads otherwise.
}
\end{flaspec}


