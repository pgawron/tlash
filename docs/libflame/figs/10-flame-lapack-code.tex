\begin{figure}[tbp]
\begin{center}
\begin{tabular}{|c|c|}
\hline
% --------------------------------------
libflame & LAPACK \\ \hline
% --------------------------------------
% & \\
% --------------------------------------
\begin{minipage}[t]{3in}
{\tiny
\begin{verbatim}
FLA_Error FLA_Chol_l_blk_var2( FLA_Obj A, dim_t nb_alg )
{
  FLA_Obj ATL,   ATR,      A00, A01, A02,
          ABL,   ABR,      A10, A11, A12,
                           A20, A21, A22;
  dim_t b;
  int value;

  FLA_Part_2x2( A,    &ATL, &ATR,
                      &ABL, &ABR,     0, 0, FLA_TL );

  while ( FLA_Obj_length( ATL ) < FLA_Obj_length( A ) )
  {
    b = min( FLA_Obj_length( ABR ), nb_alg );

    FLA_Repart_2x2_to_3x3( ATL, /**/ ATR,       &A00, /**/ &A01, &A02,
                        /* ************* */   /* ******************** */
                                                &A10, /**/ &A11, &A12,
                           ABL, /**/ ABR,       &A20, /**/ &A21, &A22,
                           b, b, FLA_BR );

    /* ---------------------------------------------------------------- */

    FLA_Syrk( FLA_LOWER_TRIANGULAR, FLA_NO_TRANSPOSE, 
              FLA_MINUS_ONE, A10, FLA_ONE, A11 );

    FLA_Gemm( FLA_NO_TRANSPOSE, FLA_TRANSPOSE, 
              FLA_MINUS_ONE, A20, A10, FLA_ONE, A21 );

    value = FLA_Chol_unb_external( FLA_LOWER_TRIANGULAR, A11 );

    if ( value != FLA_SUCCESS )
      return ( FLA_Obj_length( A00 ) + value );

    FLA_Trsm( FLA_RIGHT, FLA_LOWER_TRIANGULAR, 
              FLA_TRANSPOSE, FLA_NONUNIT_DIAG, 
              FLA_ONE, A11, A21 );

    /* ---------------------------------------------------------------- */

    FLA_Cont_with_3x3_to_2x2( &ATL, /**/ &ATR,       A00, A01, /**/ A02,
                                                     A10, A11, /**/ A12,
                            /* ************** */  /* ****************** */
                              &ABL, /**/ &ABR,       A20, A21, /**/ A22,
                              FLA_TL );
  }

  return value;
}
\end{verbatim}
}
\end{minipage}
&
\begin{minipage}[t]{3in}
{\tt \tiny
\begin{verbatim}
      SUBROUTINE DPOTRF( UPLO, N, A, LDA, INFO )

      CHARACTER          UPLO
      INTEGER            INFO, LDA, N
      DOUBLE PRECISION   A( LDA, * )

      DOUBLE PRECISION   ONE
      PARAMETER          ( ONE = 1.0D+0 )
      LOGICAL            UPPER
      INTEGER            J, JB, NB
      LOGICAL            LSAME
      INTEGER            ILAENV
      EXTERNAL           LSAME, ILAENV
      EXTERNAL           DGEMM, DPOTF2, DSYRK, DTRSM, XERBLA
      INTRINSIC          MAX, MIN

      INFO = 0
      UPPER = LSAME( UPLO, 'U' )
      IF( .NOT.UPPER .AND. .NOT.LSAME( UPLO, 'L' ) ) THEN
         INFO = -1
      ELSE IF( N.LT.0 ) THEN
         INFO = -2
      ELSE IF( LDA.LT.MAX( 1, N ) ) THEN
         INFO = -4
      END IF
      IF( INFO.NE.0 ) THEN
         CALL XERBLA( 'DPOTRF', -INFO )
         RETURN
      END IF

      INFO = 0
      UPPER = LSAME( UPLO, 'U' )

      IF( N.EQ.0 )
     $   RETURN

      NB = ILAENV( 1, 'DPOTRF', UPLO, N, -1, -1, -1 )
      IF( NB.LE.1 .OR. NB.GE.N ) THEN
         CALL DPOTF2( UPLO, N, A, LDA, INFO )
      ELSE
         IF( UPPER ) THEN    
*********** Upper triangular case omited for purposes of fair comparison.
         ELSE
            DO 20 J = 1, N, NB
               JB = MIN( NB, N-J+1 )
               CALL DSYRK( 'Lower', 'No transpose', JB, J-1, -ONE,
     $                     A( J, 1 ), LDA, ONE, A( J, J ), LDA )
               CALL DPOTF2( 'Lower', JB, A( J, J ), LDA, INFO )
               IF( INFO.NE.0 )
     $            GO TO 30
               IF( J+JB.LE.N ) THEN
                  CALL DGEMM( 'No transpose', 'Transpose', N-J-JB+1, JB,
     $                        J-1, -ONE, A( J+JB, 1 ), LDA, A( J, 1 ),
     $                        LDA, ONE, A( J+JB, J ), LDA )
                  CALL DTRSM( 'Right', 'Lower', 'Transpose', 'Non-unit',
     $                        N-J-JB+1, JB, ONE, A( J, J ), LDA,
     $                        A( J+JB, J ), LDA )
               END IF
   20       CONTINUE
         END IF
      END IF
      GO TO 40
   30 CONTINUE
      INFO = INFO + J - 1
   40 CONTINUE
      RETURN
      END
\end{verbatim}
}
\end{minipage}
\\
% --------------------------------------
 & \\ \hline
% --------------------------------------
\end{tabular}
\end{center}
\caption{
%The algorithm shown in Figure \ref{fig:chol-alg} implemented with
%FLAME/C code (left), representing the
%style of coding found in libflame, and Fortran-77 code (right)
%obtained from LAPACK. 
The algorithm shown in Figure \ref{fig:chol-alg} implemented with
FLAME/C code (left) and Fortran-77 code (right).
The FLAME/C code represents the style of coding found in libflame
while the Fortran-77 code was obtained from LAPACK.
}
\label{fig:flame-lapack-code}
\end{figure}
