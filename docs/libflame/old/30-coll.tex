\chapter{Collective Communication}
\label{chapter:coll}


\begin{figure*}[p]
\begin{center}
\begin{tabular}{ | p{0.5in} I c | c |} \hline
{\footnotesize Operation}
& Before & After \\ \whline
& & \\
{\footnotesize Broadcast} &
{\footnotesize
$
\begin{array}{ c | c | c | c }
\mbox{Node 0 } &
\mbox{Node 1 } &
\mbox{Node 2 } &
\mbox{Node 3 } \\ \hline
x & 
& 
& 
\end{array}
$
}
&
{\footnotesize
$
\begin{array}{ c | c | c | c }
\mbox{Node 0 } &
\mbox{Node 1 } &
\mbox{Node 2 } &
\mbox{Node 3 } \\ \hline
x & x & x & x 
\end{array}
$
}
\\ 
& & \\ \hline
& & \\
{\footnotesize Reduce(-to-one)}
&
{\footnotesize
$
\begin{array}{ c | c | c | c }
\mbox{Node 0 } &
\mbox{Node 1 } &
\mbox{Node 2 } &
\mbox{Node 3 } \\ \hline
x^{(0)} & 
x^{(1)} & 
x^{(2)} & 
x^{(3)}
\end{array}
$
}
&
{\footnotesize
$
\begin{array}{ c | c | c | c }
\mbox{Node 0 } &
\mbox{Node 1 } &
\mbox{Node 2 } &
\mbox{Node 3 } \\ \hline
\sum_j x^{(j)} & & & \\
\end{array}
$
}
\\ 
& & \\ \whline
& & \\
{\footnotesize Scatter} &
{\footnotesize
$
\begin{array}{ c | c | c | c }
\mbox{Node 0 } &
\mbox{Node 1 } &
\mbox{Node 2 } &
\mbox{Node 3 } \\ \hline
x_0 & 
& 
& 
\\
x_1 & 
& 
& 
\\
x_2 & 
& 
& 
\\
x_3 & 
& 
& 
\end{array}
$
}
&
{\footnotesize
$
\begin{array}{ c | c | c | c }
\mbox{Node 0 } &
\mbox{Node 1 } &
\mbox{Node 2 } &
\mbox{Node 3 } \\ \hline
x_0 & 
& 
& 
\\
& 
x_1 & 
& 
\\
& 
& 
x_2 & 
\\
& 
& 
& 
x_3
\end{array}
$
}
\\ 
& & \\ \hline
& & \\
{\footnotesize Gather} &
{\footnotesize
$
\begin{array}{ c | c | c | c }
\mbox{Node 0 } &
\mbox{Node 1 } &
\mbox{Node 2 } &
\mbox{Node 3 } \\ \hline
x_0 & 
& 
& 
\\
& 
x_1 & 
& 
\\
& 
& 
x_2 & 
\\
& 
& 
& 
x_3
\end{array}
$
}
&
{\footnotesize
$
\begin{array}{ c | c | c | c }
\mbox{Node 0 } &
\mbox{Node 1 } &
\mbox{Node 2 } &
\mbox{Node 3 } \\ \hline

x_0 & 
& 
& 
\\
x_1 & 
& 
& 
\\
x_2 & 
& 
& 
\\
x_3 & 
& 
& 
\end{array}
$
}
\\ 
& & \\ \whline
& & \\
{\footnotesize Allgather} &
{\footnotesize
$
\begin{array}{ c | c | c | c }
\mbox{Node 0 } &
\mbox{Node 1 } &
\mbox{Node 2 } &
\mbox{Node 3 } \\ \hline
x_0 & 
& 
& 
 \\
& 
x_1 & 
& 
\\
& 
& 
x_2 & 
\\
& 
& 
& 
x_3 
\end{array}
$
}
&
{\footnotesize
$
\begin{array}{ c | c | c | c }
\mbox{Node 0 } &
\mbox{Node 1 } &
\mbox{Node 2 } &
\mbox{Node 3 } \\ \hline
x_0 & 
x_0 & 
x_0 & 
x_0 \\
x_1 & 
x_1 & 
x_1 & 
x_1 \\
x_2 & 
x_2 & 
x_2 & 
x_2 \\
x_3 & 
x_3 & 
x_3 & 
x_3 
\end{array}
$
}
\\ 
& & \\ \hline
& & \\
{\footnotesize Reduce-scatter} &
{\footnotesize
$
\begin{array}{ c | c | c | c }
\mbox{Node 0 } &
\mbox{Node 1 } &
\mbox{Node 2 } &
\mbox{Node 3 } \\ \hline
x_0^{(0)} & 
x_0^{(1)} & 
x_0^{(2)} & 
x_0^{(3)} \\
x_1^{(0)} & 
x_1^{(1)} & 
x_1^{(2)} & 
x_1^{(3)} \\
x_2^{(0)} & 
x_2^{(1)} & 
x_2^{(2)} & 
x_2^{(3)} \\
x_3^{(0)} & 
x_3^{(1)} & 
x_3^{(2)} & 
x_3^{(3)} 
\end{array}
$
}
&
{\footnotesize
$
\begin{array}{ c | c | c | c }
\mbox{Node 0 } &
\mbox{Node 1 } &
\mbox{Node 2 } &
\mbox{Node 3 } \\ \hline
\sum_j x_0^{(j)} & & & \\
& \sum_j x_1^{(j)} & &  \\
& & \sum_j x_2^{(j)} & \\
& & & \sum_j x_3^{(j)}  \\
\end{array}
$
}
\\ 
& & \\ \whline
& & \\
{\footnotesize Allreduce} &
{\footnotesize
$
\begin{array}{ c | c | c | c }
\mbox{Node 0 } &
\mbox{Node 1 } &
\mbox{Node 2 } &
\mbox{Node 3 } \\ \hline
x^{(0)} & 
x^{(1)} & 
x^{(2)} & 
x^{(3)} 
\end{array}
$
}
&
{\footnotesize
$
\begin{array}{ c | c | c | c }
\mbox{Node 0 } &
\mbox{Node 1 } &
\mbox{Node 2 } &
\mbox{Node 3 } \\ \hline
\sum_j x^{(j)}&\sum_j x^{(j)}&\sum_j x^{(j)}&\sum_j x^{(j)} 
\end{array}
$
}
\\ 
& & \\ \hline
\end{tabular}
\end{center}
\caption{Collective communications considered in this paper.}
\label{fig:operations}
\end{figure*}


When the nodes of a distributed-memory architecture collaborate
to solve a given problem, inherently computation previously
performed on a single node is now distributed among the nodes,
and communication is performed when data is shared, or contributions from
different nodes must be consolidated.
Communication operations that simultaneously involve a group of nodes
are called {\em collective communication} operations.
In our discussions, we will assume that the group includes all nodes.
\plapack makes heavy use of collective communication,
so its efficient implementation is crucial to attaining 
scalable and high performance.

The most typically encountered collective communications, discussed
in this section, fall into two categories:
\begin{itemize}
\item
{\bf Data redistribution operations:}
broadcast, scatter, gather, and allgather.
These operations move data between processors.
\item
{\bf Data consolidation operations:}
reduce(-to-one), reduce-scatter, and allreduce.
These operations consolidate contributions from different processors
by applying a reduction operation.
We will only consider reduction operations that are both commutative
and associative.
\end{itemize}
The operations discussed are illustrated in Fig.~\ref{fig:operations}.
In that figure, $ x $ indicates a vector of data of length $ n $.
For some operations, $ x $ is subdivided into subvectors
$ x_i $, $ i = 0, \cdots, p-1 $, where $ p $ equals the number of nodes.
A superscript is used to indicate a vector that must be reduced
with other vectors from other nodes.  $ \sum_{j} x^{(j)} $ indicates the
result of that reduction.
The summation sign is used because summation is the
most commonly encountered reduction operation.

We present these collective communications as pairs of
{\em dual} operation.  We will show later that an implementation
of one of the dual operations can be transformed into that of the other dual
operation by reversing the communication (and adding to or deleting
reduction operations from the implementation).  These dual pairs are indicated
by the groupings in Fig.~\ref{fig:operations} (separated by the thick lines):
{broadcast and reduce(-to-one)},
{scatter and gather}, and
{allgather and reduce-scatter}.
{Allreduce} is the only operation that does not have a dual
(or it can be viewed as its own dual).
